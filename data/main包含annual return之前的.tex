\documentclass{article}
\usepackage{amsmath}
\usepackage{enumitem}
\usepackage{authblk}
\usepackage{comment}
\usepackage{multirow}
\usepackage[utf8]{inputenc}
\usepackage{natbib}
\usepackage{graphicx} % Required for inserting images
\usepackage[table]{xcolor}
\usepackage{colortbl}


\title{Comparing Technical Analysis-based Trading Strategies by Analogy to Machine Learning}
\author{Xiang Zhang and Eugene Pinsky\footnote{Correposning author: Eugene Pinsky, epinsky@bu.edu}}
\affil{\small Department of Computer Science, Met College, 
           Boston University, Boston, MA\\    
          email: xz0224@bu.edu, epinsky@bu.edu}

% \affil[1]{Department of Computer Science, Met College, Boston University}
\date{\today}

\begin{document}
\maketitle


\begin{abstract}
We are given a choice to invest either in the growth or the value index for the next day. We suggest and compare some simple algorithms based on the winning or losing index over the last $k$ days. 
We find that for fixed $k$, choosing the losing index
is better than choosing the winning index in terms of growth, maximum drawdown, and annualized volatility. We find that $k=3$ gives us the best performance suggesting that the market has limited memory. 
\end{abstract}

\medskip
\noindent
{\bf Keywords: } Technical analysis, algorithmic trading, machine learning, nearest-neighbor classification, 
trading signals, return efficiency index
%%\pacs[JEL Classification]{D8, H51}

\section{Introduction}\label{section_introduction}

In the fundamental analysis of stock pricing, we focus on financial statements and economic statistics to predict the intrinsic value. By contrast, in the technical analysis, we examine historical market data to find patterns to predict future asset price movements and design trading strategies accordingly. This is very similar to what we do in machine learning: use existing data to find patterns that help us classify or predict unseen data patterns. Not surprisingly, in recent years there has been an increased interest in using machine learning to design new algorithms or improve the existing ones using machine learning techniques.
 
These include
combining technical indicators with machine learning models~\cite{tech-Ayala2021,tech-Beg2019},
deep learning~\cite{tech-Buachuen2023,chen2023,tech-Chen2020,tech-Dash2016,tech-Gu2018,tech-Agrawal2019,  
tech-Mahfooz2022,tech-Park2019,tech-Pholsri2023,tech-Wang2018, 
tech-yao2022stock, tech-Yu2023, tech-Zhang2021, tech-Zhong2019}, natural language processing~\cite{tech-Meesad2023}, ML-based FOREX trading~\cite{tech-Gerlein2016},
trend prediction~\cite{tech-Grigoryan2017, tech-Jagadisha2022, 
 tech-Nousi2018, tech-Ntakaris2018, tech-Oyewola2019,tech-Pradip2018}, 
sentiment analysis~\cite{, tech-BK2023, tech-Joiner2022, tech-Mndawe2022},
re-enforcement learning~\cite{tech-Tsantekidis2020,tech-Van2023,tech-Zarkias2019},
ensemble methods~\cite{tech-Chavarnakul2009,tech-Choudhry2008,tech-Kim2018,tech-Pasupulety2019},
portfolio construction~\cite{, tech-Ndikum2020,tech-Padhi2022}, risk management~\cite{tech-Karthik2023},
high-frequency and short-term trading~\cite{tech-Bitvai2015,tech-Wong2023}, to name just some.
A comparison of some machine learning methods in technical analysis is presented in ~\cite{tech-Hsu2016, tech-khan2023performance, tech-Lumoring2023,tech-Patel2015}.

In these models, the effectiveness of using machine learning is 
evaluated using the standard machine learning metrics
(like accuracy, confusion matrix, etc.) and the effectiveness 
of trading is evaluated using the financial metrics
(like return, volatility). We propose a unified approach 
to evaluate strategies using machine learning. We derive
(approximate) expressions relating the entries of the confusion matrix to return and volatility.
We introduce the new index, which we call the return efficiency index. This index 
is analogous to the area under the curve (AOC) measure in machine learning but relates both the accuracy of the underlying classifier and the return characteristics of the assets. This new metric has a simple 
interpretation: measures the proportion of the possible return range between the worst and the best trading 
strategies. It allows a direct comparison of the performance of different trading strategies
with possibly different underlying assets as well as a comparison to a random flip. it has a simple geometrical interpretation as the cosine similarity between the metrics that reflect the accuracy of the strategy as a classifier and the underlying return profiles of the traded securities. 
We illustrate we consider two daily strategies built by analogy to the nearest-neighbor 
classification: Growth-Value (trading the S{\&}P Growth vs. S{\&}P Value) and Market-Cash
(trading S{\&}P-500 vs. cash) strategies. 

This paper is organized as follows:
In Section~\ref{section_ml_interpretation} we describe trading strategies in terms of machine learning
and list some important statistics from an ML viewpoint.
In Sections~\ref{section_strategy_analysis_ml} and~\ref{section_volatility}, we present a framework to express return and volatility in terms of 
the confusion matrix and its corresponding ratios. In Section~\ref{section_return_efficiency_index} we introduce and discuss the 
return efficiency index to compare trading strategies. We illustrate this by a detailed example in 
Section~\ref{section_detailed_example}.
In Section~\ref{section_example_knn_strategies} we introduce "Winner" and "Loser" Growth-Value and Market-Cash 
strategies based on analogies to $k$-NN.
In Section~\ref{section:results} we present a detailed comparison and show that "loser" strategies outperform 
"winner" strategies with the best choice being the Growth-Value strategies. 
In Section~\ref{section_choosing_k} we address the question of choosing the best value of $k$ and show that this value
is $k=1$. We summarize our key findings and conclude in Section~\ref{section_concluding_remarks}. 

\section{Machine-Learning Interpretation of \\ Trading Strategies}
\label{section_ml_interpretation}


In a typical trading strategy, for any subsequent time period $t_{i}$, we choose between investing in asset $A$ or asset $B$ according to some rule for that strategy. In our discussion, we assume that our periods are days, and our rule tries to predict whether the daily return for $A$ is higher or lower than the corresponding daily return for $B$. 

In the ideal case, we would like to invest in $A$ for the day $d_{i}$ if the daily return for $A$ for that day is higher than the daily return for $B$. Similarly, we would like to invest in $B$ for the day $d_{i}$ if the daily return for $B$ for that day is higher than the daily return for $A$.
Accordingly, we can assign the so-called "true" labels $T_{i}$ to each trading day $d_{i}$ as follows:
\medskip
\begin{enumerate}[nosep]
\item a true label $T_{i}="+"$ for the day $d_{i}$ means we would like to be invested in $A$ for that day
\item a true label $T_{i}="-"$ for the day $d_{i}$ means we would like to be invested in $B$ for
that day.
\end{enumerate}

We illustrate this with the following two examples.

\medskip
\noindent
{\bf Example 1: } Security $A$ is the S{\&}P index and security $B$ is cash. 
We would call such strategies {\it Market-Cash} strategies. They can be implemented by trading the SPY Exchange-Traded 
Fund SPY.  The benchmark for this class of strategies is 
    "Buy-and-Hold" strategy: a passive investing in the S{\&}P-500 index.

In these strategies, we would like to be invested on positive return days and be in cash on negative return days. We try to predict $P_{i}="+"$ (invest) or
$P_{i}="-"$ (cash) decisions for day $d-{i}$ based on historical returns of the S{\&}P-500 index.

Suppose that for 10 consecutive days 
$d_{1},\ldots, d_{10}$, the daily returns $r_{i}$ for S{\&}P-500 index are known. 
We can then assign daily true labels as shown in Table~\ref{tab_true_label_assignment_market_cash} below:
\begin{table}[!ht]
    \centering
    \caption{Assignment of True Labels for Market-Cash (MC) Strategies}
    \vspace{0.1in}
    \begin{tabular}{l  c c c c  c c c c c c}
    \hline
        Day  &  $d_{1}$ & $d_{2}$ & $d_{3}$ & $d_{4}$ & $d_{5}$ & $d_{6}$ & $d_{7}$ & $d_{8}$ & $d_{9}$ & $d_{10}$ \\
        S{\&}P return $r_{i}$ &  1   & 3  & -2   & -4   & 2   & 2  & -1  & 3  & 1 & -1\\
        True Labels: & $+$ & $+$ & $-$ & $-$ & $+$ & $+$ & $-$ & $+$ & $+$ & $-$\\
        \hline
\end{tabular}
\label{tab_true_label_assignment_market_cash}
\end{table}

Positive true labels  ($"+"$) are assigned to six days $d_{1}$, $d_{2}$, $d_{5}$, $d_{6}$, $d_{8}$ and $d_{9}$ with non-negative daily returns $r_{i}\geq 0$. On these five days, we would like to be invested in the S{\&}P-500 index.
By contrast, negative true labels ($"-"$) are assigned to the remaining four days $d_{3}$, $d_{4}$, $d_{7}$, and $d_{10}$
with negative daily returns $r_{i}<0$. We would like to remain in a cash position on these four days.

\medskip
\noindent
{\bf Example 2: } Security $A$ is the S{\&}P Growth index and security $B$ is the S{\&}P value index. We would call such strategies
{\it Growth-Value} strategies. They can be implemented by
trading S{\&}P Growth and S{\&}P Value Exchange-Traded Funds SPYG and SPYV respectively.
In these strategies, we are always fully invested. The benchmark for this class of strategies is 
 "Buy-and-Hold" strategy: a passive investing in the S{\&}P Growth or S{\&}P value indices. 

We try to predict $P_{i}="+"$ (invest in S{\&}P Growth index) or
$P_{i}="-"$ (invest in S{\&}P Value index) decisions for day $d_{i}$ based on comparative returns of Value and Growth indices. For example, we believe that Growth stocks have higher average returns 
than Value stocks.

Suppose that for 10 consecutive days 
$d_{1}, d_{2},\ldots, d_{10}$ the daily returns $r_{i}^{G}$ for Growth and daily returns $r_{i}^{V}$ for Value 
indices are known. We can then assign daily True labels as shown in Table~\ref{tab_true_label_assignment_growth_value} below:
\begin{table}[!ht]
    \centering
    \caption{Assignment of True Labels for Growth-Value (GV) Strategies}
    \vspace{0.1in}
    \begin{tabular}{l c c c c c c  c c c c c}
    \hline
        Day $d_{i}$ &   $d_{1}$ & $d_{2}$ & $d_{3}$ & $d_{4}$ & $d_{5}$ & $d_{6}$ & $d_{7}$ & $d_{8}$ & $d_{9}$ & $d_{10}$ \\
       Growth return $r_{i}^{G}$  & 1 & 2 & -3 & -1  & 3 & 2 & -2 & 5 & 1 & 1\\
       Value  return $r_{i}^{V}$  & 0 & 1 & -2  & -2  & 2 & 1 & -1 & 2 & 3 & -1\\
       True Labels  &  $+$ & $+$ & $-$ & $-$ & $+$ & $+$ & $-$ & $+$ & $+$ & $-$\\
        \hline
\end{tabular}
\label{tab_true_label_assignment_growth_value}
\end{table}

Positive true labels  ($"+"$) are assigned to six days $d_{1}$, $d_{2}$, $d_{5}$, $d_{6}$, $d_{8}$ and $d_{9}$ when the Growth index outperforms the Value index
($r_{i}^{G}\geq r_{i}^{V})$).  These days we would like to be invested in the S{\&}P Growth index.
By contrast, negative true labels ($"-"$) are assigned to the remaining four days $d_{3}$, $d_{4}$, $d_{7}$, and $d_{10}$ when the Growth index underperforms the Value index
($r_{i}^{G}< r_{i}^{V})$).  These days, we would like to invest in the S{\&}P Value index.

\medskip
\noindent
{\bf Schematic Representation: } A trading strategy decides on investing for the day $d_{i}$ by predicting a label $P_{i}$ according to some rule. For every day, we have a predicted label, and our trading strategy is implemented as follows:
\medskip
\begin{enumerate}[nosep]
\item label $P_{i}="+"$ for the day $d_{i}$ means we invest in $A$ for that day
\item label $P_{i}="-"$ for the day $d_{i}$ means we invest in $B$ for that day
\end{enumerate}

\medskip
\noindent
If $P^{+}$ and $P^{-}$ represent days with predicted $"+"$ and $"-"$ labels then schematically, our trading strategy can be represented as follows
\begin{equation*}
    \hbox{Strategy:}  \quad \overbrace{\underbrace{ +,+,\cdots, \cdots ,
    \cdots, +,+ }_{\hbox{invested in } A}}^{\hbox{predicted "+" labels }P^{+}}
    \; , \; 
\overbrace{
\underbrace{-,-,\cdots,\cdots,\cdots,-,-}_{\hbox{invested in } B}}^{\hbox{predicted }"-"
\hbox{ labels } P^{-}} 
\end{equation*}
Trading according to true labels represents an ideal strategy where $P_{i}=T_{i}$: we never make a mistake
in predicting labels. However, in practice, that is not possible. Therefore, our strategy would make mistakes
in predicting "+" and/or "minus" labels. As a result, its performance will be lower than that of an ideal strategy. The higher the accuracy of our strategy in predicting true labels, the better its performance.

A similar situation arises in many problems in machine learning, such as supervised learning. In these problems, we are given true $"+"$ and $"-"$ labels to some dataset and we need to design rules for classifying the unseen data points. In the language of machine learning, we are given true labels for past trading days and are asked to predict future labels. 
However, our predictions of labels are not perfect, and on some days, we predict labels incorrectly. The resulting statistics of classification is summarized in the so-called 4-element "confusion" matrix, 
Therefore, as in machine learning, we can split all days into four disjoint groups:
\medskip
\begin{enumerate}[nosep]
\item  $\text{TP}$ (true positive): on these days, the true label was $+$ and we correctly invested in security $A$
\item  $\text{FP}$ (false positive): on these days, the true label was $+$ but we predicted it incorrectly $"-"$.
As a result, we invested in security $B$ instead of $A$. Such misclassification of true $"+"$ labels is called a Type I error
\item  $\text{TN}$ (true negative): on these days, the true label was $-$ and we correctly invested in security $B$
\item  $\text{FN}$ (false negative): on these days, the true label was $-$ but we predicted it incorrectly as
$"+"$ label. As a result, we invested in security $A$ instead of $B$. 
Such misclassification of true $"-"$ labels is called a Type II error
\end{enumerate}

\medskip
\noindent
Define $T^{+}=(\text{TP}+\text{FN})$ and $T^{-}=(\text{TN}+\text{FP})$.
Then, $T^{+}$ and $T^{-}$ represent the number of days days with true $"+"$ and $"-"$ labels
respectively. From the above entries in the confusion matrix, we can compute the following ratios:

\medskip
\begin{enumerate}[nosep]
\item True Positive Rate  (recall or sensitivity): \quad $\text{TPR}=\text{TP}/T^{+}$
\item True Negative Rate  (specificity): \qquad\qquad\; $\text{TNR} = \text{TN}/T^{-}$
\item Accuracy: \qquad\qquad\qquad $\text{ACC}=(\text{TP}+\text{TN})/(T^{+}+T^{-})$
\item Positive Predicted Value (precision): \quad $\text{PPV}=\text{TP}/(\text{TP}+\text{FP})$
\item Negative Predicted Value:\qquad\qquad\qquad $\text{NPV}=\text{TN}/(\text{TN}+\text{FN})$
\item Prevalence: \qquad \qquad\qquad \qquad\qquad\qquad  $\pi^{+}=T^{+}/T$
\end{enumerate}

\medskip


Intuitively,  the performance of the trading strategy would depend on the number of 
False Positive days (Type I error),  False Negative days (Type II error), and 
the difference in returns of securities $A$ and $B$ on these days. We will quantify
this more precisely in the next section.

\section{Analysis of Strategy Performance by Machine Learning Metrics}
\label{section_strategy_analysis_ml}

To analyze the performance of trading strategies in terms of machine learning metrics, we need to relate 
its return to our ability to correctly predict true labels. 
Our interpretation of strategies and their relative performance is based on the two following observations:

\medskip
\begin{enumerate}[nosep]
\item if we have $k$ days with daily returns $r_{1},\ldots, r_{k}$ then the total return over these $k$ days
\begin{equation*}
    R =(1+r_{1})(1+r_{2})\cdots (1+r_{k}) -1
    \label{total_return}
\end{equation*}
The order of labels corresponding to these days is not important

\item let $\mu(r)=(r_{1}+\cdots +r_{k})/k$ denote the mean of these $k$ daily returns. Then
from the above equation~\eqref{total_return}, we have
\begin{equation*}
    R =  (r_{1}+\cdots +r_{k})  + O(r_{i}^{2})\\
 \approx k\mu(r)
\label{eq_r_mean}
\end{equation*}
\end{enumerate}

From the above observations it follows that if we have $k_{1}$ days with average return $r_{1}$ and
$k_{2}$ days with average daily returns $r_{2}$, the total return $R$ over $k=k_{1}+k_{2}$ days
is (approximately)
\begin{equation}
    R = (1+r_{1})^{k_{1}}(1+r_{2})^{k_{2}}-1     \approx k_{1}r_{1} + k_{2}r_{2}
    \label{eq_r_mean_counts}
\end{equation}

The above equation~\eqref{eq_r_mean_counts}  allows us to analyze and compare the strategies by partitioning trading days into disjoint groups
of correctly and incorrectly
classified true labels and then (approximately) relate the strategy returns to average returns for these groups.

We proceed as follows. Let $r_{A}^{+}$ and $r_{B}^{+}$ denote the average daily returns for $A$ and $B$ on days with true "$+$" labels. Similarly,
let $r_{A}^{-}$ and $r_{B}^{-}$ denote the average daily returns for $A$ and $B$ on days with 
true "$-$" labels. 

Finally, let us define
\begin{equation}
    \Delta^{+} = r_{A}^{+} > r_{B}^{+}\qquad \hbox{and} \qquad \Delta^{-}=r_{B}^{-}-r_{A}^{-}
    \label{define_deltas}
\end{equation}
Our definition of true labels implies that we expect  $\Delta^{+}>0$ and $\Delta^{-}<0$

\medskip
A trading strategy invests in $A$ or $B$ for each day $d_{i}$ based on predicted labels $P_{i}$ for that day.
Since the order of days is not important for the final return, we can schematically describe any trading strategy
as follows:
\begin{equation*}
    \hbox{Strategy:}  \quad \overbrace{\underbrace{ +,+,\cdots, \cdots ,
    \cdots, +,+ }_{\hbox{invested in } A}}^{\hbox{predicted "+" labels }P^{+}}
    \; , \; 
\overbrace{
\underbrace{-,-,\cdots,\cdots,\cdots,-,-}_{\hbox{invested in } B}}^{\hbox{predicted }"-"
\hbox{ labels } P^{-}} 
\end{equation*}
We can measure the performance of strategies relative to the ideal ('max") strategy and the worst ("min") strategy, as well as to buy-and-hold benchmark strategies  $A$ and $B$ (tracking errors).
Each of these strategies can be represented schematically as follows:

\medskip
\begin{itemize}[nosep]
\item a generic strategy "str" is schematically composed of four parts, corresponding to entries in the confusion matrix as follows:
\begin{equation*}
\hbox{"str":} \quad \overbrace{\underbrace{\underbrace{ +,+,\cdots ,+,+}_{TP\; \times\; r_{A}^{+}}}_{\hbox{invested in } A}\; , 
    \;\underbrace{\underbrace{+,+,\cdots, + }_{FN\; \times \; r_{B}^{+}}}_{\hbox{invested in } B}}^{\hbox{true "+" labels }T^{+}}
    \;\; , \;\; 
\overbrace{\underbrace{\underbrace{-,-,\cdots  ,-,-}_{TN\; \times\; r_{B}^{-}}}_{\hbox{invested in } B}\; , \; 
\underbrace{\underbrace{ -,-,\cdots ,-}_{FP\; \times \; r_{A}^{-}}}_{\hbox{invested in } A}}^{\hbox{true }"-"
\hbox{ labels } T^{-}} 
\end{equation*}
\begin{comment}
Therefore,
\begin{equation}
    R^{str} \approx \text{TP}\cdot r_{A}^{+} + \text{FN}\cdot r_{B}^{+} + \text{TN}\cdot r_{B}^{-} +
    \text{FP}\cdot r_{A}^{-}
    \label{strategy_return}
\end{equation}
\end{comment}
\medskip

\item ideal ("max") strategy: predicted labels are all correct $P_{i}=T_{i}$ for each day $i$.This means that we invest in $A$ on all days with true positive labels and invest in
$B$ on all days with true negative labels. 
\begin{comment}
\begin{equation*}
    \hbox{"max" Strategy:}  \quad \overbrace{\underbrace{\underbrace{ +,+,\cdots, \cdots ,
    \cdots, +,+ }_{\hbox{true "+" labels } T^{+}} }_{\hbox{invested in } A}}^{\hbox{predicted "+" labels }P^{+}}
    \; , \; 
\overbrace{
\underbrace{\underbrace{-,-,\cdots,\cdots,\cdots,-,-}_{\hbox{true }"-"\hbox{ labels } T^{-}}}_{\hbox{invested in } B}}^{\hbox{predicted }"-"
\hbox{ labels } P^{-}} 
\end{equation*}

The return $R^{\max}$ of the ideal strategy is schematically
\end{comment}
\begin{equation*}
   \hbox{"max":}  \quad \overbrace{\underbrace{\underbrace{ +,+,\cdots ,+,+}_{TP\; \times\;  r_{A}^{+}}\; , 
    \;\underbrace{+,+,\cdots, + }_{FN;\times\; r_{A}^{+}}}}_{\hbox{invested in } A}^{\hbox{true "+" labels }T^{+}}
    \;\; , \;\; 
\overbrace{\underbrace{\underbrace{-,-,\cdots  ,-,-}_{TN\; \times \; r_{B}^{-}}\; , \; 
\underbrace{ -,-,\cdots ,-}_{FP\; \times\;  r_{B}^{-}}}}_{\hbox{invested in }B}^{\hbox{true }"-"
\hbox{ labels } T^{-}} 
\end{equation*}
\begin{comment}
Therefore, for the ideal "max" strategy we have
\begin{equation}
    R^{\max} \approx T^{+}r_{A}^{+} + T^{-}r_{B}^{+}=
        (\text{TP} + \text{FN}) \, r_{A}^{+}  +
    (\text{TN} + \text{FP})\,  r_{B}^{+} 
    \label{ideal_strategy_return}
\end{equation}
\end{comment}
\item worst ("min") strategy: predicted labels are all incorrect $P_{i}\neq T_{i}$ for each day $i$. This means that we invest in $B$ on all days with true positive labels and invest in $A$ on all days with true negative labels as follows:
\begin{comment}
\begin{equation*}
    \hbox{Worst Strategy:}  \quad \overbrace{\underbrace{\underbrace{ +,+,\cdots, \cdots ,
    \cdots, +,+ }_{\hbox{true "+" labels } T^{+}} }_{\hbox{invested in } B}}^{\hbox{predicted "+" labels }P^{+}}
    \; , \; 
\overbrace{
\underbrace{\underbrace{-,-,\cdots,\cdots,\cdots,-,-}_{\hbox{true }"-"\hbox{ labels } T^{-}}}_{\hbox{invested in } A}}^{\hbox{predicted }"-"
\hbox{ labels } P^{-}} 
\end{equation*}
\end{comment}

% The return $R^{\min}$ of the worst strategy is schematically
\begin{equation*}
   \hbox{"min":}  \quad \overbrace{\underbrace{\underbrace{ +,+,\cdots ,+,+}_{TP\; \times\;  r_{B}^{+}}\; , 
    \;\underbrace{+,+,\cdots, + }_{FN;\times\; r_{B}^{+}}}}_{\hbox{invested in } B}^{\hbox{true "+" labels }T^{+}}
    \;\; , \;\; 
\overbrace{\underbrace{\underbrace{-,-,\cdots  ,-,-}_{TN\; \times \; r_{A}^{-}}\; , \; 
\underbrace{ -,-,\cdots ,-}_{FP\; \times\;  r_{A}^{-}}}}_{\hbox{invested in }A}^{\hbox{true }"-"
\hbox{ labels } T^{-}} 
\end{equation*}
\begin{comment}
Therefore, for the worst "min" strategy we have
\begin{equation}
    R^{\min} \approx T^{+}r_{B}^{+} + T^{-}r_{A}^{-}=
        (\text{TP} + \text{FN}) \, r_{A}^{+}  +
    (\text{TN} + \text{FP})\,  r_{B}^{+} 
    \label{ideal_strategy_return}
\end{equation}
\end{comment}

\item Buy-and-Hold (B{\&}H) "A" strategy: all predicted labels $P_{i}="+"$ for each day $i$. This means that we invest in security $A$ on all days as follows:
\begin{comment}
\begin{equation*}
   \hbox{A Strategy:}  \quad \underbrace{
   \overbrace{\underbrace{ +,+\cdots +,+}_{TP} , 
    \;\underbrace{+,+,\cdots, + }_{FN}}^{\hbox{true "+" labels }T^{+}}
    \;\; , \;\; 
\overbrace{\underbrace{-,-,\cdots  ,-,-}_{TN}\; , \; 
\underbrace{ -,-,\cdots ,-}_{FP}}^{\hbox{true }"-"
\hbox{ labels } T^{-}} }_{\hbox{invested in A}}
\end{equation*}
\end{comment}
% The return $R^{A}$ of such a passive investment strategy is schematically:
\begin{equation*}
   \hbox{B{\&}H A:}  \quad \underbrace{
   \overbrace{\underbrace{ +,+\cdots +,+}_{TP\;\times\; r_{A}^{+}} , 
    \;\underbrace{+,+,\cdots, + }_{FN\; \times \; r_{A}^{+}}}^{\hbox{true "+" labels }T^{+}}
    \;\; , \;\; 
\overbrace{\underbrace{-,-,\cdots  ,-,-}_{TN\; \times \; r_{A}^{-}}\; , \; 
\underbrace{ -,-,\cdots ,-}_{FP\; \times\; r_{A}^{-}}}^{\hbox{true }"-"
\hbox{ labels } T^{-}} }_{\hbox{invested in A}}
\end{equation*}
\begin{comment}
and can be computed approximately as
\begin{equation}
    R^{A} \approx T^{+}r_{A}^{+} + T^{-}r_{A}^{-} = (\text{TP} + \text{FN}) \, r_{A}^{+}  +
    (\text{TN} + \text{FP})\,  r_{A}^{-} 
    \label{buy_hold_strategy_return_a}
\end{equation}
\end{comment}
\item Buy-and-Hold (B{\&}H) "B" strategy: all predicted labels $P_{i}="-"$ for each day $i$. This means that we invest in security $B$ on all days as follows:
\begin{comment}
\begin{equation*}
   \hbox{B:}  \quad \underbrace{
   \overbrace{\underbrace{ +,+\cdots +,+}_{TP} , 
    \;\underbrace{+,+,\cdots, + }_{FN}}^{\hbox{true "+" labels }T^{+}}
    \;\; , \;\; 
\overbrace{\underbrace{-,-,\cdots  ,-,-}_{TN}\; , \; 
\underbrace{ -,-,\cdots ,-}_{FP}}^{\hbox{true }"-"
\hbox{ labels } T^{-}} }_{\hbox{invested in B}}
\end{equation*}
\end{comment}
% The return $R^{B}$ of such a passive investment strategy is schematically:
\begin{equation*}
   \hbox{B{\&}H B:}  \quad \underbrace{
   \overbrace{\underbrace{ +,+\cdots +,+}_{TP\;\times\; r_{B}^{+}} , 
    \;\underbrace{+,+,\cdots, + }_{FN\; \times \; r_{B}^{+}}}^{\hbox{true "+" labels }T^{+}}
    \;\; , \;\; 
\overbrace{\underbrace{-,-,\cdots  ,-,-}_{TN\; \times \; r_{B}^{-}}\; , \; 
\underbrace{ -,-,\cdots ,-}_{FP\; \times\; r_{B}^{-}}}^{\hbox{true }"-"
\hbox{ labels } T^{-}} }_{\hbox{invested in B}}
\end{equation*}
\begin{comment}
The return of this strategy is
\begin{equation}
    R^{B} \approx T^{+}r_{B}^{+} + T^{-}r_{B}^{-}=(\text{TP} + \text{FN}) \, r_{B}^{+}  +
    (\text{TN} + \text{FP})\,  r_{B}^{-} 
    \label{buy_hold_strategy_return_B}
\end{equation}
\end{comment}
\end{itemize}
We can now summarize the returns for these strategies as follows:
\begin{equation}
\begin{split}
\begin{cases}
    R_{str} & \approx \text{TP}\cdot r_{A}^{+} + \text{FN}\cdot r_{B}^{+} + \text{TN}\cdot r_{B}^{-} +
    \text{FP}\cdot r_{A}^{-} \\
      R_{\max} & \approx  T^{+}\, r_{A}^{+} \; + \; T^{-}\, r_{B}^{-} \\
      %  (\text{TP} + \text{FN}) \, r_{A}^{+}  +     (\text{TN} + \text{FP})\,  r_{B}^{-}   \\
    R_{\min} & \approx  T^{+}\,r_{B}^{+} \; + \; T^{-}\,r_{A}^{-} \\
     %   (\text{TP} + \text{FN}) \, r_{B}^{+}  +     (\text{TN} + \text{FP})\,  r_{A}^{-} \\
        R_{A} &  \approx  T^{+}\,r_{A}^{+}\; +\;  T^{-}\,r_{A}^{-} \\
       % = (\text{TP} + \text{FN}) \, r_{A}^{+}  +     (\text{TN} + \text{FP})\,  r_{A}^{-} \\
        R_{B} &  \approx  T^{+}\,r_{B}^{+} \; + \; T^{-}\,r_{B}^{-}
  %      (\text{TP} + \text{FN}) \, r_{B}^{+}  +     (\text{TN} + \text{FP})\,  r_{B}^{-} 
    \end{cases}
\end{split}
\label{return_all_strategies}
\end{equation}

The underperformance of the strategy relative to the ideal strategy is
\begin{equation}
R_{\max}-R_{str}\approx\underbrace{\text{FN} \cdot \Delta^{+}}_{\text{type II loss}}
\; + \; \underbrace{\text{FP} \cdot \Delta^{-}}_{\text{type I loss}}
\end{equation}
The first contribution to underperformance is due to our error in predicting days with true 
"$+$" labels. The second contribution is due to an error in predicting days with "$-$" true labels.

The tracking errors  $\text{TE}^{A}$ and $\text{TE}^{B}$ relative to $A$ and
$B$ are given by
\begin{equation}
\begin{split}
    \text{TE}_{A} & = R_{str}-R_{A} \approx  -\text{FN} \cdot \Delta^{+}
    + \,\text{TN}\cdot \Delta^{-}\\
  \text{TE}_{B} &= R_{str}-R_{B} \approx \;\;\, \text{TP}\cdot \Delta^{+} -\; \text{FP}\cdot \Delta^{-}
\end{split}
\end{equation}
To outperform benchmark $A$ we need 
$\text{FN}< \text{TN}\left(\Delta^{-}/\Delta^{+}\right)$. To outperform benchmark $B$
we need the $\text{FP}< \text{TP}\left(\Delta^{-}/\Delta^{+}\right)$. To outperform both benchmarks, we
need
\begin{equation}
    \text{FN}< \text{TN}\left(\Delta^{-}/\Delta^{+}\right) \qquad \hbox{and} \qquad
    \text{FP}< \text{TP}\left(\Delta^{-}/\Delta^{+}\right)
    \label{min_requirements}
\end{equation}

The underperformance of Buy-and-Hold in $A$ and $B$ relative to the ideal strategy is
\begin{equation}
    R_{\max}-R_{A}  \approx T^{-} \Delta^{-} \qquad\hbox{and}\qquad
     R_{\max}-R_{B}  \approx T^{+}\Delta^{+} 
\label{r_bh_vs_ideal_mc}
\end{equation}

For $T^{-}=(\text{TN}+\text{FP})$ days, we have an average negative return $\Delta^{-}=(r_{B}^{-}-r_{A}^{-})$. Unlike the ideal strategy, in the Buy-and-Hold strategy BH-A, we were fully invested in $A$ during these days, resulting in a loss given by equation~\eqref{r_bh_vs_ideal_mc}. Although we do not make trading decisions in a Buy-and-Hold strategy,
intuitively, we can think of this loss as consisting of two components: we lost a portion of the potential return 
in $\text{TN}$ negative true labels days. These $\text{TN}$ were correctly identified by the strategy. The second component of the loss in $\text{FP}$ days.  These
$\text{FP}$ days were incorrectly identified by our strategy as well.


\subsection{Machine-learning interpretations for Market-Cash}

For the Market-Cash strategies, security $A$ is S{\&}P-500 index, and security $B$ is cash.
For S{\&}P-500, we have $r_{A}^{+} \geq 0$ and $r_{A}^{-}<0$. For cash, we have $r_{B}^{-}=r_{B}^{+}=0$,
and therefore, $\Delta^{+}=r_{A}^{+}$ and $\Delta^{-}=-r_{A}^{-}$
Therefore, the expressions relating returns to machine learning metrics reduce to a much simpler form.
For the Market-Cash strategy we have:
\begin{equation*}
    \hbox{MC:}  \quad \overbrace{\underbrace{
    \underbrace{ +,+\cdots +,+}_{TP}}_{\hbox{S{\&}P}}\; , 
    \;\underbrace{\underbrace{+,\cdots + }_{FN}}_{\hbox{cash}}}^{\hbox{true "+" labels }T^{+}}
    \;\; , \;\; 
    \overbrace{
\underbrace{\underbrace{-,-,\cdots  ,-,-}_{TN}}_{\hbox{cash}}\; , \; 
\underbrace{\underbrace{ -,\cdots ,-}_{FP}}_{\hbox{S{\&}P}}}
^{\hbox{true } "-"
\hbox{ labels } T^{-}} 
\end{equation*}
The expressions for returns in equation~\eqref{return_all_strategies} are reduced to a much simpler form
\begin{equation}
\begin{split}
\begin{cases}
    R_{str} & \approx \text{TP}\cdot r_{A}^{+}\,  + \,
    \text{FP}\cdot r_{A}^{-} \\
      R_{\max} & \approx  T^{+} \, r_{A}^{+} \\
    R_{\min} & \approx  T^{-}\, r_{A}^{-} \\
        R_{A} &  \approx  T^{+}\, r_{A}^{+} \, + \, T^{-}\, r_{A}^{-}  
     \end{cases}
\end{split}
\label{return_all_mc_strategies}
\end{equation}

The difference in returns with the ideal strategy is then
\begin{equation}
 R_{\max}- R_{MC}  \approx  \text{FN}\cdot r^{+} -\; \text{FP}\cdot r^{-} 
    \label{mc_underperformance}
\end{equation}

This has a simple and intuitive interpretation: our under-performance 
relative to the ideal strategy is composed of two parts: 
\medskip
\begin{enumerate}[nosep]
    \item for FP days, we misclassified the true negative days as $"+"$ and invested in the losing days, and therefore, losing
       $(-\text{FP}\cdot   r^{-})$
    \item for FN days, we misclassified the true positive days as $"-"$, stayed in cash, and did not invest in the positive days. This resulted in the "opportunity" loss of $(\text{FN}\cdot  r^{+})$
\end{enumerate}

\begin{comment}
Since $\text{FN}=(1-\text{TPR})\, T^{+}$ and $\text{FP}=(1-\text{TNR})\, T^{-}$,
we can re-write the expression in equation~\eqref{mc_underperformance} as follows
\begin{equation}
 R^{\max}- R^{mc}  \approx  (1-\text{TPR})\, T^{+}\cdot \vert r^{+} \vert
 + (1-\text{TNR})\, T^{-}\cdot \vert r^{-} \vert
    \label{mc_underperformance_ratios}
\end{equation}
The above equation~\eqref{mc_underperformance_ratios} states that the under-performance of our strategy
is determined by our failure rate to detect true positive and true negative days and by the absolute values of the returns for the days when we make these mistakes.
\end{comment}

Finally, from equations~\eqref{return_all_mc_strategies} we can estimate the tracking error as
\begin{equation}
\text{TE}_{A}=R_{MC}-R_{A} \approx\underbrace{-\text{FN} \cdot r_{A}^{+}}_{\hbox{type II loss}}
\; + \;  \underbrace{\text{TN}\cdot r_{A}^{-} }_{\hbox{type I loss}}
\label{r_mc_tracking_error}
\end{equation}
Therefore, the  tracking error consists of two components:
\medskip
\begin{enumerate}[nosep]
\item For $\text{FN}$ days, the market was positive. The market-cash strategy was in cash, while the Buy-and-Hold 
was invested. This resulted in a loss of approximately $\text{FN} \cdot r^{+}$ to the tracking error.
\item For $\text{TN}$ days, the market was negative.  The market-cash strategy was in cash, while the Buy-and-Hold was invested. This resulted in a gain of approximately
$\text{FN} \cdot \vert r^{-}\vert $
   
\end{enumerate}


\section{Analysis of Volatility and Sharpe Ratios by Machine Learning Metrics}\label{section_volatility}

We now consider volatility. We will use $S$ to denote volatility with appropriate subscripts to
indicate specific trading strategies or benchmarks. If $\sigma$ denotes the standard deviation of daily returns then the volatility over $t$ days is given by  $S=\sigma\sqrt{t}$. Let $\pi^{+}=T^{+}/T$ denote the prevalence.

Let $\sigma_{A}$ and $\sigma_{B}$ denote the standard deviation of daily returns for $A$ and $B$, then the volaitities for the buy-hold strategies $A$ and $B$ we have
\begin{equation}
    S_{A}=\sqrt{T}\,\sigma_{A} \qquad \hbox{and}\qquad S_{B} = \sqrt{T}\,\sigma_{B}
\end{equation}
For the ideal strategy, we have
\begin{equation}
S_{\max}=\sqrt{T^{+}\sigma_{A}^{2} + T^{-}\sigma_{B}^{2}}
= \sqrt{ \pi^{+} S_{A}^{2} +(1-\pi^{+})S_{B}^{2} }
\end{equation}
Similarly, for the worst strategy we have
\begin{equation}
S_{\min}=\sqrt{T^{+}\sigma_{B}^{2} + T^{-}\sigma_{A}^{2}}
= \sqrt{ \pi^{+}S_{B}^{2} +(1-\pi^{+})S_{A}^{2} }
\end{equation}
 For a generic strategy, we have
\begin{equation}
    S_{str} = \sqrt{ (\text{TP} + \text{FP})\sigma_{A}^{2} + (\text{TN}+\text{FN})\sigma_{B}^{2}}
%     & = \sqrt{ \frac{(\text{TP} + \text{FP})}{T} S_{A}^{2} + 
%    \frac{(\text{TN}+\text{FN})}{T} S_{B}^{2}}\\
     = \sqrt{
\left(\frac{P^{+}}{T}\right)  S_{A}^{2} +
 \left( \frac{P^{-}}{T} \right) S_{B}^{2} }
    \label{volatility_generic}
\end{equation}
Since 
\begin{equation*}
    P^{+}=\left(\frac{\text{TPR}}{\text{PPV}}\right)\pi^{+}\, T \quad \hbox{and} \quad
    P^{-}=\left(\frac{\text{TNR}}{\text{NPV}}\right)(1-\pi^{+})\, T 
\end{equation*}
we can rewrite the expression for volatility for a generic strategy as 
\begin{equation}
    S_{str} = \sqrt{
\left(\frac{\text{TPR}}{\text{PPV}}\right)\pi^{+}\,  S_{A}^{2} +
 \left(\frac{\text{TNR}}{\text{NPV}}\right)(1-\pi^{+}) S_{B}^{2} }
    \label{volatility_generic_2}
\end{equation}


\begin{comment}
Since
\[
\text{TP} + \text{FP} = T \cdot (\pi^{+} + (1 - \text{TNR}) \cdot (1 - \pi^{+}))
\]
and
\[
\text{TN} + \text{FN} = T \cdot ((1 - \pi^{+}) + (1 - \text{TPR}) \cdot \pi^{+})
\]
we can rewrite the above expression for volatility as
\begin{equation*}
S_{str} = \sqrt{\left( \pi^{+} + (1 - \text{TNR})  (1 - \pi^{+}) \right) S_{A}^{2} + \left( (1 - \pi^{+}) + (1 - \text{TPR})  \pi^{+} \right) S_{B}^{2}}
\end{equation*}

\end{comment}

Ignoring the risk-free rate we can write the expressions relating the Sharpe Ratios with confusion matrix entries.
We will use the notation $\text{SR}$ to denote the Sharpe ratio.

For the Buy-and-Hold strategist $A$ and $B$ we have
\begin{equation}
        \text{SR}_{A}=\frac{R_{A}}{S_{A}} =
        \frac{T^{+}\,r_{A}^{+} + T^{-}\, r_{A}^{-}}{S_{A}}
        \quad \hbox{and} \quad 
         \text{SR}_{B}=\frac{R_{B}}{S_{B}} =
        \frac{T^{+}\,r_{B}^{+} + T^{-}\, r_{B}^{-}}{S_{B}}
\end{equation}

The Sharpe's ratio for the ideal strategy
\begin{equation}
    \text{SR}_{\max}=\frac{R_{\max}}{S_{\max}} = \frac{T^{+}r_{A}^{+}+T^{-}r_{B}^{-}}
    {\sqrt{ \pi^{+} S_{A}^{2} +(1-\pi^{+})S_{B}^{2}}}
\end{equation}

The Sharpe's ratio for the worst strategy
\begin{equation}
    \text{SR}_{\min}=\frac{R_{\min}}{S_{\min}} = \frac{T^{+}r_{B}^{+}+T^{-}r_{A}^{-}}
    {\sqrt{ \pi^{+} S_{B}^{2} +(1-\pi^{+})S_{A}^{2}}}
\end{equation}

Note that although $R_{max}>R_{\min}$, it is possible that
$SR_{\max} < SR_{\min}$. For example, if the numbers of true positive and true negative labels are the same then $T^{+}=T^{-}$ and $\pi^{+}=0.5$. In this case, from the above two equations we obtain
\begin{equation}
    SR_{max} =\left(\frac{r_{A}^{+}+r_{B}^{-}}{r_{A}^{-} + r_{B}^{+}}\right)\cdot SR_{\min}
\end{equation}
Although we assume $r_{A}^{-}<r_{B}^{-}<r_{B}^{+}<r_{A}^{+}$, it is still 
quite possible to have $(r_{A}^{+}+r_{B}^{-}) < (r_{A}^{-} + r_{B}^{+})$ resulting in  $SR_{\max} < SR_{\min}$. 

Finally, the Sharpe's ratio for a generic strategy is
\begin{equation}
\text{SR}_{str}=\frac{R_{str}}{S_{str}} =
\frac{\text{TP}\cdot r_{A}^{+} + \text{FN}\cdot r_{B}^{+} + \text{TN}\cdot r_{B}^{-} +
    \text{FP}\cdot r_{A}^{-}}
{\sqrt{
\left( P^{+}/{T} \right)  S_{A}^{2} +
\left(P^{-}/{T}\right) S_{B}^{2} }}
\label{sgarpe_generic}
\end{equation}

If we can improve the strategy by identifying only one extra true positive day,
this means that we increase $\text{TP}$ by 1 and decrease $\text{FN}$ by 1. This results in an increase of $\Delta^{+}=(r_{A}^{+}-r_{B}^{+})$ to the return in the numerator. This will also increase $P^{+}$ by 1 and decrease $P^{-}$ by 1. As a result, the expression in the denominator under the square root will increase by 
$(S_{A}^{2}-S_{B}^{2})/T=(\sigma_{A}^{2}-\sigma_{B}^{2})$.

On the other hand, if we can improve the strategy by identifying only one extra true negative day,
this means that we increase $\text{TN}$ by 1 and decrease $\text{FP}$ by 1. This results in an increase of $\Delta^{-}=(r_{B}^{-}-r_{A}^{-})$ to the return in the numerator. This will also decrease $P^{+}$ by 1 and 
increase $P^{-}$ by 1. As a result, the expression in the denominator under the square root will decrease by $(S_{A}^{2}-S_{B}^{2})/T$.

Finally, if we can improve the strategy by identifying only one extra true positive and one extra negative day, then $P^{+}$ and $P^{-}$
will remain unchanged. Therefore, the volatility will remain unchanged, whereas the return will increase  by $(\Delta^{+}+\Delta^{-})$
increasing the Sharpe's ratio.

This means that we always increase both the returns and Sharpe's ratio by increasing our accuracy in identifying only negative labels. 
On the other hand, if we increase our accuracy in identifying only the positive labels, we increase our returns but not necessarily the Sharpe's ratio.
Increasing the accuracy in both positive and negative will increase both the return and Sharpe's ratio.

For the Market-Cash strategies, $A$ is the $S{\&}P$-500 index and $B$ is cash. In this case,  we have $r_{B}^{-}=0$, $r_{B}^{+}=0$, and $S_{B}=0$. Therefore, the above expressions
for volatility and Sharpe's ratio reduce to a much simpler form
\begin{equation}
S_{MC} = S_{A} \sqrt{ P^{+}/T}  \qquad \hbox{and} \qquad 
\text{SR}_{MC}=
\frac{\text{TP}\cdot r_{A}^{+} +     \text{FP}\cdot r_{A}^{-}}
{S_{A}\,\sqrt{P^{+}/T}}
\end{equation}


\section{Comparing Strategies by Machine Learning Metrics":  the "Return Efficiency Index"}
\label{section_return_efficiency_index}

In the previous section, we derived (approximate) expressions for strategy performance in terms of machine learning metrics. 
But how should we compare any two strategies? Many metrics are used in finance, such as tracking error, Sharpe ratio, drawdowns, and others. One drawback of such metrics is that they do not take into account how close the trading strategy is to the ideal case. Maybe it was not possible to significantly outperform the benchmark. 

Is there a way to quantify this and is there a way to compare strategies not by comparing their relative absolute performance but by assigning them a universal score from 0 to 1, reflecting their ability to capture the maximum possible return?

We suggest proceeding as follows: consider the worst and the best trading strategy with corresponding returns
$R_{\min}$ and $R_{\max}$ respectively. For any strategy, we have $R_{\min}\leq R_{str} \leq R_{\max}$. Unless all daily returns are the same, we
have $R_{\min} < R_{\max}$. We suggest to define the strategy {\it return efficiency index} $I_{str}$ as
\begin{equation}
\hbox{(return) efficiency index } I_{str} =\frac{R_{str}-R_{\min}}{R_{\max}-R_{\min}}
\end{equation}
For brevity, we will refer to it as the {\it efficiency index}. The above formula is analogous to 
"min-max" scaling of data widely used in machine learning.

The above definition implies that for any strategy
$0\leq I_{str}\leq 1$. The numerator $(R_{str}-R_{\min})$ is the excess return compared to the worst return $R_{\min}$, whereas the denominator is the maximum possible excess return generated by predicting all true labels correctly. 

Therefore, the return capture efficiency has the following simple interpretation: it tells us what fraction of the possible return range (from best to worst possible strategy) is captured by our strategy. For the worst strategy, this index is 0, and for the best possible strategy, it is 1. 

For any strategy, its efficiency index would depend on how good the strategy is in predicting positive and negative true labels. To express the efficiency index in terms of machine learning metrics, we proceed as follows. 
As before, recall $\Delta^{+}=r_{A}^{+}-r_{B}^{+}$ and
$\Delta^{-}=r_{B}^{-}-r_{A}^{-}$. Note that $\Delta^{+}>0$ and $\Delta^{-}>0$.

From our  equations~\eqref{return_all_strategies} we obtain the following for the efficiency  index:
\begin{equation}
I_{str} = \frac{\text{TP}\cdot \Delta^{+} +\text{TN}\cdot \Delta^{-}}
{T^{+} \Delta^{+} + T^{-}\Delta^{-}}
\label{accuracy_index_delta}
\end{equation}

We can rewrite this in terms of recall $\text{TPR}$ and
$\text{TNR}$ as follows:
\begin{equation}
I_{str} = \text{TPR} \cdot \left[\frac{T^{+}\Delta^{+} }
{T^{+} \Delta^{+} + T^{-}\Delta^{-}}\right] 
+ \text{TNR} \cdot \left[ 
\frac{T^{-} \Delta^{-}}
{T^{+} \Delta^{+} + T^{-}\Delta^{-}} \right]
\label{accuracy_index_as_sum}
\end{equation}

To interpret the index in equation~\eqref{accuracy_index_as_sum} we note that
for the ideal strategy $I_{\max}=1$ and its efficiency index can be written as 
\begin{equation}
I_{\max} = \left[ \frac{T^{+}\Delta^{+} }
{T^{+} \Delta^{+} + T^{-}\Delta^{-}} \right]
+ \left[ \frac{T^{-} \Delta^{-}}
{T^{+} \Delta^{+} + T^{-}\Delta^{-}}\right]
\label{ideal_index_as_sum}
\end{equation}
The first term in $I_{\max}$ in the above equation~\eqref{ideal_index_as_sum} is the fraction of excess return over $R^{\min}$ captured by the positive True labels, and the second term is the fraction of excess return captured by the negative True labels.

Therefore, the (return) efficiency index for any strategy $I_{str}$ has the following interpretation: it is the weighted sum of fractions excess returns captured by the positive and negative true labels with precision 
$\text{TPR}$ and recall $\text{TNR}$ as the corresponding weights.  

Next, consider a buy-and-hold strategy of investing only in $A$ or $B$. These can be thought of as trading strategies where all predictive labels are $"+"$.

From equations~\eqref{return_all_strategies}
and~\eqref{accuracy_index_as_sum} the 
return efficiency indices for these buy-and-hold strategies are
\begin{equation}
I_{{A}} = \frac{T^{+}\Delta^{+}}{T^{+} \Delta^{+} + T^{-}\Delta^{-}} \qquad \hbox{and} \qquad
I_{{B}} = \frac{T^{-}\Delta^{-}}{T^{+} \Delta^{+} + T^{-}\Delta^{-}}
\label{accuracy_index_benchmark}
\end{equation}
Therefore, we can rewrite the efficiency index for any strategy in terms of the efficiency indices of buy and hold strategies for $A$ and $B$ as follows:
\begin{equation}
I_{str} = \text{TPR}\cdot I_{A} + \text{TNR}\cdot I_{B}
\label{accuracy_as_benchmarks}
\end{equation}

This provides an alternative interpretation of the efficiency index: it is the weighted sum of the efficiency indices of buy-and-hold strategies taken with weights $\text{TPR}$ and $\text{TNR}$ respectively.

We can re-write the above equation ~\eqref{accuracy_as_benchmarks} in terms of the (unscaled) dot product as 
follows:
\begin{equation}
I_{str} = \underbrace{(\; \text{TPR} \, , \, \text{TNR}\; )}_{U_{1}} \,  \cdot \, 
\underbrace{(\; I_{A},\;  I_{B}\; )}_{U_{2}}
\label{dot_product}
\end{equation}
The above expression for $I_{str}$ is the dot-product of two vectors:
\medskip
\begin{enumerate}[nosep] 
\item the vector $U_{1}=(\text{TPR}\, , \, \text{TNR})$ is machine learning related. It describes the accuracy of our strategy as a classifier predicting positive and negative labels.
\item the vector 
 $U_{2}=(I_{A},\,  I_{B})$ is returns-related. It 
 describes the returns profiles defined by the benchmarks $A$ and $B$.
 \end{enumerate}
 
Geometrically, this unscaled dot product $U_{1}\cdot U_{2}$ is related to the cosine of the angle between
$U_{1}$ and $U_{2}$ via 
$U_{1}\cdot U_{2}=\lvert\rvert U_{1}\lvert\rvert \, \lvert\rvert U_{2}\lvert\rvert\, \cos\alpha$. This is illustrated in Figure~\ref{fig:winer_loser_fridays}.

\medskip
We can compare our strategy to benchmarks by comparing the corresponding efficiency indices instead of examining the tracking errors.
From equations~\eqref{accuracy_index_benchmark} 
and~\eqref{accuracy_index_delta} we have
\begin{equation}
I_{str}-I_{A} = \frac{\text{TN}\cdot\Delta^{-}-\text{FN}\cdot\Delta^{+}}{T^{+} \Delta^{+} + T^{-}\Delta^{-}} \quad \hbox{and} \quad 
I_{str}-I_{B}  = \frac{\text{TP}\cdot\Delta^{+}-\text{FP}\cdot\Delta^{-}}{T^{+} \Delta^{+} + T^{-}\Delta^{-}} 
\label{tracking_ratios}
\end{equation}
Therefore, our ability to outperform the benchmark $A$  depends on our ability to correctly identify 
labels and on the relative values of $\Delta^{+}$ and $\Delta^{-}$. From equation~\eqref{tracking_ratios} we have
\begin{equation}
\begin{split}
I_{str}-I_{A} & > 0 \; \longleftrightarrow \; \text{FN}<\text{TN}\, (\Delta^{+}/\Delta^{-})\\
I_{str}-I_{B} & > 0 \; \longleftrightarrow \; \text{FP}<\text{TP}\, (\Delta^{+}/\Delta^{-})
\end{split}
\end{equation}

Finally, consider a "random flip" strategy $p\text{-Random}$ where we flip a coin to decide True labels. 
Let $p$ be the probability that we invest in security $A$ and $(1-p)$ be the probability that we invest in security $B$.
of "head" (we choose security $A$) and $(1-p)$ be the probability of tails. This means that out of $T^{+}$ true labels we correctly identify $pT^{+}$ positive true labels and out of $T^{-}$ true $"-"$ labels we 
correctly identify $(1-p)T^{-}$ negative true labels. This is equivalent to $\text{TPR}=p$ and 
$\text{TNR}=(1-p)$.
Therefore, from equation~\eqref{accuracy_index_as_sum} we obtain
\begin{equation}
I_{\text{random}} = p \cdot I_{A} + (1-p)\cdot I_{B}
% \left[\frac{T^{+}\Delta^{+} }
% {T^{+} \Delta^{+} + T^{-}\Delta^{-}}\right] 
% + (1-p) \cdot \left[ 
% \frac{T^{-} \Delta^{-}}
%{T^{+} \Delta^{+} + T^{-}\Delta^{-}} \right]
\label{random_index_as_sum}
\end{equation}
Therefore, the efficiency index of the random strategy is a weighted average of $I^{A}$ and $I^{B}$ with weights
$p$ and $(1-p)$ respectively. 
% Note that from equation~\eqref{accuracy_index_benchmark} we have $I^{A} +I^{B}=1$.
% Therefore, since $I^{A}>0$ and $I^{B}>0$, it follows from the above equation~\eqref{random_index_as_sum} that
% $I^{tandom} < I^{\max}$ for all $p$.

From equations~\eqref{accuracy_index_as_sum} and~\eqref{random_index_as_sum}, it follows that 
to outperform a random index, we must have
\begin{equation*}
\begin{split}
 I_{str} >  I_{\text{random}} & \Longleftrightarrow
(\text{TPR}-p)I_{A} +
\left(\text{TNR}- (1-p))\right)I_{B} >  0\\
% &  \Longleftrightarrow 
%  \frac{\text{TPR}\cdot I^{A}+(1-\text{TNR})\cdot I^{B}}
% {I^{A}+I^{B}} > p \\
& \Longleftrightarrow
 \text{TPR}\cdot I_{A}-(1-\text{TNR})\cdot I_{B} > p(I_{A}-I_{B})
\end{split}
\end{equation*}

If $I_{A}>I_{B}$ then our strategy outperforms a $p$-random flip strategy if
\begin{equation}
    \frac{\text{TPR}\cdot I_{A}-(1-\text{TNR})\cdot I_{B}}{I_{A}-I_{B}}\, > \, p
    \label{equivalent_p}
\end{equation}

% For example, a sufficient (but not a necessary) condition for a strategy to outperform a random flip strategy is
% $\text{TPR} > p$ and $\text{TNR}< 1-p$.
%In this case,
% $\text{TPR}=\text{TNR}=1/2$ and from 
% equations~\eqref{accuracy_index_as_sum} and~\eqref{ideal_index_as_sum} we obtain that for such
% random flip strategy $I^{random}=1/2$. 

Therefore, we can compare our strategy to the efficiency index of some benchmarks and to a random strategy. This is similar to analyzing classifiers in machine learning, where we compute the so-called Area Under Curve and compare it to $1/2$ to see how different our predictions on labels are from those generated by a random coin flip. 

The equation~\eqref{accuracy_index_as_sum} gives us a universal way to compare any 
strategies in terms of their ability to capture the potential excess return over the worst strategy. 


\medskip
\noindent
{\bf Example 3: } Consider the Market-Cash (MC) strategies.
for these strategies, we have $r_{B}^{+}=r_{B}^{-1}=0$ and
therefore, $\Delta^{+}=r^{+}$ and $\Delta^{-}=-r^{-}$. Note that $r^{-}<0$ and therefore $\Delta^{-}>0$. They can be interpreted as the absolute average daily for positive and negative return days respectively.

The efficiency index of the strategy and benchmark S{\&}P-500 index is given by
\begin{equation}
I_{str} = \frac{\text{TP}\cdot r^{+} +\text{TN}\cdot \vert r^{-}\vert }
{T^{+} r^{+} + T^{-}\vert r^{-}\vert } \quad \hbox{and}
\quad 
I_{MC} = \frac{\text{TN}\cdot \vert r^{-}\vert  -\text{FN}\cdot  r^{+} }
{T^{+} r^{+} + T^{-}\vert r^{-}\vert }
\end{equation}
The market-cash strategy can outperform the index if we do not have too many False Negative (FN) predictions, namely we must have $\text{FN}<\text{TN}\, (r^{+}/\vert r^{-}\vert)$.
\begin{figure}[htbp]
  \centering
  \includegraphics[width=\linewidth]{plots/AUC.png}
  \caption{AUC}
  \label{fig:winer_loser_fridays}
\end{figure}


\section{A Detailed Example}
\label{section_detailed_example}

Let us present a detailed example of comparing two strategies, $X$ and $Y$. Strategy $X$ is a Growth-Value strategy and strategy $Y$ is a Market-Cash strategy. We assume that we have the following daily data 
for eleven days
$d_{0},\ldots, d_{10}$ starting with day $d_{0}$. The detailed results for returns $r_{i}$, balances $B_{i}$,
true labels $T_{i}$ and predicted labels $P_{i}$ for each day $d_{i}$ are summarized in Table~\ref{tab_example_11_days}.


\begin{table}[!ht]
    \centering
    \caption{Daily Data for $X$ and $Y$ Strategies}
    \vspace{0.1in}
    \begin{tabular}{l | rr | rr | rr | rrcc | rrcc} \hline
 \multirow{2}{*}{$d_{i}$}    & \multicolumn{2}{c|}{S{\&}P} & \multicolumn{2}{c|}{G} & \multicolumn{2}{c|}{V} & 
    \multicolumn{4}{c|}{Strategy $X$} & \multicolumn{4}{c}{Strategy $Y$} \\ 
    \cline{2-15}
 & $r_{i}$ & $B_{i}$ & $r_{i}$ & $B_{i}$ & $r_{i}$ & $B_{i}$ & $r_{i}$ & $B_{i}$  & $T_{i}$ & $P_{i}$ & $r_{i}$ & $B_{i}$ & $T_{i}$ & $P_{i}$ \\ \hline
%        $d_{0}$ & 3 & 100 & 3 & 100 & 1 & 100 & 0 & 100 & + & $*$ & $*$ & 100 & + & $*$ \\ 
        $d_{1}$ & 1   & 101 & 1  & 101  &  0 & 100  &  1 & 101   & +     & +     &  1   & 101 &  +  & + \\ 
        $d_{2}$ & 3   & 104 & 2  & 103  &  1 & 101  &  2 & 103   & +     & +     & 3    & 104 & $+$ & + \\ 
        $d_{3}$ & -2  & 102 & -3 & 100  & -2 & 99   &  -3 & 100  & $-$   & $+$  & -2   & 102 & $-$ & $+$ \\ 
        $d_{4}$ & -4  & 98  & -1 & 99   & -2 & 97   &  -2 & 98   & $-$   & $-$   & 0    & 102 & $-$ & $-$ \\ 
        $d_{5}$ & 2   & 100 & 3  & 102  &  2 & 99   &  3 & 101    & +     & +    & 0    & 102 & +   & $-$ \\ 
        $d_{6}$ & 2   & 102 & 2  & 104  &  1 & 100   &  2 & 103   & +     & +     & 3    & 105 & +   & + \\ 
        $d_{7}$ & -1  & 101 & -2 & 102  & -1 & 99   & -1 & 102   & $-$   & $-$   & -1   & 104 & $-$ & + \\ 
        $d_{8}$ & 3  & 104  & 5  & 107  &  2 & 101   &  2 & 104   & $+$   & $-$     & 0    & 104 & $+$ & $-$ \\ 
        $d_{9}$ & 1   & 105  & 2  & 109  &  3 & 104  &  3 & 107   & $+$   & $-$   & 2    & 106 & + & $+$ \\ 
        $d_{10}$ & -1 & 104 & -1  & 108  & -1 & 103  &  -1 & 106  & $-$     & +   & -1    & 105 & $-$ & + \\ 
    \hline
\end{tabular}
\label{tab_example_11_days}
\end{table}

The growth of \$100 investments in S{\&}P Buy-and-Hold strategies and our two strategies in shown below
in Figure~\ref{fig:example_10_days}.

We first compute the relative performance of the two strategies over 10 days starting with day $d_{2}$.


\begin{figure}[h!]
  \centering
%  \includegraphics[width=\linewidth]{plots/example_10_days.jpg}   
  \includegraphics[width=\linewidth]{plots/growth_comparison_3.jpg} 
  \caption{Example of Strategy Comparison}
  \label{fig:example_10_days}
\end{figure}

Our results are summarized in the Table~\ref{tab_comparison_11_days}:

All strategies start with the same balance $B_{0}=\$100$.  The higher final balance (after rounding to the nearest integer) of {\$}106 is achieved by the Growth-Value $X$, while the lower final balance of {\$}105 was achieved by the Market-Cash $Y$.
The corresponding total returns for these strategies are $R_{X}=6\%$ and $R_{Y}=5\%$, respectively. 
The total return of Strategy-$X$ is more than twice the total return of Strategy-$Y$.

Next, we compare volatility. If 
$\sigma$  denotes the standard deviations of daily returns 
$\{ r_{1}, \ldots, r_{10}\} $ over $t=10$ days, then we can compute the risk (volatility of returns over $D$ days) as $S=\sigma\sqrt{t}$. The volatility for Strategy-$X$ is $S_{X}=6.51$ and is higher than  the volatility of 
$S_{Y}=5.15$  for Strategy $Y$. 

From the above, we can compute the corresponding Sharpe 
ratios for the two strategies.
For strategy-$Y$ the Sharpe's ratio (after rounding to decimals) is $\text{SR}_{X}=R_{X}/S_{X}=1.1$, whereas for Strategy-$Y$  the Sharpe's ratio 
is $\text{SR}_{Y}=R_{Y}/S_{Y}=1.0$. Therefore, Strategy-$X$ has a higher absolute return than Strategy-$Y$, but not on a risk-adjusted basis as shown by the Sharpe's ratio.

Next,  we examine the maximum drawdowns. For Strategy-$X$, the maximum decrease was from $B_{2}=103$ to $B_{4}=98$. This gives $\text{MDD}=(B_{4}-B_{2})/B_{1}\approx -5\%$. For Strategy-$Y$, the largest decrease in balance was  from $B_{1}=104$ to $B_{2}=102$. This gives the maximum drawdown $\text{MDD}=(B_{2}-B_{1})/B_{1}\approx -2\%$.
For the S{\&}P-500, the largest decrease was from $B_{2}=104$ to $B_{4}=98$, giving us $\text{MDD}\approx -6\%$.
For the S{\&}P-Growth index,
the maximum decrease was from 
decrease was from $B_{2}=103$ to $B_{4}=99$ giving us 
$\text{MDD}\approx -4\%$. Finally, for the S{\&}P-value index, the maximum decrease was from $B_{2}=103$ to $B_{4}=98$, giving us $\text{MDD}\approx -5\%$.

We summarize the results (after rounding) in Table~\ref{tab_comparison_11_days}
\begin{table}[!h]
    \centering
    \caption{Performance Measures for Strategies}
    \vspace{0.1in}
    \begin{tabular}{ l | rrr | rr } \hline
     \multirow{2}{*}{Metrics} & \multicolumn{3}{c|}{Buy-and-Hold} &
  \multicolumn{2}{c}{Strategies} \\
  & S{\&}P & Growth & Value & $X$ & $Y$ \\ \hline
  final balance $B$ & 104 & 108 & 103 & 106 & 105 \\
   total return $R$  & 4  &   8    &   3   &   6   &   5\\ 
   (daily) st.deviation & 2.2 & 2.4 & 1.7 & 2.1 & 1.6 \\
%   $\mu(r_{i})$     &    0   &   -0.1    &   0.1   &   0.3   &   -0.25\\ 
%  count($r_{i}>0$) &    5   &   4    &   5   &   5   &   3\\ 
%  count($r_{i}<0$) &    5   &   5    &   4   &   4   &   3\\ 
   volatility (risk)  $S$  &  7.0   &   7.5   &  5.3   &   6.5   &   5.2\\ 
   tracking error $E$ & * & 4 & -2 & 2 & 1 \\
    Sharpe ratio $SR$ &  0.6   &   1.1    &   0.6   &   0.9   &   1.0\\ 
% $\rho$ with S{\&}P  &  1   &   0.9   &   0.8   &   0.9   &   0.7\\ 
   MDD - max draw      &    -6   &   -4   &   -5   &   -5   &   -2\\ 
   {\#} trades  &    1  &  1    &   1   &   5   &   5\\ 
    \hline
      \hline
\end{tabular}
\label{tab_comparison_11_days}
\end{table}


% For the Strategy $X$, the average daily return $\mu_{X}(r_{i})=0.3$ was positive, and for the Strategy $Y$
% the average daily return $\mu_{Y}(r_{i})=-0.25$ was negative. The difference in performance lies in the relative advantage of the $X$ in prediction.
% This can be explained in the language of machine learning as follows.


\subsection{Machine Learning Description of Strategy-$X$} 

From Table~\ref{tab_example_11_days} we can represent this strategy Schematically as: 
\begin{equation*}
    \hbox{Strategy-}X:  \quad \overbrace{\underbrace{\underbrace{ d_{1},\; d_{2},\; d_{5},\; d_{6}}_{TP}}_{\text{Growth }r_{G}^{+}}\; , 
    \;\underbrace{\underbrace{d_{8},\; d_{9}}_{FN}}_{\text{Value }r_{V}^{+}}}^{\hbox{true "+" labels } T^{+}}
    \;\; , \;\; 
    \overbrace{
\underbrace{\underbrace{d_{4},\; d_{7}\;}_{TN}}_{\text{Value }r_{V}^{-}}\; , \; 
\underbrace{\underbrace{d_{3},\; d_{10}}_{FP}}_{\text{Growth } r_{G}^{-}}}^{\hbox{true }"-"
\hbox{ labels } T^{-}} 
\end{equation*}
There are $T^{+}=6$ days $d_{1}$, $d_{2}$, $d_{5}$, $d_{6}$, $d_{8}$, and $d_{9}$ with Positive True Labels.
The average daily returns of Growth and Value indices on these days
are 
\begin{equation*}
r_{G}^{+}=(1+2+3+2+5+2)/6=2.5  \quad \hbox{and} \quad r_{V}^{+}=(0+1+2 + 1 + 2+3)/6=1.5
\end{equation*}
There are $T^{-}=4$ days $d_{3}$, $d_{4}$, $d_{7}$,  and $d_{10}$ with Negative True Labels.
The average daily returns of Growth and Value indices on these days
\begin{equation*}
r_{G}^{-}=(-3-1-2-1)/5=-1.8 \qquad \hbox{and} \qquad r_{V}^{-}=(-2-2-1-1)/4=-1.5
\end{equation*}
This gives us $\Delta^{+}=r_{G}^{+}-r_{V}^{+}=1$ and
$\Delta^{-}=r_{V}^{-}-r_{G}^{-} = 0.3$.

Next, we examine the confusion matrix for Strategy-$X$:
\begin{equation}
\text{CF}=\begin{bmatrix} TP & FP\\ FN & TN\end{bmatrix}=
\begin{bmatrix} 4 & 2 \\ 2 & 2 \end{bmatrix}
\label{confusion_x}
\end{equation}

The corresponding recall $\text{TPR}=2/3$, specificity $\text{TNR}=1/2$ and
accuracy $\text{ACC}=3/5$. 
From Table~\ref{tab_comparison_11_days}, we have $R_{X}=6$, $R_{G}=8$ and $R_{V}=3$. From
 equation~\eqref{return_all_strategies} we have
\begin{equation}
\begin{split}
\begin{cases}
      R_{\max} & \approx  T^{+}r_{G}^{+} + T^{-}r_{V}^{-}=
%        (\text{TP} + \text{FN}) \, r_{A}^{+}  +     (\text{TN} + \text{FP})\,  r_{B}^{-}  =
        6\cdot 2.5 + 4(-1.5)=9 \\
    R_{\min} & \approx  T^{+}r_{V}^{+} + T^{-}r_{G}^{-}=
%        (\text{TP} + \text{FN}) \, r_{B}^{+}  +     (\text{TN} + \text{FP})\,  r_{A}^{-} =
    6\cdot 1.5 + 4(-1.8) =1.8\\
    \end{cases}
\end{split}
\label{return_x_example}
\end{equation}

The Predicted Positive Value $\text{PPV}=\text{TP}/(\text{TP}+\text{FP})=2/3$ 
and the Predicted Negative Value
$\text{PNV}=\text{TN}/(\text{TN} + \text{FN}) = 1/2$. The volatility of strategy-$X$ is then
\begin{equation}
S_{X}=\sqrt{ \text{PPV} \cdot S_{A}^{2} + \text{PNV}\cdot S_{B}^{2}}
\approx 6.51
\end{equation}
and this gives us the corresponding Sharpe's ratio $SR_{X}=0.9$

We now compute the return efficiency indices $I_{G}$, $I_{V}$ and $I_{X}$.
From Table~\ref{tab_comparison_11_days} and equation~\eqref{return_x_example} we have
\begin{equation*}
I_{G} = %\frac{R^{G}-R^{\min}}{R^{\max}-R^{\min}} 
=\frac{5}{6}\approx 0.83, \quad 
I_{V} % = \frac{R^{V}-R^{\min}}{R^{\max}-R^{\min}}
=\frac{1}{6}\approx 0.17, \quad 
I_{X} % = \frac{R^{X}-R^{\min}}{R^{\max}-R^{\min}} 
=\frac{7}{12}\approx 0.58
\label{accuracy_index_benchmark_x}
\end{equation*}
To compute the equivalent $p$-random strategy, we obtain from equation~\eqref{equivalent_p}
\begin{equation*}
p=\frac{\text{TPR}\cdot I_{A}-(1-\text{TNR})\cdot I_{B}}{I_{A}-I_{B}}=\frac{17}{24}\approx 0.71
\end{equation*}

\noindent
The strategy recovered about 2/3 of the difference in returns $(R_{\max}-R_{\min})=7.2$ between the best and the worst strategy.
In terms of the return efficiency index, we have $I_{V}<I_{X}<I_{G}$: strategy $X$ outperforms the buy-and-hold in Value but underperforms the buy-and-hold in Growth strategy. And this strategy 
outperforms the random flip strategy with $p\leq 0.71$.

\subsection{Machine Learning Description of Strategy-$Y$} 

For this strategy, security $A$ is the market ("M") - S{\&}P-500 index and security $B$ is cash ('C").
From Table~\ref{tab_example_11_days} we can compute represent for Strategy-$Y$ schematically as:
\begin{equation*}
    \hbox{Strategy-}Y  \quad \overbrace{\underbrace{
    \underbrace{ d_{1},\; d_{2},\; d_{6},\; d_{9}}_{TP}}_{\hbox{S{\&}P}}\; , 
    \;\underbrace{\underbrace{d_{5},\; d_{8}}_{FN}}_{\hbox{cash}}}^{\hbox{true "+" labels }T^{+}}
    \;\; , \;\; 
    \overbrace{
\underbrace{\underbrace{d_{4}}_{TN}}_{\hbox{cash}}\; , \; 
\underbrace{\underbrace{d_{3},\; d_{7},\; d_{10}}_{FP}}_{\hbox{S{\&}P}}}
^{\hbox{true } "-"
\hbox{ labels } T^{-}} 
\end{equation*}

There are $T^{+}=6$ days $d_{1}$, $d_{2}$, $d_{5}$, $d_{6}$, $d_{8}$ and $d_{9}$ with Positive True Labels
and there are $T^{-}=4$ days $d_{3}$, $d_{4}$, $d_{7}$, and $d_{10}$ with negative True Labels
The corresponding average daily returns of S{\&}P-500 (market index) index on these days
are 
\begin{equation*}
r_{M}^{+}=(1+3+2+2+3+1)/6= 2  \quad \hbox{and} 
\quad r_{M}^{-}=(-2-4-1-1)/4 =-2
\end{equation*}
The returns for investing in cash on these days are $r_{C}^{-}=r_{C}^{+}=0$.
This gives $\Delta^{+}=(r_{M}^{+}-r_{C}^{+}) =2$ and $\Delta^{-}=(r_{C}^{-}-r_{M}^{-})=2$.

The confusion matrix is
\begin{equation}
\text{CF}=\begin{bmatrix} TP & FP\\ FN & TN\end{bmatrix}=
\begin{bmatrix} 4 & 3 \\ 2 & 1 \end{bmatrix}
\label{confusion_y}
\end{equation}

\noindent
The corresponding recall $\text{TPR}=2/3$, specificity $\text{TNR}=1/4$ and
accuracy $\text{ACC}=1/2$. 
From Table~\ref{tab_comparison_11_days} we have $R_{Y}=5$, $R_{M}=4$ and $R_{C}=0$. For the returns of the best and worst strategies, from equation~\eqref{return_all_mc_strategies} we have
\begin{equation}
\begin{split}
\begin{cases}
%    R^{Y} & \approx \text{TP}\cdot r_{M}^{+} +      \text{FP}\cdot r_{M}^{-} =5\\
      R_{\max} & \approx  T^{+}\, r_{M}^{+} =12 \\
%        (\text{TP} + \text{FN}) \, r_{A}^{+} 
    R_{\min} & \approx  T^{-}\, r_{M}^{-} =-8\\
%     (\text{TN} + \text{FP})\,  r_{A}^{-} \\
%        R^{M} &  \approx T^{+}\, r_{M}^{+} + T^{-}\, r_{M}^{-} = 4\\
%        R^{{C}} & = 0
%        (\text{TP} + \text{FN}) \, r_{A}^{+}  +     (\text{TN} + \text{FP})\,  r_{A}^{-} \\
    \end{cases}
    \label{return_y_example}
\end{split}
\end{equation}

The Predicted Positive Value $\text{PPV}=\text{TP}/(\text{TP}+\text{FP})=4/7\approx 0.57$ and the Predicted Negative Value
$\text{PNV}=\text{TN}/(\text{TN} + \text{FN}) = 1/3$.
The volatility of strategy-$Y$ is then
\begin{equation}
S_{Y}=S_{M} \sqrt{ \text{PPV}}\approx 5.2
\end{equation}
and this gives us the corresponding Sharpe's ratio $SR_{X}=1.0$


We now compute the return efficiency indices $I_{M}$, $I_{C}$ and $I_{Y}$.
From Table~\ref{tab_comparison_11_days} and equation~\eqref{return_y_example} we have
\begin{equation*}
I_{M} % = \frac{R^{M}-R^{\min}}{R^{\max}-R^{\min}} 
=0.6, \quad
I_{C}  % =\frac{R^{C}-R^{\min}}{R^{\max}-R^{\min}}
=0.4,\quad
I_{Y} % =\frac{R^{Y}-R^{\min}}{R^{\max}-R^{\min}}
=0.65
\label{accuracy_index_benchmark_y}
\end{equation*}

\begin{comment}
The return efficiency index for strategy $Y$ 
\begin{equation*}
I^{Y} = \frac{R^{Y}-R^{\min}}{R^{\max}-R^{\min}}=\frac{13}{20}=0.65
% \text{TPR}\cdot I^{M} + \text{TNR}\cdot I^{C}= (2/3)\cdot 0.6 + 0.25\cdot 0.4=0.5
\label{accuracy_as_benchmarks_y}
\end{equation*}
\end{comment}

To compute the equivalent $p$-random strategy, we obtain from equation~\eqref{equivalent_p}
\begin{equation*}
p=\frac{\text{TPR}\cdot I^{A}-(1-\text{TNR})\cdot I^{B}}{I^{A}-I^{B}}=0.5
\end{equation*}


The strategy recovered $65\%$ or about $2/3$ of the difference in returns $(R_{\max}-R_{\min})=20$ between the best and the worst strategy. Its return efficiency index is greater than that of a random flip strategy.

\begin{comment}
\begin{equation*}
R^{Y} \approx \text{TP}\cdot r^{+} + \text{FP}\cdot r^{-} = 4r^{+}+2r^{-} =4
\end{equation*}
The real value for return is $R_{y}=5$
\end{comment}
In terms of accuracy indices (machine learning-based metric)
$I^{Y}-I^{M}=0.05$. This result tells us that strategy provides very low advantage over buy-and-hold: it is only 5\% better at recovering the potential return compared with buy-and-hold.

\subsection{Machine Learning Comparison of $X$ and $Y$} 


We summarize the comparison results for strategies $X$ and $Y$ in Table~\ref{tab_ml_comparison_example}

\begin{table}[!h]
    \centering
    \caption{Machine Learning Metrics for Strategies}
    \vspace{0.1in}
    \begin{tabular}{ l |  rr } \hline
     \multirow{2}{*}{Metrics} & 
  \multicolumn{2}{c}{\quad Strategies\quad} \\
  & $X$ &  $Y$ \\ \hline
True Positive (TP)  &   4   & 4 \\
False Positive (FP) &   2   &   3\\ 
True Negative (TN)  &   2   &   1\\ 
False Negative (FN) &   2   &   2 \\
Recall (TPR) & 0.67 & 0.6 \\
Specificity (TNR) & 0.5 & 0.25 \\
Accuracy (ACC)           &   0.6   &   0.5\\ 
Precision (PPV) &  0.67 & 0.57\\
Predicted Negative Value (PNV) &  0.5  & 0.33\\
% Return range $(R_{\max}-R_{\min})$      &  7.2 & 20\\
Return Efficiency Index $I$    &   0.58   &   0.65\\
Equivalent Random  probability &   0.71   & 0.5\\
    \hline
      \hline
\end{tabular}
\label{tab_ml_comparison_example}
\end{table}





Strategy $X$ predicts both the positive and negative True labels at a higher rate, resulting in higher overall accuracy ACC (0.6 vs. 05). However, in terms of return efficiency, strategy $Y$ is more efficient: it recovers about 65\% of the potential return vs.
58\% for $X$. Recall that this potential return depends on the benchmarks used and is different for these strategies. For the Growth-Value, the range of returns was 7.2, and for Market-Cash, the range was 20 and was much wider.
For the equivalent random strategy, Growth-Value requires a larger value of $p$ (0.71 vs. 0.5). This is consistent with higher accuracy for $X$ and a lower True negative Rate for $Y$.

Comparing these two strategies, we see that the $X$ strategy has a higher return but this comes at the cost of higher volatility and drawdowns as well as increased trading frequency. The risk-adjusted return of $X$ is lower than for $Y$ as measured by the Sharpe's ratio,

\section{Example: $k$-NN "Winners" and "Losers"  \\ Trading  Strategies}
\label{section_example_knn_strategies}

In the previous sections, we draw an analogy between a trading strategy of choosing the appropriate asset to invest
and a machine learning problem of predicting a label. We explore this idea further and suggest some trading strategies based on analogies to machine learning.

One of the simplest algorithms for classification in machine learning is the so-called $k$-NN - nearest neighbor classification.
In this method, we assume that we are given a distance $D(x,y)$ metric between any two points $x$ and $y$. To assign a label to
any point $x$, we find the closest $k$ labeled points (the so-called "neighbors") $x_{1},\ldots, x_{k}$ of $x$
and assign a label to $x$ based on the majority of labels. The number $k$ must be odd to have a well-defined predicted label.
The simplest case is $k=1$ - we assign a label to $x$ based on the label of its closest neighbor.

An example is illustrated in Figure~\ref{fig:knn_picture}.
\begin{figure}[h!]
  \centering
%  \includegraphics[width=\linewidth]{plots/example_10_days.jpg}   
  \includegraphics[width=0.8\linewidth, height=0.5\linewidth]{plots/knn_simple_example.pdf} 
  \caption{Example of Nearest Neighbor Classification}
  \label{fig:knn_picture}
\end{figure}
In this example, we need to assign a label to point $A$. If we take $k=1$, then the nearest neighbor is point
1 with a "green" label. In this case, we assign the label "green" to $A$. 
If we take $k=3$ neighbors, the nearest three neighbors to $A$ are 
point 1 ("green"),  point 2 ("red") and point 3 ("red"). The majority of these labels are "red" and therefore $A$ 
will be assigned the label "red". Finally, take $k=5$.  The nearest five neighbors to $A$ are point
1 ("green"),  point 2 ("red"), point 3 ("red"), point 4 ("green") and point 5 ("green"). Most of these five points
have the label "green," and therefore, $A$ will be assigned "green".  This example shows that the final label depends on the value $k$.
This value is computed by experiments.

Let us consider a trading strategy based on this analogy. For any two days $d_{i}$ and $d_{j}$ define the distance as the number
of days in between $D(d_{i}, d_{j})=\vert i - j\vert$. With this definition, the neighbors of any day $d_{i}$ are the previous days.
In the simplest case of $k=1$, the neighbor of $d_{i}$ is the previous day $d_{i-1}$ - we assign a predicted label $P_{i}$
to day $d_{i}$ based on the true label $T_{i-1}$ of the previous day $d_{i-1}$. In the more general setting of $k>1$, we assign
a predicted label $P_{i}$ do day $d_{i}$ based on the majority label of $k$ preceeding days
$d_{i-1},d_{i-2},\ldots, d_{i-k}$

If the predicted label $P_{i}$ is taken to be the majority of the true label of $k$ neighbors, we will call such strategy $k$-winners strategy. We denote this strategy as AB-$k$W to indicate that it trades in two securities $A$ and $B$, 
and uses the majority ("winning") of true labels $T_{i-1},\ldots, T_{i-k}$ of the previous $k$ days ("neighbors").
Formally, in AB-$k$W strategy, we assign predicted label $P_{i}$ for day $d_{i}$ as:
\begin{equation*}
P_{i} = \hbox{Majority}(T_{i-k},\ldots,T_{i-1})
\end{equation*}


If the predicted label $P_{i}$ is taken to be the minority of the true labels of $k$ neighbors, we will call such 
strategy $k$-losers strategy. We denote such a strategy as AB-$k$L  to indicate that it 
trades in two securities $A$ and $B$, and uses the minority ("losing") of true labels of the previous $k$ days ("neighbors").
Formally, in AB-$k$L strategies, the predicted label for day $d_{i}$ 
\begin{equation*}
P_{i} = \hbox{Minority}(T_{i-k},\ldots,T_{i-1})
\end{equation*}


\subsection{Market-Cash $k$-NN Strategies}

In this class of strategies, we predict the choice of investments for the next day (market or cash) based on the daily returns of the S{\&}P-500 index over the last $k$ days. 

Recall that in Market-Cash strategies,
to each day $d_{i}$ we assign a True label $T_{i}="+"$  or $T_{i}="-"$ depending on the daily return $r_{i}$ of the S{\&}P-500
index as follows:
\begin{equation*}
T_{i}="+" \hbox{ if } R_{i}\geq 0 \quad\hbox{and}\quad
 T_{i}="-" \hbox{ if } R_{i} < 0
 \end{equation*}

Once the true labels are assigned to each trading day, we generate predicted labels (trading signal) $P_{i+1}$ for the day $d_{i+1}$ based on the majority ("winning") or the minority ("winning") of  True labels in the $k$ previous day(s).

Our trading algorithm invests in day $d_{i+1}$ based on predicted label $P_{i+1}$ for that day as follows:
\medskip
\begin{enumerate}[nosep]
    \item predicted label $P_{i+1}=\, "+"$: (re)invest in S{\&}P-500 Index for day $d_{i+1}$
    \item predicted label $P_{i+1}=\, "-"$: be in Cash for day $d_{i+1}$
\end{enumerate}

\medskip
In MC-$k$W (Market-Cash $k$-winners) strategies, we predict the next day label $P_{i+1}="+"$ if most daily returns of S{\&}P-500 index of the previous $k$ days were non-negative. In such strategies, we tend to believe that the current positive momentum has some inertia and will continue. In MC-$k$L (Market-Cash $k$-losers) strategies, we predict the next day label $P_{i+1}="-"$ if most daily returns of S{\&}P-500 index of the previous $k$ days were positive.  In these strategies, we tend to believe that the current positive momentum is about to change.

\medskip
\noindent
{\bf Example 4: } We illustrate this by the following example of $k=1$ and $k=3$ for 11 days $d_{0}$, $d_{1}$, \ldots, $d_{10}$ with
true and predicted labels as shown in Table~\ref{tab_signals_mc_11}.

\begin{table}[!h]
    \centering
    \caption{Predicted Labels for Different MC-$k$* Trading Strategies}
    \vspace{0.1in}
    \begin{tabular}{l | c c c c c  c c c c c c}
    \hline
        Day & $d_{0}$ & $d_{1}$ & $d_{2}$ & $d_{3}$ & $d_{4}$ & $d_{5}$ & $d_{6}$ & $d_{7}$ & $d_{8}$ & $d_{9}$ & $d_{10}$ \\ \hline
       $T_{i}$\;(MC)  & $+$ & $+$ & $-$ & $-$ & $+$ & $+$ & $+$ & $-$ & $-$ & $+$ & $-$\\

%        B{\&}H  & $+$ & $+$ & $+$ & $+$ & $+$
%         & $+$ & $+$ & $+$ & $+$ & $+$\\
        $P_{i}$\;(MC-1W) & n/a & $+$ & $+$ & $-$ & $-$ & $+$ & $+$ & $+$ & $-$ & $-$ & $+$\\
        $P_{i}$\;(MC-1L)     & n/a & $-$ & $-$ & $+$ & $+$ & $-$ & $-$ & $-$ & $+$ & $+$ & $-$\\
        $P_{i}$\;(MC-3W)  & n/a & n/a & n/a & $+$ & $-$ & $+$ & $+$ & $+$ & $+$ & $-$ & $-$\\
        $P_{i}$\;(MC-3L)  & n/a & n/a & n/a & $-$ & $+$ & $-$ & $-$ & $-$ & $-$ & $+$ & $+$\\
        \hline
\end{tabular}
\label{tab_signals_mc_11}
\end{table}

For $k=3$,  we can assign a signal starting on the day $d_{3}$. For the three preceding ("neighbor") days have true labels 
$T_{0}="+"$, $T_{1}="+"$ and $T_{2}="-"$ respectively. This means that on these days, the S{\&}P-500 index had mostly non-negative daily returns. The majority of these labels are $"+"$ and therefore, the 3-Day Losers strategy assigns predicted label (trading signal) $P_{3}="-"$ for the fourth day $d_{4}$, indicating to be in Cash position.

For example, consider MC-3W (winners) strategy and the assignment of predicted labels starting with day $d_{3}$. The true labels 
for the preceding three ("neighbor") days $d_{0}$, $d_{1}$, and $d_{2}$ were $"+"$, $"+"$ and $"-"$ respectively. The majority of these three labels was $"+"$. Therefore, for $k=3$ nearest neighbor winning strategy, we would assign a predicted label $P_{3}="+"$ for 
the day $d_{3}$ in strategy MC-3W. Therefore, the MC-3W strategy suggests to be invested in S{\&}P index for the day $d_{3}$.
By contrast, for the same day $d_{3}$ we would assign a predicted label $P_{3}="-"$ by the "losers" strategy MC-3L. The MC-3L strategy suggests being in cash position for the day $d_{3}$



\subsection{Growth-Value $k$-NN Strategies}\label{subsec:growth_value_strategies}

In this class of strategies, we predict the choice of investments for the next day (growth or value) based on the relative performance of the corresponding indices over the last $k$ days ("neighbors")  

Recall that for Growth-Value strategies, to each day $d_{i}$ we assign a true label $T_{i}="+"$ or $T_{i}="-"$ depending on
the daily returns $r_{i}^{G}$ and $r_{i}^{V}$ of the Growth and Value indices as follows:
\begin{equation*}
T_{i}="+" \hbox{ if } r_{i}^{G}\geq r_{i}^{V} \quad\hbox{and}\quad
T_{i}="-" \hbox{ if } r_{i}^{G}< r_{i}^{V}
\end{equation*}

Once the true labels are assigned to each trading day, we generate predicted label (trading signal) $P_{i+1}$ for the day $d_{i+1}$ based on the assigned True labels in the previous $k$ day(s). 
Our trading algorithm invests in day $d_{i+1}$ based on predicted label $P_{i+1}$ for that day as follows:
We will consider two trading signals:
\medskip
\begin{enumerate}[nosep]
    \item predicted label $P_{i+1}=\, "+"$: choose S{\&}P-500 Growth index for day $d_{i+1}$
    \item predicted label $P_{i+1}=\, "-"$: choose S{\&}P-500 Value index for day $d_{i+1}$
\end{enumerate}
\medskip
In GV-$k$W (Growth-Value $k$-winners) strategies, we predict the next day label $P_{i+1}="+"$ if most daily returns of S{\&}P Growth index of the previous $k$ days were higher than the daily returns of the S{\&}P Value index. And we predict the next day label $P_{i+1}="-"$ if most daily returns of S{\&}P Growth index of the previous $k$ days were lower than the daily returns of the S{\&}P Value index.
In such strategies, we tend to believe that the current overperformance of one index over the other has some inertia and will continue. By contrast, in GV-$k$L (Growth-Value $k$-losers) strategies, we believe that the current overperformance of one index over another is about to change.

\medskip
\noindent
{\bf Example 5: } We illustrate this by the following example of $k=1$ and $k=3$ for 11 days $d_{0}$, $d_{1}$, \ldots, $d_{10}$ with
true and predicted labels as shown in Table~\ref{tab_signals_gv_11} below:

\begin{table}[!ht]
    \centering
    \caption{Predicted Labels for Different GV-$k$ Trading Strategies}
    \vspace{0.1in}
    \begin{tabular}{l | c c c c c  c c c c c c}
    \hline
        Day & $d_{0}$ & $d_{1}$ & $d_{2}$ & $d_{3}$ & $d_{4}$ & $d_{5}$ & $d_{6}$ & $d_{7}$ & $d_{8}$ & $d_{9}$ & $d_{10}$ \\ \hline
       True Label  & $+$ & $+$ & $-$ & $-$ & $+$ & $+$ & $-$ & $+$ & $-$ & $+$ & $-$\\
%        B{\&}H Growth  & $+$ & $+$ & $+$ & $+$ & $+$
%         & $+$ & $+$ & $+$ & $+$ & $+$\\
%        B{\&}H Value  & $-$ & $-$ & $-$ & $-$ & $-$ & $-$ & $-$ & $-$ & $-$ & $-$\\
 GV-1W     & n/a & $+$ & $+$ & $-$ & $-$ & $+$ & $+$ & $-$ & $+$ & $-$ & $+$\\
 GV-1L     & n/a & $-$ & $-$ & $+$ & $+$ & $-$ & $-$ & $+$ & $-$ & $+$ & $-$\\
 GV-3W & n/a & n/a & n/a & $+$ & $-$ & $-$ & $+$ & $+$ & $+$ & $-$ & $+$\\
 GV-3L & n/a & n/a & n/a & $-$ & $+$ & $+$ & $-$ & $-$ & $-$ & $+$ & $-$\\
        \hline
\end{tabular}
\label{tab_signals_gv_11}
\end{table}

For $k=3$,  we can assign a signal starting on the day $d_{3}$. For the first three 
days $d_{0}$, $d_{1}$ and $d_{2}$ have true labels 
$T_{0}="+"$, $T_{1}="+"$ and $T_{2}="-"$ respectively. This means that on these days, 
the S{\&}P  Growth index overperformed the Value index (in terms of number of days 
when $r_{i}^{G} \geq r_{i}^{V}$). The majority of these true labels are $"+"$ and 
therefore, the 3-Day "Winners" strategy GV-3W assigns predicted label 
(trading signal) $P_{3}="+"$ for the day $d_{3}$, indicating to be invested in
the S{\&}P Growth index. By contrast, the 3-day losers strategy $GV-3L$ assigns 
predicted label (trading signal) $P_{3}="-"$ for the day $d_{3}$, indicating to 
be invested in the S{\&}P Value index. 


\section{Results}\label{section:results}

We now turn to analyze results for the nearest neighbor Growth-Value and 
Market-Cash strategies using Machine learning metrics.
\medskip

We present the following comparisons:
\begin{enumerate}[nosep]
    \item growth comparison
    \item comparison of returns and machine learning metrics
    \item comparison of volatility and drawdowns
    \item comparison of tracking errors and Sharpe ratios
    \item comparison by return efficiency ratio
    \item choosing the number of neighbors $k$ and transaction costs
\end{enumerate}

\subsection{Growth Comparison}\label{subsec:growth_comparison}

We start by considering the three Buy-and-hold strategies investing in the three 
indices S{\&}P-500, S{\&}P-Growth and S{\&}P-Value and in $k=1$ day Growth-Value 
and Market Cash Strategies. 
The growth of {\$}100 for these is shown in Figure~\ref{fig:Comparison of Growth}

\begin{figure}[htbp]
  \centering
  \includegraphics[width=\linewidth]{plots/Comparison of Growth.png}
  \caption{Comparison of Growth}
  \label{fig:Comparison of Growth}
\end{figure}

As can be seen from this graph, the highest growth is achieved by the 
Growth-Value "Loser" strategy GV-1L with $k=1$.
The detailed annual growth is summarized in Table~\ref{tab:end_balances}. 
The summary is provided in
Table~\ref{tab:end_balances_summary}.
\begin{table}[!ht]
\caption{Summary Growth Comparison (from table~\ref{tab:end_balances})}
    \centering
    \begin{tabular}{l | rrr | rr | rr}
    \hline
\multirow{2}{*}{Year} & \multicolumn{3}{c|}{Buy-and-Hold}   
& \multicolumn{2}{c|}{Growth-Value}  &  \multicolumn{2}{c}{Market-Cash}  \\\cline{2-8}
  & S\&P & G & V & GV-1W & GV-1L & MC-1W & MC-1L \\ \hline
  Start Balance & 100 & 100 & 100 & 100  & 100   & 100 & 100 \\
 End Balance    & 555 & 461 & 486 & 85   & 2,642 & 78 & 717 \\ 
 annual IRR  & 7.7 & 6.9 & 7.1 & -0.7 & 15.3 & -1.1 & 8.9 \\ \hline
    \end{tabular}
    \label{tab:end_balances_summary}
\end{table}

The GV-1L strategy ends with the highest balance of {\$}2,642 whereas 
the Market-Cash MC-1W ends with 
{\$}717. Both strategies outperform
the buy-and-hold S{\&}P-500, S{\&}P growth and S{\&}P value strategies 
that yielded {\$}555, {\$}461 and {\$}486 respectively.
Both Growth-Value and Market-Cash "Winner" strategies resulted in a loss. 

The outperformance of GV-1L strategy over other strategies is very 
significant - about 700 basis points 
higher than buy-and-hold strategies and about 600 basis points higher 
than MC-1L. This is somewhat unexpected: intuitively, we would expect the opposite since in Growth-Value strategies we are always invested and the indices themselves are correlated, 
whereas in Market-Cash we seek to avoid losses by having a cash position. 
However, these results tell us 
that our intuition is wrong. One plausible explanation for this is that on most days markets overreact to news 
and this is somewhat corrected in the next day(s).

Therefore, we remove the "winner" strategies from the analysis and focus on a 
comparison of Growth-Value and Market-Cash 
"loser" strategies. We first consider the case $k=1$.


\subsection{Comparison of Returns and Machine Learning \\ Metrics}

Next, we compare annual returns and relate these to machine learning metrics. 
A detailed comparison is presented in the Appendix 
in Table~\ref{tab:comparison_returns_ml_metrics}. A summary of median values in presented in Table~\ref{tab:comparison_returns_ml_metrics_summary}.


\begin{table}[!ht]
    \centering
    \caption{Summary of Comparison of Annual Returns and Machine Learning Metrics (from Table~\ref{tab:comparison_returns_ml_metrics})}
    \medskip
    \begin{tabular}{l || rrr || rr || cc||cc||cc}
    \hline
    \multirow{3}{*}{Year} & \multicolumn{5}{c||}{Annual Returns} & \multicolumn{6}{c}{Recall, Specificity, and Accuracy}  \\
    \cline{2-12}
    & \multicolumn{3}{c||}{Buy-and-Hold}   
& \multicolumn{2}{c||}{$k=1$  day} 
& \multicolumn{2}{c}{TPR}  & \multicolumn{2}{c}{TNR}  & \multicolumn{2}{c}{ACC}  \\ \cline{2-12}
 & S\&P & G & V  & GV  & MC  & GV & MC  & GV  & MC & GV & MC \\ \hline
        $\max$ & 32.3 & \cellcolor{green!25}37.0 & \cellcolor{red!25}31.8 & \cellcolor{green!25}51.4 & 36.1 & \cellcolor{green!25}69 & 61 & 63 & \cellcolor{green!25}67 & \cellcolor{green!25}64 & 59 \\ 
        $\min$ & -36.8 & \cellcolor{red!25}-37.4 & \cellcolor{green!25}-36.3 & -26.5 & \cellcolor{green!25}-12.6 & \cellcolor{green!25}38 & \cellcolor{green!25}38 & \cellcolor{green!25}46 & 43 & 45 & \cellcolor{green!25}45 \\ 
        $M$ & \cellcolor{green!25}13.5 & \cellcolor{red!25}10.8 & 13.2 & \cellcolor{green!25}17.0 & 10.8 & \cellcolor{green!25}53 & 45 & 55 & \cellcolor{green!25}57 & \cellcolor{green!25}53 & 51 \\ 
%        $\mu$ & \cellcolor{green!25}9.4 & \cellcolor{green!25}9.4 & \cellcolor{red!25}8.5 & \cellcolor{green!25}16.6 & 9.5 & \cellcolor{green!25}52 & 47 & 55 & \cellcolor{green!25}57 & \cellcolor{green!25}53 & 51 \\ 
%        $\sigma$ & 18.3 & \cellcolor{green!25}22.3 & \cellcolor{red!25}16.6 & \cellcolor{green!25}17.4 & 10.8 & \cellcolor{green!25}7 & 6 & 5 & \cellcolor{green!25}6 & \cellcolor{green!25}4 & 4 \\ 

\hline
    \end{tabular}
    \label{tab:comparison_returns_ml_metrics_summary}
\end{table}

In 23 years from 2001 to 2023, compared with the MC-1W, the GV-1L strategy has a higher True Positive Rate in 17 years, higher TNR in 10 years, and higher overall accuracy in 18 years. The difference in TNR was quite significant (53 vs. 45 median values),
the underperformance in TNR was minor (55 vs. 57) as was the overall accuracy (53 vs. 51). Nevertherless, 
the significant over-performance in TPR resulted in significant overperformance of the GV-1L strategy.

\subsection{Comparison of Volatility and Drawdowns}

next, we consider the volatility. We saw in Section~\ref{section_volatility} that increasing TPR increases both the returns and the volatility whereas increasing TNR increases the return by a smaller amount and decreases volatility. 
increase in TNR. A detailed comparison of volatility and drawdowns is presented in the Appendix
in Table~\ref{tab:comparison_drawdown}.
A summary of median values is presented in Table~\ref{tab:comparison_drawdown_summary}.

\begin{table}[!ht]
    \centering
    \caption{Summary of Annual Maximum Drawdown and Volatility of Investment Strategies (from Table~\ref{tab:comparison_drawdown})}
    \medskip
    \begin{tabular}{l || rrr || rr || rrr || rr }
    \hline
\multirow{3}{*}{Year} & \multicolumn{5}{c||}{Maximum Drawdowns}& \multicolumn{5}{c}{Annual Volatility}     \\\cline{2-11}
 & \multicolumn{3}{c||}{Buy-and-Hold}   
& \multicolumn{2}{c||}{$k=1$  day} &   \multicolumn{3}{c||}{Buy-and-Hold}   & \multicolumn{2}{c}{$k=1$  day}   \\  \cline{2-11}
  & S\&P & G & V   & GV & MC & S\&P & G & V  & GV &  MC \\ \hline
         $\max$ & -2.6 & \cellcolor{green!25}-2.4 & \cellcolor{red!25}-4.4 & \cellcolor{red!25}-3.3 & -1.8 & 21.9 & \cellcolor{green!25}10.9 & \cellcolor{red!25}40.9 & \cellcolor{red!25}39.9 & 33.3 \\ 
        $\min$ & -47.1 & \cellcolor{red!25}-48.6 & \cellcolor{green!25}-47.0 & \cellcolor{red!25}-41.8 & -20.9 & \cellcolor{green!25}6.7 & \cellcolor{red!25}7.2 & \cellcolor{red!25}7.2 & \cellcolor{red!25}7.2 & 4.9 \\ 
        $M$ & \cellcolor{green!25}-10.0 & -10.8 & \cellcolor{red!25}-12.0 & \cellcolor{red!25}-10.8 & -7.0 & \cellcolor{red!25}10.0 & \cellcolor{green!25}9.0 & 9.8 & \cellcolor{red!25}15.7 & 10.9 \\ 
%        $\mu$ & \cellcolor{green!25}-15.7 & \cellcolor{red!25}-17.2 & \cellcolor{green!25}-15.7 & \cellcolor{red!25}-15.2 & -10.2 & 12.9 & \cellcolor{green!25}9.0 & \cellcolor{red!25}17.4 & \cellcolor{red!25}18.6 & 13.0 \\ 
%        $\sigma$ & 11.5 & \cellcolor{green!25}13.6 & \cellcolor{red!25}11.2 & 10.8 & \cellcolor{red!25}6.2 & 8.0 & \cellcolor{green!25}2.6 & \cellcolor{red!25}13.9 & \cellcolor{red!25}9.0 & 6.8 \\ 
\hline
    \end{tabular}
    \label{tab:comparison_drawdown_summary}
\end{table}

Not surprisingly, much higher TNR for GV-1L translated into significantly higher volatility.
In fact, in each of the 23 years from 2001 to 2023, the volatility of GV-1L strategy was higher than the volatility of MC-1L
by about 50\% as measured by the median values (15.7 vs. 10.9). For the maximum drawdowns, GV-1L had higher drawdowns than
MC-1L in 19 out of 23 years, also about 50\% as measured by the median values ($-10.8\%$ vs.  $-7\%$). 


% -----------------

\begin{table}[!ht]
    \centering
    \caption{Summary Comparison of Annual Volatility and Machine Learning 
    Metrics (from Table~\ref{tab:comparison_volatility_ml_metrics})}
    \medskip
    \begin{tabular}{l || rrr || rr || cc||cc||cc}
    \hline
    \multirow{3}{*}{Year} & \multicolumn{5}{c||}{Annual Volatility} & \multicolumn{6}{c}{Precision, NPV, and Prevalence} \\
    \cline{2-12}
    & \multicolumn{3}{c||}{Buy-and-Hold}   
& \multicolumn{2}{c||}{$k=1$  day} 
& \multicolumn{2}{c}{PPV}  & \multicolumn{2}{c}{NPV} & \multicolumn{2}{c}{$\pi^+$}   \\ \cline{2-12}
 & S\&P & G & V  & GV  & MC  & GV & MC  & GV  & MC & GV & MC \\ \hline
        $\max$ & 21.9 & \cellcolor{green!25}10.9 & \cellcolor{red!25}40.9 & \cellcolor{red!25}39.9 & 33.3 & 63 & \cellcolor{green!25}67 & \cellcolor{green!25}68 & 61 & 59 & \cellcolor{green!25}60 \\ 
        $\min$ & \cellcolor{green!25}6.7 & \cellcolor{red!25}7.2 & \cellcolor{red!25}7.2 & \cellcolor{red!25}7.2 & 4.9 & \cellcolor{green!25}45 & 43 & \cellcolor{green!25}38 & \cellcolor{green!25}38 & 43 & \cellcolor{green!25}44 \\ 
        $M$ & \cellcolor{red!25}10.0 & \cellcolor{green!25}9.0 & 9.8 & \cellcolor{red!25}15.7 & 10.9 & 56 & \cellcolor{green!25}57 & \cellcolor{green!25}53 & 45 & 50 & \cellcolor{green!25}56 \\ 
%        $\mu$  & 12.9 & \cellcolor{green!25}9.0 & \cellcolor{red!25}17.4 & \cellcolor{red!25}18.6 & 13 & 55 & \cellcolor{green!25}57 & \cellcolor{green!25}52 & 47 & 51 & \cellcolor{green!25}55 \\ 
%        $\sigma$ & 8 & \cellcolor{green!25}2.6 & \cellcolor{red!25}13.9 & \cellcolor{red!25}9.0 & 6.8 & 5 & \cellcolor{green!25}6 & \cellcolor{green!25}7 & 6 & \cellcolor{green!25}5 & 4 \\ 
        \hline
    \end{tabular}
    \label{tab:comparison_volatility_ml_metrics_summary}
\end{table}


\subsection{Comparison of Tracking Errors and Sharpe Ratios}

\begin{table}[!ht]
    \centering
    \caption{Summary Comparison of Tracking Errors 
    and Sharp Ratios (from Table~\ref{tab:comparison_tracking_errors_sharp_ratios})}
    \medskip
\begin{tabular}{ c | r || rr || rr || rr}
\hline
\multirow{2}{*}{Year} & \multicolumn{1}{c||}{Return}   & \multicolumn{4}{c||}{Tracking Error} 
& \multicolumn{2}{c}{Sharp Ratio}   \\ \cline{2-8}
  & S\&P & G & V  & GV  & MC  & GV  & MC  \\ 
\hline
        $\max$ & 32.3 & \cellcolor{green!25}15.1 & 12.9 & \cellcolor{green!25}55.7 & 47.3 & \cellcolor{green!25}3.7 & 2.7 \\ 
        $\min$ & -36.8 & \cellcolor{green!25}-14.1 & -17 & \cellcolor{green!25}-12.4 & -22 & \cellcolor{green!25}-0.7 & -0.9 \\ 
        $M$ & 13.5 & \cellcolor{green!25}0.3 & -0.5 & \cellcolor{green!25}5.1 & -5.3 & \cellcolor{green!25}1.2 & 1 \\ 
%        $\mu$ & 9.4 & \cellcolor{green!25}0.0 & -0.9 & \cellcolor{green!25}7.2 & 0 & \cellcolor{green!25}1.2 & 0.9 \\ 
%        $\sigma$ & 18.3 & \cellcolor{green!25}6.8 & 6 & 14.4 & \cellcolor{green!25}16.1 & \cellcolor{green!25}1.1 & 0.9 \\ 
\hline
\end{tabular}
\label{tab:comparison_tracking_errors_sharpe_ratios_summary}
\end{table}

\subsection{Comparison of Strategies by Return Efficiency Ratios}

So far, we compared the strategies to each other by examining returns, volatilities and Sharpe's ratio. 
We now ask the question: how efficient are the strategies and do they outperform a simple random flip strategy.
To that end, we compute the return efficiency index of the strategies and compare it to the return efficiency index of random flip.

The results are presented in Figure~\ref{fig:Return Capture}.

\begin{figure}[htbp]
  \centering
  \includegraphics[width=\linewidth]{plots/Return Capture.png}
  \caption{Return Efficiency Index of Strategies}
  \label{fig:Return Capture}
\end{figure}


The detailed comparison is presented in the Appendix in Table~\ref{tab:return_effiency_ratios}. A Summary is presented in
Table~\ref{tab:return_effiency_ratios_summary}.

\begin{table}[!ht]
    \centering
    \caption{Summery of Return Efficiency Ratios 
    (from Table~\ref{tab:return_effiency_ratios})}
    \begin{tabular}{l| ccc||cc||cc}
    \hline
    \multirow{2}{*}{Year} & \multicolumn{3}{c||}{Buy-and-Hold}   
& \multicolumn{2}{c||}{Growth-Value}  &  \multicolumn{2}{c}{Market-Cash}  \\\cline{2-8}
         & S\&P & G  & V & GV-1L& Random  & MC-1L& Random \\ \hline
        $\max$ & 0.43 & 0.49 & 0.51 & \cellcolor{green!25}0.57 & 0.43 & \cellcolor{green!25}0.65 & 0.43 \\ 
        $\min$ & 0.13 & 0.11 & 0.14 & \cellcolor{green!25}0.18 & 0.13 & \cellcolor{green!25}0.14 & 0.14 \\ 
        $M$ & 0.39 & 0.39 & 0.35 & \cellcolor{green!25}0.41 & 0.39 & \cellcolor{green!25}0.32 & 0.31 \\ 
%        $\mu$  & 0.36 & 0.36 & 0.35 & \cellcolor{green!25}0.40 & 0.35 & \cellcolor{green!25}0.34 & 0.31 \\ 
%        $\sigma$ & 0.08 & 0.10 & 0.08 & \cellcolor{green!25}0.10 & 0.08 & \cellcolor{green!25}0.12 & 0.07 \\   
\hline
    \end{tabular}
    \label{tab:return_effiency_ratios_summary}
\end{table}

In 23 years from 2001-20023,  in terms of Return Efficiency Index, the GV-1L outperformed the random flip strategy
in 16 year. The MC-1L outperformed the random flip in 14 years. The median value of the Return efficiency index for $GV-1L$ was
0.41 and is about 25\% much higher than the median value of 0.32 for MC-1L. Not only is GV-1L gives higher return and Sharpe's ratio, it is also more efficient as a strategy in capturing the potential returns range.



\begin{table}[!ht]
    \centering
    \caption{Summary Annual Metrics by Investment Type and 
    Market Condition (from Table~\ref{tab:comparison_ml_metrics})}
    \begin{tabular}{l| cc | cc || rrrr|rr}
    \hline
    \multirow{3}{*}{Year} & \multicolumn{4}{c||}{True Labels}   & \multicolumn{6}{c}{Average Return for True Labels}  \\ \cline{2-11}
     & \multicolumn{2}{c|}{GV}  &  \multicolumn{2}{c||}{MC}  & \multicolumn{4}{c|}{GV}  &  \multicolumn{2}{c}{MC}  \\ \cline{2-11}
    & $T^{+}$ & $T^{-}$ & $T^{+}$ & $T^{-}$ & $r^{+}_{G}$ & $r^{-}_{G}$ & $r^{+}_{V}$ & $r^{-}_{V}$ & $r^{+}$ & $r^{-}$ \\ \cline{1-11}
         $\max$ & 150 & 142 & 150 & 141 & 1.54 & 0.12 & 0.48 & 0.71 & 1.56 & -0.27 \\ 
        $\min$ & 109 & 103 & 110 & 102 & -0.31 & -1.64 & -1.02 & -0.42 & 0.33 & -1.84 \\ 
        $M$ & 127 & 125 & 141 & 111 & 0.26 & -0.16 & -0.08 & 0.22 & 0.67 & -0.68 \\ 
%        $\mu$ & 129 & 122 & 138 & 114 & 0.38 & -0.30 & -0.12 & 0.21 & 0.78 & -0.83 \\ 
%        $\sigma$ & 11.43 & 11.29 & 10.6 & 10.4 & 0.42 & 0.47 & 0.28 & 0.26 & 0.32 & 0.37 \\ 
        \hline
    \end{tabular}
    \label{tab:comparison_ml_metrics_summary}
\end{table}

\subsection{Transaction Costs}



\begin{figure}[htbp]
  \centering
  \includegraphics[width=\linewidth]{plots/Return Efficient Index For Different K.png}
  \caption{Return Efficiency Index For Different $k$}
  \label{fig:Return Efficient Index For Different K}
\end{figure}

\subsection{Comparing by number of transactions and average positive and negative returns}




\begin{comment}
\begin{figure}[htbp]
  \centering
  \includegraphics[width=\linewidth]{plots/gv vs mc.png}
  \caption{Predict Labels of GV-1L And MC-1L Strategy}
  \label{fig:winer_loser_fridays}
\end{figure}
\end{comment}



\subsection{Choosing the Number of Nearest Neighbors}\label{section_choosing_k}

In Nearest Neighbor algorithms, one of the important hyperparameters (besides the distance metric) is $k$ - the number of neighbors to use. Since the decision on predicting labels is made by the majority (minority) of true labels, the number $k$ must be odd and, in the simplest case, $k=1$.

We start by comparing the Growth for different $k$. The results are presented in
Figure~\ref{fig:Growth, Maximum Drawdown And Volatility For Different $k$}.

We can see that as we increase $k$, the performance of the Growth-Value strategy degrades rapidly. Intuitively, we can explain this as follows: in daily movements, the market overreacts to positive and negative news. This is corrected in the very near term (the next day). However, this effect is very short-lived. For very large $k$, we will end up predicting labels depending on the proportion of true labels in the past $k$ trading days for both strategies.

Next, we compare Maximum drawdowns. This is presented in 
Figure~\ref{fig:Growth, Maximum Drawdown And Volatility For Different $k$}
\begin{comment}
    

\begin{figure}[h!]
  \centering
  \includegraphics[width=\linewidth]{plots/drawdown.png}
  \caption{Comparison of  for Different $k$}
  \label{fig:mdd_winner_loser_fridays}
\end{figure}
\end{comment}
As we can see from this graph, the MDD for both strategies remain stable with MDD for Growth-Value strategy
about 50\% higher than for Market-Cash strategies.

A comparison of annualized volatility is shown in 
Figure~\ref{fig:Growth, Maximum Drawdown And Volatility For Different $k$}
\begin{comment}
\begin{figure}[htbp]
  \centering
  \includegraphics[width=\linewidth]{plots/volatility.png}
  \caption{Comparing  for Different $k$}
  \label{fig:volatility_winner_loser_fridays}
\end{figure}
\end{comment}
Just as is the case with maximum drawdowns, this value does not change with $k$.
On average, the annualized volatility of Growth-Value is about 18\% whereas the volatility 
of Market-Cash is about 13\%. The volatility of the Growth-Value strategy is about 50\% higher than 
Market-Cash that is similar to the volatility of the S{\&}P-500 index.


Since the returns decrease for larger $k$ whereas volatility does not change, we see that the 
risk-adjusted returns as measured by Sharp's ratio will decrease for larger $k$.

Finally, we compare the transaction costs for different $k$ in Table~\ref{tab:transaction_costs_summary}.

\begin{table}[!ht]
    \centering
    \caption{Summary of Number of Transactions for Different $k$ (from Table~\ref{tab:transaction_costs}}
\begin{tabular}{r| ccc|| cccc||cccc}
\hline
\multirow{2}{*}{Year} & \multicolumn{3}{c||}{Buy-and-Hold}   & \multicolumn{4}{c||}{GV-$k$L Strategy}& \multicolumn{4}{c}{MC-$k$L Strategy}\\ \cline{2-12}
  & S\&P & G & V  & $k=1$  & $3$  & $5$  & $7$  & $k=1$  & $3$  & $5$  & $7$ \\ 
\hline
%Year & S\&P & Growth & Value & 1 day & 3 day & 5 day & 7 day \\
%\hline
        $\max$ & 1 & 1 & 1 & 160 & 92 & 72 & 56 & 146 & 86 & 70 & 58 \\ 
        $\min$ & 1 & 1 & 1 & 114 & 55 & 38 & 24 & 112 & 48 & 33 & 26 \\ 
        $M$ & 1 & 1 & 1 & 132 & 69 & 49 & 40 & 127 & 61 & 46 & 39 \\ 
%        $\mu$ & 1 & 1 & 1 & 133 & 69 & 51 & 41 & 128 & 63 & 47 & 39 \\ 
%        $\sigma$ & 0 & 0 & 0 & 11 & 9 & 9 & 8 & 10 & 9 & 9 & 8 \\ 
\hline
\end{tabular}
\label{tab:transaction_costs_summary}
\end{table}


The number of transactions is comparable for both Growth Value and 
Market Cash strategies and drops rapidly as we increase
$k$. 








\begin{comment}
The results are presented in Table~\ref{tab_comparison}.
S{\&}P refers to the Standard \& Poor's 500 Index. Growth and Value represent Buy-Hold for SPYG and SPYV indices respectively.
"$\max$" and "$\min$" strategies are based on selecting the higher growth or value from the previous day. Finally, 5-year, 3-year, and 1-year indicate the training periods of five, three, and one year, respectively.

In Table~\ref{tab_comparison_example}, each strategy's performance is symbolized as follows: 'G' for Growth, indicating an investment in SPYG, and 'V' for Value, indicating an investment in SPYV. The 'Label' column classifies daily market outcomes based on the comparative performance of SPYV versus SPYG; a '+' label indicates a day where SPYV outperformed SPYG, and a '-' label signifies the reverse.The 'Winner' and 'Loser' 
columns are derived from the prior day's data (hence 'N/A' for the initial date), 

\end{comment}


In Table~\ref{tab:comparison_drawdown} we compare the drawdowns of the strategies.







\section{Concluding Remarks}\label{section_concluding_remarks}




\medskip
\medskip
\medskip
\medskip
\noindent
{\bf DECLARATIONS}

\medskip
\noindent
{\bf Conflict of Interest}
We declare that there are no conflicts of interest regarding the publication of this paper.

\medskip
\noindent
{\bf Author Contributions}
All authors contributed equally to the effort.

\medskip
\noindent
{\bf Funding}
This research was conducted without any external funding. All aspects of the study, including design, data collection, analysis, and interpretation, were carried out using the resources available within the authors' institution.

\begin{comment}
\medskip
\noindent
{\bf Acknowledgements}
We want to thank Metropolitan College Boston University, for their support.
\end{comment}


\medskip
\noindent
{\bf Data Availability}
All the relevant data and analysis are available via:
\begin{verbatim}
https:????
\end{verbatim}













\newpage

\bibliographystyle{plain}
\bibliography{references}

\newpage
\section{Appendix: Detailed Tables and Figures}

This Appendix contains tables with detailed annual statistics for strategies

\begin{table}[!ht]
    \centering
    \caption{Comparison of Annual End Balances of Investment Strategies.}
    \medskip
    \begin{tabular}{l | rrr | rr | rr}
    \hline
\multirow{2}{*}{Year} & \multicolumn{3}{c|}{Buy-and-Hold}   
& \multicolumn{2}{c|}{Growth-Value}  &  \multicolumn{2}{c}{Market-Cash}  \\\cline{2-8}
  & S\&P & G & V & GV-1W & GV-1L & MC-1W & MC-1L \\ \hline
        2001 & 88 & 74 & 94 & 60 & 117 & 93 & 95 \\ 
        2002 & 69 & 50 & 77 & 25 & 157 & 67 & 103 \\ 
        2003 & 89 & 65 & 97 & 26 & 238 & 73 & 122 \\ 
        2004 & 98 & 68 & 110 & 27 & 281 & 77 & 127 \\ 
        2005 & 103 & 70 & 116 & 26 & 308 & 73 & 141 \\ 
        2006 & 119 & 76 & 141 & 29 & 367 & 81 & 147 \\ 
        2007 & 125 & 84 & 143 & 29 & 413 & 71 & 176 \\ 
        2008 & 79 & 53 & 91 & 16 & 303 & 41 & 194 \\ 
        2009 & 100 & 72 & 106 & 19 & 408 & 43 & 234 \\ 
        2010 & 115 & 84 & 123 & 21 & 501 & 43 & 265 \\ 
        2011 & 117 & 88 & 122 & 20 & 547 & 46 & 256 \\ 
        2012 & 136 & 100 & 143 & 19 & 745 & 53 & 258 \\ 
        2013 & 180 & 133 & 188 & 24 & 1,045 & 60 & 299 \\ 
        2014 & 204 & 153 & 211 & 26 & 1,222 & 62 & 332 \\ 
        2015 & 207 & 161 & 205 & 26 & 1,278 & 62 & 333 \\ 
        2016 & 232 & 172 & 240 & 29 & 1,419 & 59 & 392 \\ 
        2017 & 282 & 218 & 276 & 35 & 1,720 & 63 & 445 \\ 
        2018 & 269 & 218 & 252 & 34 & 1,604 & 69 & 388 \\ 
        2019 & 353 & 285 & 331 & 47 & 2,016 & 83 & 424 \\ 
        2020 & 418 & 381 & 336 & 60 & 2,137 & 72 & 577 \\ 
        2021 & 538 & 503 & 419 & 85 & 2,486 & 82 & 659 \\ 
        2022 & 440 & 355 & 397 & 65 & 2,161 & 72 & 615 \\ 
        2023 & 555 & 461 & 486 & 85 & 2,642 & 78 & 717 \\ \hline
        IRR & 7.7 & 6.9 & 7.1 & -0.7 & 15.3 & -1.1 & 8.9 \\ \hline
%                $\max$ & 555 & 503 & 486 & 85 & 2,642 & 93 & 717 \\ 
%        $\min$ & 69 & 50 & 77 & 16 & 117 & 41 & 95 \\ 
%        $M$ & 136 & 100 & 143 & 27 & 745 & 69 & 265 \\ 
%        $\mu$  & 214 & 171 & 205 & 36 & 1049 & 66 & 317 \\ 
%         $\sigma$ & 150 & 136 & 119 & 21 & 822 & 14 & 185 \\  \hline
    \end{tabular}
    \label{tab:end_balances}
\end{table}


\newpage
\begin{table}[!ht]
    \centering
    \caption{Comparison of Annual Returns and Machine Learning Metrics}
    \medskip
    \begin{tabular}{l || rrr || rr || cc||cc||cc}
    \hline
    \multirow{3}{*}{Year} & \multicolumn{5}{c||}{Annual Returns} & \multicolumn{6}{c}{Recall, Specificity, and Accuracy}  \\
    \cline{2-12}
    & \multicolumn{3}{c||}{Buy-and-Hold}   
& \multicolumn{2}{c||}{$k=1$  day} 
& \multicolumn{2}{c}{TPR}  & \multicolumn{2}{c}{TNR}  & \multicolumn{2}{c}{ACC}  \\ \cline{2-12}
 & S\&P & G & V  & GV  & MC  & GV & MC  & GV  & MC & GV & MC \\ \hline
        2001 & -11.8 & \cellcolor{red!25}-25.9 & \cellcolor{green!25}-5.6 & \cellcolor{green!25}17.0 & -5.0 & \cellcolor{green!25}55 & 51 & \cellcolor{green!25}53 & 52 & \cellcolor{green!25}54 & 52 \\ 
        2002 & -21.6 & \cellcolor{red!25}-32.1 & \cellcolor{green!25}-18.0 & \cellcolor{green!25}34.2 & 8.9 & \cellcolor{green!25}69 & 61 & \cellcolor{green!25}60 & 55 & \cellcolor{green!25}64 & 58 \\ 
        2003 & 28.2 & \cellcolor{green!25}28.3 & \cellcolor{red!25}25.2 & \cellcolor{green!25}51.4 & 17.8 & \cellcolor{green!25}57 & 46 & 58 & \cellcolor{green!25}61 & \cellcolor{green!25}58 & 52 \\ 
        2004 & 10.7 & \cellcolor{red!25}5.3 & \cellcolor{green!25}13.2 & \cellcolor{green!25}18.4 & 4.5 & \cellcolor{green!25}58 & 41 & 54 & \cellcolor{green!25}57 & \cellcolor{green!25}56 & 48 \\ 
        2005 & 4.8 & \cellcolor{red!25}2.8 & \cellcolor{green!25}5.4 & 9.6 & \cellcolor{green!25}10.8 & \cellcolor{green!25}55 & 46 & 50 & \cellcolor{green!25}58 & \cellcolor{green!25}52 & \cellcolor{green!25}52 \\ 
        2006 & 15.8 & \cellcolor{red!25}9.0 & \cellcolor{green!25}21.6 & \cellcolor{green!25}19.0 & 4.3 & \cellcolor{green!25}61 & 41 & 46 & \cellcolor{green!25}53 & \cellcolor{green!25}53 & 46 \\ 
        2007 & 5.1 & \cellcolor{green!25}10.8 & \cellcolor{red!25}1.4 & 12.5 & \cellcolor{green!25}19.3 & \cellcolor{green!25}51 & \cellcolor{green!25}51 & \cellcolor{green!25}63 & \cellcolor{green!25}63 & \cellcolor{green!25}57 & \cellcolor{green!25}57 \\ 
        2008 & -36.8 & \cellcolor{red!25}-37.4 & \cellcolor{green!25}-36.3 & -26.5 & \cellcolor{green!25}10.6 & \cellcolor{green!25}53 & 50 & \cellcolor{green!25}52 & 49 & \cellcolor{green!25}53 & 49 \\ 
        2009 & 26.4 & \cellcolor{green!25}37.0 & \cellcolor{red!25}17.1 & \cellcolor{green!25}34.4 & 20.3 & \cellcolor{green!25}46 & 44 & \cellcolor{green!25}63 & 57 & \cellcolor{green!25}53 & 50 \\ 
        2010 & \cellcolor{red!25}15.1 & \cellcolor{green!25}16.2 & 15.5 & \cellcolor{green!25}22.7 & 13.5 & \cellcolor{green!25}54 & 43 & 54 & \cellcolor{green!25}59 & \cellcolor{green!25}54 & 50 \\ 
        2011 & 1.9 & \cellcolor{green!25}4.6 & \cellcolor{red!25}-0.7 & \cellcolor{green!25}9.2 & -3.4 & \cellcolor{green!25}55 & 43 & \cellcolor{green!25}55 & 52 & \cellcolor{green!25}55 & 47 \\ 
        2012 & 16.0 & \cellcolor{red!25}14.2 & \cellcolor{green!25}17.2 & \cellcolor{green!25}36.3 & 0.7 & \cellcolor{green!25}65 & 41 & \cellcolor{green!25}56 & 50 & \cellcolor{green!25}60 & 45 \\ 
        2013 & 32.3 & \cellcolor{green!25}32.6 & \cellcolor{red!25}31.8 & \cellcolor{green!25}40.3 & 16.0 & \cellcolor{green!25}53 & 43 & 59 & \cellcolor{green!25}62 & \cellcolor{green!25}56 & 51 \\ 
        2014 & 13.5 & \cellcolor{green!25}14.8 & \cellcolor{red!25}12.2 & \cellcolor{green!25}16.9 & 11.0 & \cellcolor{green!25}47 & 44 & 59 & \cellcolor{green!25}64 & \cellcolor{green!25}52 & \cellcolor{green!25}52 \\ 
        2015 & 1.2 & \cellcolor{green!25}5.1 & \cellcolor{red!25}-3.2 & \cellcolor{green!25}4.6 & 0.3 & 48 & \cellcolor{green!25}55 & \cellcolor{green!25}58 & 50 & \cellcolor{green!25}53 & 52 \\ 
        2016 & 12.0 & \cellcolor{red!25}6.8 & \cellcolor{green!25}17.1 & 11.1 & \cellcolor{green!25}17.9 & 49 & \cellcolor{green!25}54 & 51 & \cellcolor{green!25}65 & 50 & \cellcolor{green!25}59 \\ 
        2017 & 21.7 & \cellcolor{green!25}27.2 & \cellcolor{red!25}15.4 & \cellcolor{green!25}21.2 & 13.4 & 46 & \cellcolor{green!25}47 & 62 & \cellcolor{green!25}63 & \cellcolor{green!25}53 & \cellcolor{green!25}53 \\ 
        2018 & -4.6 & \cellcolor{green!25}-0.1 & \cellcolor{red!25}-9.0 & \cellcolor{green!25}-6.7 & -12.6 & 42 & \cellcolor{green!25}44 & \cellcolor{green!25}55 & 51 & \cellcolor{green!25}47 & \cellcolor{green!25}47 \\ 
        2019 & 31.2 & \cellcolor{red!25}30.8 & \cellcolor{green!25}31.7 & \cellcolor{green!25}25.7 & 9.2 & \cellcolor{green!25}47 & 38 & 46 & \cellcolor{green!25}56 & \cellcolor{green!25}46 & 45 \\ 
        2020 & 18.3 & \cellcolor{green!25}33.5 & \cellcolor{red!25}1.4 & 6 & \cellcolor{green!25}36.1 & 38 & \cellcolor{green!25}49 & 55 & \cellcolor{green!25}67 & 45 & \cellcolor{green!25}57 \\ 
        2021 & 28.7 & \cellcolor{green!25}32.0 & \cellcolor{red!25}24.9 & \cellcolor{green!25}16.3 & 14.1 & \cellcolor{green!25}47 & 43 & 46 & \cellcolor{green!25}60 & 47 & \cellcolor{green!25}50 \\ 
        2022 & -18.2 & \cellcolor{red!25}-29.4 & \cellcolor{green!25}-5.3 & -13.1 & \cellcolor{green!25}-6.6 & \cellcolor{green!25}57 & 55 & \cellcolor{green!25}46 & 43 & \cellcolor{green!25}51 & 49 \\ 
        2023 & 26.2 & \cellcolor{green!25}30.0 & \cellcolor{red!25}22.2 & \cellcolor{green!25}22.2 & 16.5 & 44 & \cellcolor{green!25}45 & \cellcolor{green!25}58 & \cellcolor{green!25}58 & \cellcolor{green!25}50 & \cellcolor{green!25}50 \\ \hline
        $\max$ & 32.3 & \cellcolor{green!25}37.0 & \cellcolor{red!25}31.8 & \cellcolor{green!25}51.4 & 36.1 & \cellcolor{green!25}69 & 61 & 63 & \cellcolor{green!25}67 & \cellcolor{green!25}64 & 59 \\ 
        $\min$ & -36.8 & \cellcolor{red!25}-37.4 & \cellcolor{green!25}-36.3 & -26.5 & \cellcolor{green!25}-12.6 & \cellcolor{green!25}38 & \cellcolor{green!25}38 & \cellcolor{green!25}46 & 43 & \cellcolor{green!25}45 & \cellcolor{green!25}45 \\ 
        $M$ & \cellcolor{green!25}13.5 & \cellcolor{red!25}10.8 & 13.2 & \cellcolor{green!25}17.0 & 10.8 & \cellcolor{green!25}53 & 45 & 55 & \cellcolor{green!25}57 & \cellcolor{green!25}53 & 50 \\ 
        $\mu$ & \cellcolor{green!25}9.4 & \cellcolor{green!25}9.4 & \cellcolor{red!25}8.5 & \cellcolor{green!25}16.6 & 9.5 & \cellcolor{green!25}52 & 47 & 55 & \cellcolor{green!25}57 & \cellcolor{green!25}53 & 51 \\ 
        $\sigma$ & 18.3 & \cellcolor{green!25}22.3 & \cellcolor{red!25}16.6 & \cellcolor{green!25}17.4 & 10.8 & \cellcolor{green!25}7 & 6 & 5 & \cellcolor{green!25}6 & \cellcolor{green!25}4 & \cellcolor{green!25}4 \\ 

\hline
    \end{tabular}
    \label{tab:comparison_returns_ml_metrics}


  
\end{table}
\begin{table}[!ht]
    \centering
    \caption{Comparison of Annual Volatility and Machine Learning Metrics}
    \medskip
    \begin{tabular}{l || rrr || rr || cc||cc||cc}
    \hline
    \multirow{3}{*}{Year} & \multicolumn{5}{c||}{Annual Volatility} & \multicolumn{6}{c}{Precision, NPV, and Prevalence} \\
    \cline{2-12}
    & \multicolumn{3}{c||}{Buy-and-Hold}   
& \multicolumn{2}{c||}{$k=1$  day} 
& \multicolumn{2}{c}{PPV}  & \multicolumn{2}{c}{NPV} & \multicolumn{2}{c}{$\pi^+$}   \\ \cline{2-12}
 & S\&P & G & V  & GV  & MC  & GV & MC  & GV  & MC & GV & MC \\ \hline
      2001 & 21.9 & \cellcolor{red!25}40.7 & \cellcolor{green!25}17.2 & \cellcolor{red!25}33.4 & 16.2 & \cellcolor{green!25}53 & 52 & \cellcolor{green!25}55 & 51 & 49 & \cellcolor{green!25}50 \\ 
        2002 & 26.4 & \cellcolor{red!25}33.1 & \cellcolor{green!25}24.5 & \cellcolor{red!25}29.2 & 19.5 & \cellcolor{green!25}60 & 55 & \cellcolor{green!25}68 & 61 & 47 & \cellcolor{green!25}48 \\ 
        2003 & \cellcolor{green!25}16.5 & \cellcolor{red!25}18.2 & \cellcolor{green!25}16.5 & \cellcolor{red!25}17.2 & 10.9 & 58 & \cellcolor{green!25}61 & \cellcolor{green!25}57 & 46 & 50 & \cellcolor{green!25}57 \\ 
        2004 & \cellcolor{red!25}11.1 & \cellcolor{red!25}11.1 & \cellcolor{green!25}10.5 & \cellcolor{red!25}11.2 & 7.4 & 53 & \cellcolor{green!25}57 & \cellcolor{green!25}59 & 42 & 48 & \cellcolor{green!25}58 \\ 
        2005 & \cellcolor{red!25}10.3 & 10.2 & \cellcolor{green!25}9.8 & \cellcolor{red!25}10.1 & 7.1 & 50 & \cellcolor{green!25}58 & \cellcolor{green!25}55 & 46 & 48 & \cellcolor{green!25}56 \\ 
        2006 & 10 & \cellcolor{red!25}10.7 & \cellcolor{green!25}9.8 & \cellcolor{red!25}10.6 & 7 & 47 & \cellcolor{green!25}53 & \cellcolor{green!25}61 & 41 & 43 & \cellcolor{green!25}56 \\ 
        2007 & 15.9 & \cellcolor{green!25}14.6 & \cellcolor{red!25}16.7 & \cellcolor{red!25}15.9 & 11 & 63 & \cellcolor{green!25}63 & \cellcolor{green!25}51 & 51 & 55 & \cellcolor{green!25}55 \\ 
        2008 & \cellcolor{red!25}41.4 & \cellcolor{green!25}37.4 & 40.9 & \cellcolor{red!25}39.9 & 33.3 & \cellcolor{green!25}52 & 49 & \cellcolor{green!25}53 & 50 & 49 & \cellcolor{green!25}50 \\ 
        2009 & 26.6 & \cellcolor{green!25}25.5 & \cellcolor{red!25}27.8 & \cellcolor{red!25}26.9 & 18.9 & \cellcolor{green!25}63 & 56 & \cellcolor{green!25}46 & 44 & \cellcolor{green!25}58 & 56 \\ 
        2010 & 17.9 & \cellcolor{red!25}18.8 & \cellcolor{green!25}17.8 & \cellcolor{red!25}18.1 & 14.6 & 54 & \cellcolor{green!25}59 & \cellcolor{green!25}54 & 42 & 50 & \cellcolor{green!25}58 \\ 
        2011 & 23 & \cellcolor{green!25}21.5 & \cellcolor{red!25}23.7 & \cellcolor{red!25}22.2 & 16.3 & \cellcolor{green!25}56 & 51 & \cellcolor{green!25}55 & 44 & 50 & \cellcolor{green!25}54 \\ 
        2012 & 12.7 & \cellcolor{green!25}12.0 & \cellcolor{red!25}13.9 & \cellcolor{red!25}13.1 & 9.5 & \cellcolor{green!25}56 & 51 & \cellcolor{green!25}65 & 41 & 46 & \cellcolor{green!25}56 \\ 
        2013 & \cellcolor{red!25}11.1 & 10.9 & \cellcolor{green!25}10.4 & \cellcolor{red!25}10.8 & 7.7 & 59 & \cellcolor{green!25}62 & \cellcolor{green!25}53 & 43 & 53 & \cellcolor{green!25}59 \\ 
        2014 & 11.2 & \cellcolor{red!25}12.2 & \cellcolor{green!25}10.3 & \cellcolor{red!25}11.3 & 8 & 59 & \cellcolor{green!25}64 & \cellcolor{green!25}47 & 44 & 56 & \cellcolor{green!25}59 \\ 
        2015 & 15.4 & \cellcolor{red!25}15.8 & \cellcolor{green!25}15.3 & \cellcolor{red!25}15.7 & 12.6 & \cellcolor{green!25}58 & 50 & 49 & \cellcolor{green!25}55 & \cellcolor{green!25}54 & 48 \\ 
        2016 & \cellcolor{green!25}13.1 & \cellcolor{red!25}13.4 & 13.3 & \cellcolor{red!25}13.8 & 8.8 & 51 & \cellcolor{green!25}65 & 49 & \cellcolor{green!25}54 & 51 & \cellcolor{green!25}55 \\ 
        2017 & \cellcolor{green!25}6.7 & \cellcolor{red!25}7.2 & \cellcolor{red!25}7.2 & \cellcolor{red!25}7.2 & 4.9 & 62 & \cellcolor{green!25}63 & 46 & \cellcolor{green!25}47 & \cellcolor{green!25}58 & 57 \\ 
        2018 & 17 & \cellcolor{red!25}19.5 & \cellcolor{green!25}15.1 & \cellcolor{red!25}17.1 & 14 & \cellcolor{green!25}56 & 51 & 41 & \cellcolor{green!25}44 & \cellcolor{green!25}57 & 54 \\ 
        2019 & \cellcolor{green!25}12.5 & \cellcolor{red!25}12.9 & 12.8 & \cellcolor{red!25}13.0 & 9.1 & 45 & \cellcolor{green!25}56 & \cellcolor{green!25}48 & 38 & 49 & \cellcolor{green!25}60 \\ 
        2020 & \cellcolor{green!25}33.5 & 34.7 & \cellcolor{red!25}35.2 & \cellcolor{red!25}35.3 & 26.3 & 55 & \cellcolor{green!25}67 & 38 & \cellcolor{green!25}49 & \cellcolor{green!25}59 & 58 \\ 
        2021 & \cellcolor{green!25}13.0 & \cellcolor{red!25}16.3 & \cellcolor{green!25}13.0 & \cellcolor{red!25}15.3 & 9.4 & 46 & \cellcolor{green!25}60 & \cellcolor{green!25}47 & 44 & 50 & \cellcolor{green!25}58 \\ 
        2022 & 24.2 & \cellcolor{red!25}30.6 & \cellcolor{green!25}19.2 & \cellcolor{red!25}26.3 & 18 & \cellcolor{green!25}46 & 43 & \cellcolor{green!25}56 & 55 & \cellcolor{green!25}45 & 44 \\ 
        2023 & \cellcolor{green!25}13.0 & \cellcolor{red!25}13.3 & \cellcolor{red!25}13.3 & \cellcolor{red!25}13.5 & 9.3 & \cellcolor{green!25}58 & 58 & 44 & \cellcolor{green!25}45 & \cellcolor{green!25}57 & 56 \\ \hline
        $\max$ & 21.9 & \cellcolor{green!25}10.9 & \cellcolor{red!25}40.9 & \cellcolor{red!25}39.9 & 33.3 & 63 & \cellcolor{green!25}67 & \cellcolor{green!25}68 & 61 & 59 & \cellcolor{green!25}60 \\ 
        $\min$ & \cellcolor{green!25}6.7 & \cellcolor{red!25}7.2 & \cellcolor{red!25}7.2 & \cellcolor{red!25}7.2 & 4.9 & \cellcolor{green!25}45 & 43 & \cellcolor{green!25}38 & \cellcolor{green!25}38 & 43 & \cellcolor{green!25}44 \\ 
        $M$ & \cellcolor{red!25}10.0 & \cellcolor{green!25}9.0 & 9.8 & \cellcolor{red!25}15.7 & 10.9 & 56 & \cellcolor{green!25}57 & \cellcolor{green!25}53 & 45 & 50 & \cellcolor{green!25}56 \\ 
        $\mu$  & 12.9 & \cellcolor{green!25}9.0 & \cellcolor{red!25}17.4 & \cellcolor{red!25}18.6 & 13 & 55 & \cellcolor{green!25}57 & \cellcolor{green!25}52 & 47 & 51 & \cellcolor{green!25}55 \\ 
        $\sigma$ & 8 & \cellcolor{green!25}2.6 & \cellcolor{red!25}13.9 & \cellcolor{red!25}9.0 & 6.8 & 5 & \cellcolor{green!25}6 & \cellcolor{green!25}7 & 6 & \cellcolor{green!25}5 & 4 \\ \hline
    \end{tabular}
    \label{tab:comparison_volatility_ml_metrics}
\end{table}





\newpage
\begin{table}[!ht]
    \centering
    \caption{Annual Maximum Drawdown and Volatility of Investment Strategies}
    \medskip
    \begin{tabular}{l || rrr || rr || rrr || rr }
    \hline
\multirow{3}{*}{Year} & \multicolumn{5}{c||}{Maximum Drawdowns}& \multicolumn{5}{c}{Annual Volatility}     \\\cline{2-11}
 & \multicolumn{3}{c||}{Buy-and-Hold}   
& \multicolumn{2}{c||}{$k=1$  day} &   \multicolumn{3}{c||}{Buy-and-Hold}   & \multicolumn{2}{c}{$k=1$  day}   \\  \cline{2-11}
  & S\&P & G & V   & GV & MC & S\&P & G & V  & GV &  MC \\ \hline
        2001 & -28.8 & \cellcolor{red!25}-48.6 & \cellcolor{green!25}-18.1 & \cellcolor{red!25}-32.0 & -19.7 & 21.9 & \cellcolor{red!25}40.7 & \cellcolor{green!25}17.2 & \cellcolor{red!25}33.4 & 16.2 \\ 
        2002 & -33.0 & \cellcolor{red!25}-40.1 & \cellcolor{green!25}-31.2 & \cellcolor{red!25}-26.4 & -18.0 & 26.4 & \cellcolor{red!25}33.1 & \cellcolor{green!25}24.5 & \cellcolor{red!25}29.2 & 19.5 \\ 
        2003 & -13.7 & \cellcolor{green!25}-12.8 & \cellcolor{red!25}-15.3 & \cellcolor{red!25}-13.9 & -6.5 & \cellcolor{green!25}16.5 & \cellcolor{red!25}18.2 & \cellcolor{green!25}16.5 & \cellcolor{red!25}17.2 & 10.9 \\ 
        2004 & \cellcolor{green!25}-7.5 & \cellcolor{red!25}-10.8 & \cellcolor{green!25}-7.5 & -6.5 & \cellcolor{red!25}-6.6 & \cellcolor{red!25}11.1 & \cellcolor{red!25}11.1 & \cellcolor{green!25}10.5 & \cellcolor{red!25}11.2 & 7.4 \\ 
        2005 & -7.0 & \cellcolor{red!25}-8.0 & \cellcolor{green!25}-6.2 & \cellcolor{red!25}-5.6 & -4.9 & \cellcolor{red!25}10.3 & 10.2 & \cellcolor{green!25}9.8 & \cellcolor{red!25}10.1 & 7.1 \\ 
        2006 & -7.6 & \cellcolor{red!25}-9.0 & \cellcolor{green!25}-7.4 & \cellcolor{red!25}-7.7 & -6.8 & 10.0 & \cellcolor{red!25}10.7 & \cellcolor{green!25}9.8 & \cellcolor{red!25}10.6 & 7.0 \\ 
        2007 & -9.9 & \cellcolor{green!25}-9.0 & \cellcolor{red!25}-12.0 & \cellcolor{red!25}-10.1 & -5.7 & 15.9 & \cellcolor{green!25}14.6 & \cellcolor{red!25}16.7 & \cellcolor{red!25}15.9 & 11.0 \\ 
        2008 & -47.1 & \cellcolor{red!25}-48.0 & \cellcolor{green!25}-47.0 & \cellcolor{red!25}-41.8 & -20.9 & \cellcolor{red!25}41.4 & \cellcolor{green!25}37.4 & 40.9 & \cellcolor{red!25}39.9 & 33.3 \\ 
        2009 & -27.1 & \cellcolor{green!25}-23.0 & \cellcolor{red!25}-30.6 & \cellcolor{red!25}-28.4 & -14.7 & 26.6 & \cellcolor{green!25}25.5 & \cellcolor{red!25}27.8 & \cellcolor{red!25}26.9 & 18.9 \\ 
        2010 & -15.7 & \cellcolor{red!25}-16.3 & \cellcolor{green!25}-14.5 & \cellcolor{red!25}-14.3 & -8.0 & 17.9 & \cellcolor{red!25}18.8 & \cellcolor{green!25}17.8 & \cellcolor{red!25}18.1 & 14.6 \\ 
        2011 & -18.6 & \cellcolor{green!25}-16.5 & \cellcolor{red!25}-21.9 & \cellcolor{red!25}-16.8 & -16.1 & 23.0 & \cellcolor{green!25}21.5 & \cellcolor{red!25}23.7 & \cellcolor{red!25}22.2 & 16.3 \\ 
        2012 & -9.7 & \cellcolor{green!25}-8.3 & \cellcolor{red!25}-11.2 & -6.3 & \cellcolor{red!25}-9.2 & 12.7 & \cellcolor{green!25}12.0 & \cellcolor{red!25}13.9 & \cellcolor{red!25}13.1 & 9.5 \\ 
        2013 & -5.6 & \cellcolor{red!25}-5.8 & \cellcolor{green!25}-5.1 & -4.3 & \cellcolor{red!25}-4.7 & \cellcolor{red!25}11.1 & 10.9 & \cellcolor{green!25}10.4 & \cellcolor{red!25}10.8 & 7.7 \\ 
        2014 & \cellcolor{green!25}-7.3 & \cellcolor{red!25}-7.4 & \cellcolor{red!25}-7.4 & \cellcolor{red!25}-7.4 & -4.2 & 11.2 & \cellcolor{red!25}12.2 & \cellcolor{green!25}10.3 & \cellcolor{red!25}11.3 & 8.0 \\ 
        2015 & -11.9 & \cellcolor{green!25}-11.8 & \cellcolor{red!25}-13.6 & \cellcolor{red!25}-13.2 & -11.7 & 15.4 & \cellcolor{red!25}15.8 & \cellcolor{green!25}15.3 & \cellcolor{red!25}15.7 & 12.6 \\ 
        2016 & -9.2 & \cellcolor{red!25}-9.5 & \cellcolor{green!25}-8.6 & \cellcolor{red!25}-10.8 & -3.8 & \cellcolor{green!25}13.1 & \cellcolor{red!25}13.4 & 13.3 & \cellcolor{red!25}13.8 & 8.8 \\ 
        2017 & -2.6 & \cellcolor{green!25}-2.4 & \cellcolor{red!25}-4.4 & \cellcolor{red!25}-3.3 & -1.8 & \cellcolor{green!25}6.7 & \cellcolor{red!25}7.2 & \cellcolor{red!25}7.2 & \cellcolor{red!25}7.2 & 4.9 \\ 
        2018 & -19.3 & \cellcolor{red!25}-20.6 & \cellcolor{green!25}-19.2 & -19.5 & \cellcolor{red!25}-19.9 & 17.0 & \cellcolor{red!25}19.5 & \cellcolor{green!25}15.1 & \cellcolor{red!25}17.1 & 14.0 \\ 
        2019 & -6.6 & \cellcolor{green!25}-6.4 & \cellcolor{red!25}-7.7 & \cellcolor{red!25}-7.3 & -6.6 & \cellcolor{green!25}12.5 & \cellcolor{red!25}12.9 & 12.8 & \cellcolor{red!25}13.0 & 9.1 \\ 
        2020 & -33.7 & \cellcolor{green!25}-31.3 & \cellcolor{red!25}-36.9 & \cellcolor{red!25}-36.0 & -17.6 & \cellcolor{green!25}33.5 & 34.7 & \cellcolor{red!25}35.2 & \cellcolor{red!25}35.3 & 26.3 \\ 
        2021 & \cellcolor{green!25}-5.1 & \cellcolor{red!25}-8.7 & -5.8 & \cellcolor{red!25}-7.9 & -4.3 & \cellcolor{green!25}13.0 & \cellcolor{red!25}16.3 & \cellcolor{green!25}13.0 & \cellcolor{red!25}15.3 & 9.4 \\ 
        2022 & -24.5 & \cellcolor{red!25}-32.3 & \cellcolor{green!25}-17.9 & \cellcolor{red!25}-19.5 & -17.0 & 24.2 & \cellcolor{red!25}30.6 & \cellcolor{green!25}19.2 & \cellcolor{red!25}26.3 & 18.0 \\ 
        2023 & -10.0 & \cellcolor{green!25}-9.1 & \cellcolor{red!25}-10.9 & \cellcolor{red!25}-9.8 & -7.0 & \cellcolor{green!25}13.0 & \cellcolor{red!25}13.3 & \cellcolor{red!25}13.3 & \cellcolor{red!25}13.5 & 9.3 \\  \hline\hline
        $\max$ & -2.6 & \cellcolor{green!25}-2.4 & \cellcolor{red!25}-4.4 & \cellcolor{red!25}-3.3 & -1.8 & 21.9 & \cellcolor{green!25}10.9 & \cellcolor{red!25}40.9 & \cellcolor{red!25}39.9 & 33.3 \\ 
        $\min$ & -47.1 & \cellcolor{red!25}-48.6 & \cellcolor{green!25}-47.0 & \cellcolor{red!25}-41.8 & -20.9 & \cellcolor{green!25}6.7 & \cellcolor{red!25}7.2 & \cellcolor{red!25}7.2 & \cellcolor{red!25}7.2 & 4.9 \\ 
        $M$ & \cellcolor{green!25}-10.0 & -10.8 & \cellcolor{red!25}-12.0 & \cellcolor{red!25}-10.8 & -7.0 & \cellcolor{red!25}10.0 & \cellcolor{green!25}9.0 & 9.8 & \cellcolor{red!25}15.7 & 10.9 \\ 
        $\mu$ & \cellcolor{green!25}-15.7 & \cellcolor{red!25}-17.2 & \cellcolor{green!25}-15.7 & \cellcolor{red!25}-15.2 & -10.2 & 12.9 & \cellcolor{green!25}9.0 & \cellcolor{red!25}17.4 & \cellcolor{red!25}18.6 & 13.0 \\ 
        $\sigma$ & 11.5 & \cellcolor{green!25}13.6 & \cellcolor{red!25}11.2 & 10.8 & \cellcolor{red!25}6.2 & 8.0 & \cellcolor{green!25}2.6 & \cellcolor{red!25}13.9 & \cellcolor{red!25}9.0 & 6.8 \\ 

\hline
    \end{tabular}
    \label{tab:comparison_drawdown}
\end{table}








\newpage
\begin{table}[!ht]
    \centering
    \caption{Comparison of Tracking Errors and Sharp Ratios}
    \medskip

\begin{tabular}{ c | r || rr || rr || rr}
\hline
\multirow{2}{*}{Year} & \multicolumn{1}{c||}{Return}   & \multicolumn{4}{c||}{Tracking Error} 
& \multicolumn{2}{c}{Sharp Ratio}   \\ \cline{2-8}
  & S\&P & G & V  & GV  & MC  & GV  & MC  \\ 
\hline
              2001 & -11.8 & -14.1 & \cellcolor{green!25}6.2 & \cellcolor{green!25}28.8 & 6.7 & \cellcolor{green!25}0.5 & -0.3 \\ 
        2002 & -21.6 & -10.5 & \cellcolor{green!25}3.6 & \cellcolor{green!25}55.7 & 30.5 & \cellcolor{green!25}1.2 & 0.5 \\ 
        2003 & 28.2 & \cellcolor{green!25}0.2 & -3 & \cellcolor{green!25}23.3 & -10.4 & \cellcolor{green!25}3.0 & 1.6 \\ 
        2004 & 10.7 & -5.4 & \cellcolor{green!25}2.5 & \cellcolor{green!25}7.7 & -6.2 & \cellcolor{green!25}1.6 & 0.6 \\ 
        2005 & 4.8 & -2.1 & \cellcolor{green!25}0.5 & 4.8 & \cellcolor{green!25}6.0 & 1 & \cellcolor{green!25}1.5 \\ 
        2006 & 15.8 & -6.8 & \cellcolor{green!25}5.7 & \cellcolor{green!25}3.1 & -11.6 & \cellcolor{green!25}1.8 & 0.6 \\ 
        2007 & 5.1 & \cellcolor{green!25}5.6 & -3.7 & 7.4 & \cellcolor{green!25}14.2 & 0.8 & \cellcolor{green!25}1.8 \\ 
        2008 & -36.8 & -0.6 & \cellcolor{green!25}0.5 & 10.3 & \cellcolor{green!25}47.3 & -0.7 & \cellcolor{green!25}0.3 \\ 
        2009 & 26.4 & \cellcolor{green!25}10.6 & -9.3 & \cellcolor{green!25}8.1 & -6.1 & \cellcolor{green!25}1.3 & 1.1 \\ 
        2010 & 15.1 & \cellcolor{green!25}1.2 & 0.4 & \cellcolor{green!25}7.7 & -1.6 & \cellcolor{green!25}1.3 & 0.9 \\ 
        2011 & 1.9 & \cellcolor{green!25}2.8 & -2.6 & \cellcolor{green!25}7.4 & -5.3 & \cellcolor{green!25}0.4 & -0.2 \\ 
        2012 & 16 & -1.8 & \cellcolor{green!25}1.2 & \cellcolor{green!25}20.3 & -15.3 & \cellcolor{green!25}2.8 & 0.1 \\ 
        2013 & 32.3 & \cellcolor{green!25}0.3 & -0.5 & \cellcolor{green!25}7.9 & -16.3 & \cellcolor{green!25}3.7 & 2.1 \\ 
        2014 & 13.5 & \cellcolor{green!25}1.3 & -1.3 & \cellcolor{green!25}3.4 & -2.5 & \cellcolor{green!25}1.5 & 1.4 \\ 
        2015 & 1.2 & \cellcolor{green!25}3.8 & -4.4 & \cellcolor{green!25}3.3 & -1 & \cellcolor{green!25}0.3 & 0 \\ 
        2016 & 12 & -5.2 & \cellcolor{green!25}5.1 & -0.9 & \cellcolor{green!25}5.9 & 0.8 & \cellcolor{green!25}2.0 \\ 
        2017 & 21.7 & \cellcolor{green!25}5.5 & -6.3 & \cellcolor{green!25}-0.5 & -8.3 & \cellcolor{green!25}3.0 & 2.7 \\ 
        2018 & -4.6 & \cellcolor{green!25}4.5 & -4.4 & \cellcolor{green!25}-2.2 & -8.1 & \cellcolor{green!25}-0.4 & -0.9 \\ 
        2019 & 31.2 & -0.4 & \cellcolor{green!25}0.5 & \cellcolor{green!25}-5.5 & -22 & \cellcolor{green!25}2.0 & 1 \\ 
        2020 & 18.3 & \cellcolor{green!25}15.1 & -17 & -12.3 & \cellcolor{green!25}17.8 & 0.2 & \cellcolor{green!25}1.4 \\ 
        2021 & 28.7 & \cellcolor{green!25}3.3 & -3.8 & \cellcolor{green!25}-12.4 & -14.6 & 1.1 & \cellcolor{green!25}1.5 \\ 
        2022 & -18.2 & -11.2 & \cellcolor{green!25}12.9 & 5.1 & \cellcolor{green!25}11.5 & -0.5 & \cellcolor{green!25}-0.4 \\ 
        2023 & 26.2 & \cellcolor{green!25}3.8 & -4 & \cellcolor{green!25}-3.9 & -9.7 & 1.6 & \cellcolor{green!25}1.8 \\ \hline
        $\max$ & 32.3 & \cellcolor{green!25}15.1 & 12.9 & \cellcolor{green!25}55.7 & 47.3 & \cellcolor{green!25}3.7 & 2.7 \\ 
        $\min$ & -36.8 & \cellcolor{green!25}-14.1 & -17 & \cellcolor{green!25}-12.4 & -22 & \cellcolor{green!25}-0.7 & -0.9 \\ 
        $M$ & 13.5 & \cellcolor{green!25}0.3 & -0.5 & \cellcolor{green!25}5.1 & -5.3 & \cellcolor{green!25}1.2 & 1 \\ 
        $\mu$ & 9.4 & \cellcolor{green!25}0.0 & -0.9 & \cellcolor{green!25}7.2 & 0 & \cellcolor{green!25}1.2 & 0.9 \\ 
        $\sigma$ & 18.3 & \cellcolor{green!25}6.8 & 6 & 14.4 & \cellcolor{green!25}16.1 & \cellcolor{green!25}1.1 & 0.9 \\ 
\hline
\end{tabular}
\label{tab:comparison_tracking_errors_sharp_ratios}
\end{table}





\newpage
\begin{table}[!ht]
    \centering
    \caption{Return Efficiency Ratios}
    \begin{tabular}{l| ccc||cc||cc}
    \hline
    \multirow{2}{*}{Year} & \multicolumn{3}{c||}{Buy-and-Hold}   
& \multicolumn{2}{c||}{Growth-Value}  &  \multicolumn{2}{c}{Market-Cash}  \\\cline{2-8}
         & S\&P & G  & V & GV-1L& Random  & MC-1L& Random \\ \hline
 2001 & 0.13 & 0.11 & 0.14 & \cellcolor{green!25}0.18 & 0.13 & \cellcolor{green!25}0.14 & 0.14 \\ 
        2002 & 0.20 & 0.17 & 0.21 & \cellcolor{green!25}0.38 & 0.19 & \cellcolor{green!25}0.30 & 0.23 \\ 
        2003 & 0.33 & 0.33 & 0.32 & \cellcolor{green!25}0.45 & 0.32 & \cellcolor{green!25}0.28 & 0.26 \\ 
        2004 & 0.40 & 0.35 & 0.42 & \cellcolor{green!25}0.47 & 0.38 & 0.34 & \cellcolor{green!25}0.35 \\ 
        2005 & 0.41 & 0.39 & 0.42 & \cellcolor{green!25}0.47 & 0.4 & \cellcolor{green!25}0.48 & 0.38 \\ 
        2006 & 0.43 & 0.34 & 0.51 & \cellcolor{green!25}0.48 & 0.42 & 0.27 & \cellcolor{green!25}0.32 \\ 
        2007 & 0.39 & 0.46 & 0.35 & \cellcolor{green!25}0.48 & 0.4 & \cellcolor{green!25}0.56 & 0.36 \\ 
        2008 & 0.29 & 0.28 & 0.29 & \cellcolor{green!25}0.37 & 0.29 & \cellcolor{green!25}0.65 & 0.43 \\ 
        2009 & 0.34 & 0.41 & 0.29 & \cellcolor{green!25}0.39 & 0.35 & \cellcolor{green!25}0.31 & 0.27 \\ 
        2010 & 0.39 & 0.40 & 0.39 & \cellcolor{green!25}0.46 & 0.4 & \cellcolor{green!25}0.37 & 0.31 \\ 
        2011 & 0.38 & 0.41 & 0.35 & \cellcolor{green!25}0.45 & 0.38 & 0.33 & \cellcolor{green!25}0.37 \\ 
        2012 & 0.39 & 0.37 & 0.40 & \cellcolor{green!25}0.57 & 0.39 & 0.25 & \cellcolor{green!25}0.32 \\ 
        2013 & 0.43 & 0.43 & 0.42 & \cellcolor{green!25}0.53 & 0.43 & 0.22 & \cellcolor{green!25}0.22 \\ 
        2014 & 0.42 & 0.44 & 0.40 & \cellcolor{green!25}0.46 & 0.42 & \cellcolor{green!25}0.39 & 0.33 \\ 
        2015 & 0.41 & 0.46 & 0.35 & \cellcolor{green!25}0.45 & 0.41 & 0.4 & \cellcolor{green!25}0.40 \\ 
        2016 & 0.40 & 0.35 & 0.45 & 0.39 & \cellcolor{green!25}0.40 & \cellcolor{green!25}0.46 & 0.34 \\ 
        2017 & 0.42 & 0.49 & 0.35 & 0.41 & \cellcolor{green!25}0.42 & \cellcolor{green!25}0.32 & 0.29 \\ 
        2018 & 0.36 & 0.40 & 0.32 & 0.35 & \cellcolor{green!25}0.36 & 0.29 & \cellcolor{green!25}0.38 \\ 
        2019 & 0.42 & 0.41 & 0.42 & 0.35 & \cellcolor{green!25}0.42 & 0.17 & \cellcolor{green!25}0.24 \\ 
        2020 & 0.28 & 0.34 & 0.22 & 0.23 & \cellcolor{green!25}0.28 & \cellcolor{green!25}0.34 & 0.24 \\ 
        2021 & 0.29 & 0.30 & 0.28 & 0.24 & \cellcolor{green!25}0.29 & 0.23 & \cellcolor{green!25}0.24 \\ 
        2022 & 0.27 & 0.21 & 0.33 & \cellcolor{green!25}0.29 & 0.27 & \cellcolor{green!25}0.32 & 0.31 \\ 
        2023 & 0.41 & 0.44 & 0.37 & 0.37 & \cellcolor{green!25}0.40 & \cellcolor{green!25}0.31 & 0.28 \\ \hline
        $\max$ & 0.43 & 0.49 & 0.51 & \cellcolor{green!25}0.57 & 0.43 & \cellcolor{green!25}0.65 & 0.43 \\ 
        $\min$ & 0.13 & 0.11 & 0.14 & \cellcolor{green!25}0.18 & 0.13 & \cellcolor{green!25}0.14 & 0.14 \\ 
        $M$ & 0.39 & 0.39 & 0.35 & \cellcolor{green!25}0.41 & 0.39 & \cellcolor{green!25}0.32 & 0.31 \\ 
        $\mu$  & 0.36 & 0.36 & 0.35 & \cellcolor{green!25}0.40 & 0.35 & \cellcolor{green!25}0.34 & 0.31 \\ 
        $\sigma$ & 0.08 & 0.10 & 0.08 & \cellcolor{green!25}0.10 & 0.08 & \cellcolor{green!25}0.12 & 0.07 \\    \hline
    \end{tabular}
    \label{tab:return_effiency_ratios}
\end{table}






\newpage

\begin{table}[!ht]
    \centering
    \caption{Annual Metrics by Investment Type and Market Condition}
    \begin{tabular}{l| cc | cc || rrrr|rr}
    \hline
    \multirow{3}{*}{Year} & \multicolumn{4}{c||}{True Labels}   & \multicolumn{6}{c}{Average Return for True Labels}  \\ \cline{2-11}
     & \multicolumn{2}{c|}{GV}  &  \multicolumn{2}{c||}{MC}  & \multicolumn{4}{c|}{GV}  &  \multicolumn{2}{c}{MC}  \\ \cline{2-11}
    & $T^{+}$ & $T^{-}$ & $T^{+}$ & $T^{-}$ & $r^{+}_{G}$ & $r^{-}_{G}$ & $r^{+}_{V}$ & $r^{-}_{V}$ & $r^{+}$ & $r^{-}$ \\ \cline{1-11}
        2001 & 121 & 127 & 125 & 123 & 1.54 & -1.64 & 0.00 & -0.04 & 1.01 & -1.11 \\ 
        2002 & 118 & 134 & 120 & 132 & 1.11 & -1.23 & -0.06 & -0.07 & 1.26 & -1.3 \\ 
        2003 & 127 & 125 & 144 & 108 & 0.5 & -0.30 & -0.09 & 0.28 & 0.8 & -0.82 \\ 
        2004 & 120 & 132 & 145 & 107 & 0.26 & -0.19 & -0.11 & 0.19 & 0.52 & -0.6 \\ 
        2005 & 121 & 131 & 140 & 112 & 0.24 & -0.20 & -0.07 & 0.11 & 0.48 & -0.56 \\ 
        2006 & 109 & 142 & 141 & 110 & 0.29 & -0.16 & 0.06 & 0.1 & 0.47 & -0.47 \\ 
        2007 & 138 & 113 & 139 & 112 & 0 & 0.10 & -0.32 & 0.41 & 0.67 & -0.78 \\ 
        2008 & 125 & 128 & 126 & 127 & -0.31 & -0.01 & -1.02 & 0.71 & 1.56 & -1.84 \\ 
        2009 & 145 & 107 & 141 & 111 & 0.19 & 0.06 & -0.30 & 0.58 & 1.17 & -1.24 \\ 
        2010 & 125 & 127 & 147 & 105 & 0.35 & -0.22 & 0.01 & 0.12 & 0.73 & -0.88 \\ 
        2011 & 127 & 125 & 136 & 116 & -0.07 & 0.12 & -0.47 & 0.5 & 0.97 & -1.1 \\ 
        2012 & 116 & 134 & 139 & 111 & 0.05 & 0.06 & -0.34 & 0.42 & 0.59 & -0.6 \\ 
        2013 & 133 & 119 & 149 & 103 & 0.27 & -0.06 & 0.04 & 0.19 & 0.55 & -0.52 \\ 
        2014 & 140 & 112 & 149 & 103 & 0.35 & -0.31 & 0.11 & -0.03 & 0.49 & -0.58 \\ 
        2015 & 137 & 115 & 121 & 131 & 0.18 & -0.16 & -0.14 & 0.14 & 0.75 & -0.68 \\ 
        2016 & 129 & 123 & 138 & 114 & 0.21 & -0.16 & -0.08 & 0.22 & 0.57 & -0.59 \\ 
        2017 & 145 & 106 & 144 & 107 & 0.17 & -0.01 & -0.10 & 0.27 & 0.33 & -0.27 \\ 
        2018 & 144 & 107 & 135 & 116 & 0.51 & -0.67 & 0.09 & -0.2 & 0.68 & -0.81 \\ 
        2019 & 123 & 129 & 150 & 102 & 0.23 & 0.00 & -0.05 & 0.27 & 0.57 & -0.56 \\ 
        2020 & 150 & 103 & 146 & 107 & 0.4 & -0.25 & -0.35 & 0.58 & 1.23 & -1.46 \\ 
        2021 & 125 & 127 & 146 & 106 & 0.63 & -0.39 & -0.12 & 0.3 & 0.63 & -0.62 \\ 
        2022 & 113 & 138 & 110 & 141 & 1.27 & -1.25 & 0.48 & -0.42 & 1.29 & -1.13 \\ 
        2023 & 142 & 108 & 141 & 109 & 0.24 & -0.06 & -0.06 & 0.27 & 0.66 & -0.63 \\  \hline
        $\max$ & 150 & 142 & 150 & 141 & 1.54 & 0.12 & 0.48 & 0.71 & 1.56 & -0.27 \\ 
        $\min$ & 109 & 103 & 110 & 102 & -0.31 & -1.64 & -1.02 & -0.42 & 0.33 & -1.84 \\ 
        $M$ & 127 & 125 & 141 & 111 & 0.26 & -0.16 & -0.08 & 0.22 & 0.67 & -0.68 \\ 
        $\mu$ & 129 & 122 & 138 & 114 & 0.38 & -0.30 & -0.12 & 0.21 & 0.78 & -0.83 \\ 
        $\sigma$ & 11.43 & 11.29 & 10.6 & 10.4 & 0.42 & 0.47 & 0.28 & 0.26 & 0.32 & 0.37 \\ 
        \hline
    \end{tabular}
    \label{tab:comparison_ml_metrics}
\end{table}


\newpage
\begin{table}[!ht]
    \centering
    \caption{Number of Transactions for GV-$k$L And MC-$k$L Strategy}
\begin{tabular}{r| ccc|| cccc||cccc}
\hline
\multirow{2}{*}{Year} & \multicolumn{3}{c||}{Buy-and-Hold}   & \multicolumn{4}{c||}{GV-$k$L Strategy}& \multicolumn{4}{c}{MC-$k$L Strategy}\\ \cline{2-12}
  & S\&P & G & V  & $k=1$  & $3$  & $5$  & $7$  & $k=1$  & $3$  & $5$  & $7$ \\ 
\hline
%Year & S\&P & Growth & Value & 1 day & 3 day & 5 day & 7 day \\
%\hline
        2001 & 1 & 1 & 1 & 135 & 63 & 50 & 46 & 131 & 60 & 50 & 38 \\ 
        2002 & 0 & 0 & 0 & 160 & 92 & 72 & 56 & 145 & 79 & 65 & 47 \\ 
        2003 & 0 & 0 & 0 & 145 & 68 & 51 & 51 & 130 & 66 & 54 & 52 \\ 
        2004 & 0 & 0 & 0 & 140 & 75 & 49 & 37 & 125 & 64 & 36 & 32 \\ 
        2005 & 0 & 0 & 0 & 132 & 63 & 45 & 31 & 134 & 78 & 53 & 40 \\ 
        2006 & 0 & 0 & 0 & 132 & 74 & 44 & 42 & 116 & 61 & 46 & 35 \\ 
        2007 & 0 & 0 & 0 & 142 & 73 & 57 & 49 & 144 & 68 & 55 & 55 \\ 
        2008 & 0 & 0 & 0 & 133 & 83 & 67 & 51 & 125 & 61 & 47 & 39 \\ 
        2009 & 0 & 0 & 0 & 133 & 71 & 53 & 39 & 125 & 57 & 36 & 38 \\ 
        2010 & 0 & 0 & 0 & 135 & 72 & 54 & 37 & 124 & 57 & 33 & 26 \\ 
        2011 & 0 & 0 & 0 & 138 & 77 & 55 & 46 & 118 & 48 & 42 & 34 \\ 
        2012 & 0 & 0 & 0 & 150 & 83 & 67 & 53 & 112 & 56 & 39 & 39 \\ 
        2013 & 0 & 0 & 0 & 139 & 71 & 56 & 42 & 129 & 62 & 49 & 43 \\ 
        2014 & 0 & 0 & 0 & 131 & 69 & 41 & 35 & 131 & 61 & 44 & 28 \\ 
        2015 & 0 & 0 & 0 & 132 & 62 & 46 & 52 & 132 & 74 & 57 & 44 \\ 
        2016 & 0 & 0 & 0 & 125 & 64 & 48 & 33 & 146 & 86 & 70 & 58 \\ 
        2017 & 0 & 0 & 0 & 132 & 66 & 43 & 32 & 133 & 67 & 51 & 46 \\ 
        2018 & 0 & 0 & 0 & 118 & 57 & 41 & 30 & 118 & 52 & 42 & 34 \\ 
        2019 & 0 & 0 & 0 & 117 & 55 & 38 & 40 & 113 & 57 & 40 & 27 \\ 
        2020 & 0 & 0 & 0 & 114 & 61 & 47 & 40 & 144 & 71 & 44 & 42 \\ 
        2021 & 0 & 0 & 0 & 118 & 62 & 41 & 24 & 127 & 61 & 48 & 42 \\ 
        2022 & 0 & 0 & 0 & 126 & 72 & 44 & 34 & 123 & 54 & 44 & 36 \\ 
        2023 & 0 & 0 & 0 & 125 & 60 & 56 & 42 & 125 & 57 & 37 & 33 \\ \hline\hline
        $\max$ & 1 & 1 & 1 & 160 & 92 & 72 & 56 & 146 & 86 & 70 & 58 \\ 
        $\min$ & 1 & 1 & 1 & 114 & 55 & 38 & 24 & 112 & 48 & 33 & 26 \\ 
        $M$ & 1 & 1 & 1 & 132 & 69 & 49 & 40 & 127 & 61 & 46 & 39 \\ 
        $\mu$ & 1 & 1 & 1 & 133 & 69 & 51 & 41 & 128 & 63 & 47 & 39 \\ 
        $\sigma$ & 0 & 0 & 0 & 11 & 9 & 9 & 8 & 10 & 9 & 9 & 8 \\ 
\hline
\end{tabular}
\label{tab:transaction_costs}
\end{table}

\newpage
%\subsection{Growth Comparison for Different $k$}
\begin{figure}[h!]
  \centering
  \includegraphics[width=0.8\linewidth]{plots/combined_image.png} % width0.8\linewidth
  \caption{Growth, Maximum Drawdown And Volatility For Different $k$}
  \label{fig:Growth, Maximum Drawdown And Volatility For Different $k$}
  
\end{figure}
\end{document}


????

% --------------------------------------------
% --------     Commented out Tables and Graphs
% ---------------------------------------------



\begin{comment}

\newpage
\begin{table}[!ht]
    \centering
    \caption{Annual Return For Strategy}
    \begin{tabular}{l| rl | rrr | rr}
    \hline
            \multirow{2}{*}{Year} & \multicolumn{2}{c|}{Ideal} & \multicolumn{3}{c|}{Buy-and-Hold} & \multicolumn{2}{c}{Loser Strategies} \\ \cline{2-8}
         & worst & best & S\&P-500 & Growth  & Value & GV-1L & MC-1L \\ \hline
        2001 & -0.88 & 4.88 & -0.12 & -0.26 & -0.06 & 0.17 & -0.05 \\ 
        2002 & -0.83 & 2.24 & -0.22 & -0.32 & -0.18 & 0.34 & 0.09 \\ 
        2003 & -0.39 & 1.64 & 0.28 & 0.28 & 0.25 & 0.51 & 0.18 \\ 
        2004 & -0.32 & 0.76 & 0.11 & 0.05 & 0.13 & 0.18 & 0.05 \\ 
        2005 & -0.30 & 0.54 & 0.05 & 0.03 & 0.05 & 0.10 & 0.11 \\ 
        2006 & -0.16 & 0.57 & 0.16 & 0.09 & 0.22 & 0.19 & 0.04 \\ 
        2007 & -0.29 & 0.58 & 0.05 & 0.11 & 0.01 & 0.13 & 0.19 \\ 
        2008 & -0.74 & 0.56 & -0.37 & -0.37 & -0.36 & -0.27 & 0.11 \\ 
        2009 & -0.33 & 1.39 & 0.26 & 0.37 & 0.17 & 0.34 & 0.20 \\ 
        2010 & -0.24 & 0.77 & 0.15 & 0.16 & 0.15 & 0.23 & 0.13 \\ 
        2011 & -0.37 & 0.66 & 0.02 & 0.05 & -0.01 & 0.09 & -0.03 \\ 
        2012 & -0.27 & 0.84 & 0.16 & 0.14 & 0.17 & 0.36 & 0.01 \\ 
        2013 & -0.02 & 0.78 & 0.32 & 0.33 & 0.32 & 0.40 & 0.16 \\ 
        2014 & -0.18 & 0.57 & 0.13 & 0.15 & 0.12 & 0.17 & 0.11 \\ 
        2015 & -0.31 & 0.49 & 0.01 & 0.05 & -0.03 & 0.05 & 0.00 \\ 
        2016 & -0.26 & 0.69 & 0.12 & 0.07 & 0.17 & 0.11 & 0.18 \\ 
        2017 & -0.14 & 0.71 & 0.22 & 0.27 & 0.15 & 0.21 & 0.13 \\ 
        2018 & -0.46 & 0.67 & -0.05 & 0.00 & -0.09 & -0.07 & -0.13 \\ 
        2019 & -0.07 & 0.84 & 0.31 & 0.31 & 0.32 & 0.26 & 0.09 \\ 
        2020 & -0.57 & 2.12 & 0.18 & 0.33 & 0.01 & 0.06 & 0.36 \\ 
        2021 & -0.48 & 2.15 & 0.29 & 0.32 & 0.25 & 0.16 & 0.14 \\ 
        2022 & -0.71 & 1.27 & -0.18 & -0.29 & -0.05 & -0.13 & -0.07 \\ 
        2023 & -0.15 & 0.86 & 0.26 & 0.30 & 0.22 & 0.22 & 0.16 \\ \hline
        $\max$ & -0.02 & 4.88 & 0.32 & 0.37 & 0.32 & 0.51 & 0.36 \\ 
        $\min$ & -0.88 & 0.49 & -0.37 & -0.37 & -0.36 & -0.27 & -0.13 \\ 
        $M$ & -0.31 & 0.77 & 0.13 & 0.11 & 0.13 & 0.17 & 0.11 \\ 
        $\mu$  & -0.37 & 1.16 & 0.09 & 0.09 & 0.08 & 0.17 & 0.09 \\ 
        $\sigma$ & 0.24 & 0.98 & 0.18 & 0.22 & 0.17 & 0.17 & 0.11 \\ \hline
    \end{tabular}
    \label{tab:annual_returns_for_strategies}
\end{table}

\end{comment}



\begin{comment}
Schematically, the ideal  trading strategy can be represented as follows
\begin{equation*}
    \hbox{Ideal:}  \quad \overbrace{\underbrace{\underbrace{ +,+,\cdots ,+,+}_{TP}\; , 
    \;\underbrace{+,+,\cdots, + }_{FN}}}_{\hbox{invested in } A}^{\hbox{true "+" labels }T^{+}}
    \;\; , \;\; 
\overbrace{\underbrace{\underbrace{-,-,\cdots  ,-,-}_{TN}\; , \; 
\underbrace{ -,-,\cdots ,-}_{FP}}}_{\hbox{invested in }B}^{\hbox{true }"-"
\hbox{ labels } T^{-}} 
\end{equation*}
and
\begin{equation*}
    \hbox{Strategy:}  \quad \overbrace{\underbrace{\underbrace{ +,+,\cdots ,+,+}_{TP}}_{\hbox{invested in } A}\; , 
    \;\underbrace{\underbrace{+,+,\cdots, + }_{FN}}_{\hbox{invested in } B}}^{\hbox{true "+" labels }T^{+}}
    \;\; , \;\; 
\overbrace{\underbrace{\underbrace{-,-,\cdots  ,-,-}_{TN}}_{\hbox{invested in } B}\; , \; 
\underbrace{\underbrace{ -,-,\cdots ,-}_{FP}}_{\hbox{invested in } A}}^{\hbox{true }"-"
\hbox{ labels } T^{-}} 
\end{equation*}
\end{comment}

\begin{comment}
%xiang revision
\section*{Revision}
It can be written as:
\begin{equation}
S_{str} = \sqrt{\left( P + (1 - \text{TNR}) \cdot (1 - P) \right) S_{A}^{2} + \left( (1 - P) + (1 - \text{TPR}) \cdot P \right) S_{B}^{2}}
\end{equation}


\section*{Definitions}

The definitions of the terms used in the formulas are as follows:

- \(\text{TPR} (\text{True Positive Rate, Sensitivity, Recall})\):
  \[
  \text{TPR} = \frac{\text{TP}}{\text{TP} + \text{FN}}
  \]

- \(\text{TNR} (\text{True Negative Rate, Specificity})\):
  \[
  \text{TNR} = \frac{\text{TN}}{\text{TN} + \text{FP}}
  \]

- \(\text{PPV} (\text{Positive Predictive Value})\):
  \[
  \text{PPV} = \frac{\text{TP}}{\text{TP} + \text{FP}}
  \]

- \(\text{NPV} (\text{Negative Predictive Value})\):
  \[
  \text{NPV} = \frac{\text{TN}}{\text{TN} + \text{FN}}
  \]

- \(P (\text{Prevalence})\):
  \[
  P = \frac{\text{TP} + \text{FN}}{\text{TP} + \text{FP} + \text{TN} + \text{FN}}
  \]
  \section*{Additional Explanation}

Here, I discovered that the definitions of \(\text{PPV}\) and \(\text{NPV}\) are:
\[
\text{PPV} = \frac{\text{TP}}{\text{TP} + \text{FP}}, \quad \text{NPV} = \frac{\text{TN}}{\text{TN} + \text{FN}}
\]
So the previous calculations might have been incorrect. The formulas are based on the total size \(T\) and prevalence \(P\).

Given the total size \(T\) and the prevalence \(P\):
\[
P = \frac{\text{TP} + \text{FN}}{T}, \quad 1 - P = \frac{\text{TN} + \text{FP}}{T}
\]
\[
\text{TP} = \text{TPR} \cdot (\text{TP} + \text{FN})
\]
\[
\text{FN} = (1 - \text{TPR}) \cdot (\text{TP} + \text{FN})
\]
\[
\text{TN} = \text{TNR} \cdot (\text{TN} + \text{FP})
\]
\[
\text{FP} = (1 - \text{TNR}) \cdot (\text{TN} + \text{FP})
\]

Now we can express \(\text{TP} + \text{FP}\) and \(\text{TN} + \text{FN}\) as:
\[
\text{TP} + \text{FP} = T \cdot (P + (1 - \text{TNR}) \cdot (1 - P))
\]
\[
\text{TN} + \text{FN} = T \cdot ((1 - P) + (1 - \text{TPR}) \cdot P)
\]
\section*{Sharpe Ratio Analysis During 2007-2009}

We can analyze the Sharpe ratio during years when it was at the minimum. For example, during 2007-2009, I found that the ratio of true positive days to true negative days changed from 150:100 (3:2) to 1:1, and the accuracy also decreased. This matches the observed decrease in the Sharpe ratio.

If you have any questions or need further clarification, please let me know.
%xiang revision
\end{comment}


\begin{comment}
For every year, we have got GV and MC strategies

As before, let $T^{+}$ and $T^{(-)}$ be true positive and negative labels and let  $P^{+}$ and $P^{(-)}$ be predictive positive and negative labels respectively. 
Since $TP= T^{+}\cdot TPR$ and $TN=T^{(-)}\cdot TNR$


\begin{equation*}
\begin{split}
P^{+}&= TP \; +\;  T^{(-)}\cdot (1-TNR) = T^{+}\cdot TPR  \; + \; T^{(-)}\cdot (1-TNR)\\
P^{(-)}&= TN \; +\;  T^{(+)}\cdot (1-TPR) = T^{(-)}\cdot TNR \; + \; T^{(+)}\times (1-TPR)\\
\end{split}
\end{equation*}

%The number of predicted positive labels is then $\hbox{TP}+$

For our two strategies $GV-1L$ and $MC-1L$ we can represent this
schematically as follows:
\begin{equation*}
\begin{split}
    \hbox{GV-1L:} & \quad \overbrace{\underbrace{\underbrace{ +,+\cdots +,+}_{T^{(+)}\cdot TPR}\; , \;\underbrace{+,+\cdots\cdots +,+ }_{T^{(-)}\cdot (1-TNR)}}_{\hbox{(invested in S{\&}P Growth)}}}^{\hbox{predicted "+" labels }P^{(+)}}
    \;\; , \;\; 
    \overbrace{
\underbrace{\underbrace{-,-,\cdots\cdots  ,-,-}_{T^{(-)}\cdot TNR}\; , \; \underbrace{ -,-,\cdots -,-}_{T^{(+)}\cdot (1-TPR)}}_{\hbox{(invested in S{\&}P Value)}}}^{\hbox{predicted }"-"
\hbox{ labels } P^{(-)}} \\[10pt]
    \hbox{MC-1L:} &  \quad \overbrace{\underbrace{\underbrace{+,+\cdots +,+}_{T^{(+)}\cdot TPR}\; , \;\underbrace{+,+\cdots\cdots +,+ }_{T^{(-)}\cdot (1-TNR)}}_{\hbox{(invested in S{\&}P-500)}}}^{\hbox{predicted "+" labels } P^{(+)}}
    \;\; , \;\; 
    \overbrace{
\underbrace{\underbrace{-,-,\cdots\cdots  ,-,-}_{T^{(-)}\cdot TNR}\; , \; \underbrace{ -,-,\cdots -,-}_{T^{(+)}\cdot (1-TPR)}}_{\hbox{(Cash)}}}^{\hbox{predicted }"-"
\hbox{ labels }P^{(-)}} 
\end{split}
\end{equation*}

Note that for the total number of days $N$ we have 
\begin{equation*}
N=\vert P^{(+)}\vert + \vert P^{(-)}\vert  = \vert T^{(+)}\vert + \vert T^{(-)}\vert 
\end{equation*}
Let us define
 $\Delta =\vert P^{(+)}\vert  - \vert T^{(+)}\vert $.
Then we have the following three cases:
\begin{enumerate}
\item case $\Delta=0$: In this case, we have the following
\begin{equation*}
    \begin{split}
    T: & \overbrace{+,+,+,+,\ldots,+,+,+}^{T^{(+)}}\; , \;
    \overbrace{-,-,-,-,\ldots,\; -,-,-,-}^{T^{(-)}}\\
    P: & \underbrace{+,+,+,+,\ldots,+,+,+}_{P^{(+)}}\; , \;
    \underbrace{-,-,-,-,\ldots,\; -,-,-,-}_{P^{(-)}}
    \end{split}
\end{equation*}
In this case, the predicted positive and negative labels are exactly the true positive and negative labels. Since $r^{(+)}>r^{(-)}$ for this case, on the average, we will achieve the maximum return 
\begin{equation*}
R^{\max}=\vert T^{(+)}\vert r^{(+)} + \vert T^{(-)}\vert r^{(-)}
\end{equation*}

    \item Case $\Delta > 0$: In this case, we have the following
    \begin{equation*}
    \begin{split}
    T: & \overbrace{+,+,+,\ldots,+,+,+}^{T^{(+)}}\; , \;
    \overbrace{-,-,-,-,\; \ldots,\; -,-,-,-}^{T^{(-)}}\\
    P: &     \underbrace{\underbrace{+,+,+\ldots,+,+,+}_{T^{(+)}}\; , \; \underbrace{+,\ldots,+}_{\Delta}}_{P^{(+)}}\; , \; 
    \underbrace{-,-,-,\ldots,-,-}_{P^{(-)}}
\end{split}
\end{equation*}
For $\Delta$ days we are in the wrong index and would generate a return of
$\Delta \cdot r^{(+)}$ instead of $\Delta \cdot r^{(+)}$. This means that compared to the best return $R^{\max}$, our return will be $R^{\max}-\Delta\cdot (r^{(+)}-r^{(-)})$

\item Case $\Delta < 0$: 
    \begin{equation*}
    \begin{split}
    T: & \overbrace{+,+,+,+,\; \; \ldots,\; \; +,+,+,+}^{T^{(+)}}\; , \;
    \overbrace{-,-,-,\ldots,-,-,-}^{T^{(-)}}\\
    P: &   \underbrace{+,+,+,\ldots,+,+}_{P^{(+)}}\; , \; 
    \underbrace{\underbrace{-,\ldots,-}_{\Delta}\; , \;    -,-,-,\ldots,-,-}_{P^{(-)}}
\end{split}
\end{equation*}

For $\Delta$ days that are True Positive, we are in the wrong index. As a result, compared to the best return $R^{\max}$, our return will be $R^{\max}-\Delta\cdot (r^{(+)}-r^{(-)})$

  
\end{enumerate}

Therefore, the loss of return, compared to the best $R^{\max}$ is determined by two factors:
\begin{enumerate}
\item the difference in returns for True Positive and True negative label 
\item $\delta$ which is determined by our ability to predict True labels
\end{enumerate}

\end{comment}




\begin{comment}


We illustrate this with an example. Assume there are $N=250$ days in a year. Assume that there are  $T^{(+)}=100$ positive true labels
and $T^{(-)}=150$ negative true labels. Assume TPR=30\% and TNR=60\%.

Then $TP=30$ and $TN=90$. The number of predicted positive and negative labels are
\begin{equation*}
\begin{split}
    P^{(+)}&=TP + T^{(-)}\times (1-TNR) = 30+150\times 0.4 =90\\
    P^{(-)}&=TN + T^{(+)}\times (1-TPR) = 90 +100 \times 0.7= 160
    \end{split}
\end{equation*}

\end{comment}



\begin{table}[!ht]
    \centering
    \caption{Calculated p Values for Loser Strategies}
    \begin{tabular}{l|| rr}
    \hline
            \multirow{2}{*}{Year} & \multicolumn{2}{c}{Loser Strategies} \\ \cline{2-3}
         & GV-1L & MC-1L \\ \hline
        2001 & 0.22 & 0.27 \\ 
        2002 & -0.64 & 0.02 \\ 
        2003 & 3.66 & 0.55 \\ 
        2004 & -0.11 & 0.37 \\ 
        2005 & -0.13 & 0.74 \\ 
        2006 & 0.40 & 0.35 \\ 
        2007 & 0.96 & 1.32 \\ 
        2008 & -1.13 & 0.52 \\ 
        2009 & 0.69 & 0.45 \\ 
        2010 & 4.74 & 0.46 \\ 
        2011 & 1.23 & -0.53 \\ 
        2012 & -2.42 & 0.26 \\ 
        2013 & 5.27 & 0.43 \\ 
        2014 & 1.16 & 0.54 \\ 
        2015 & 0.69 & 1.80 \\ 
        2016 & 0.49 & 0.94 \\ 
        2017 & 0.66 & 0.53 \\ 
        2018 & 0.30 & 0.91 \\ 
        2019 & 3.57 & 0.37 \\ 
        2020 & 0.25 & 1.01 \\ 
        2021 & -0.25 & 0.49 \\ 
        2022 & 0.49 & 0.60 \\ 
        2023 & 0.53 & 0.46 \\ \hline\hline
        $\max$ & 5.27 & 1.80 \\ 
        $\min$ & -2.42 & -0.53 \\ 
        $M$ & 0.49 & 0.49 \\ 
        $\mu$  & 0.90 & 0.56 \\ 
        $\sigma$ & 1.81 & 0.45 \\ \hline\hline
    \end{tabular}
\end{table}
\end{comment}

We start with a growth comparison.


\begin{comment}
    

% Daily Return Contributions
The daily return contributions based on classification are defined as follows. For each category of outcome (True Positive, False Positive, etc.), a different return value is assigned:
\[
\text{Daily Return, } r_i = 
\begin{cases} 
r^{(+)}_G & \text{if True Positive (TP)} \\
r^{-}_G & \text{if False Positive (FP)} \\
r^{(-)}_V & \text{if True Negative (TN)} \\
r^{(+)}_V & \text{if False Negative (FN)}
\end{cases}
\]

% Step 2: Calculate Daily Average Return
The daily average return \(\overline{r}\) is calculated by weighting each return by its frequency:
\[
\overline{r} = \frac{TP \cdot r^{(+)}_G + FP \cdot r^{-}_G + TN \cdot r^{(-)}_V + FN \cdot r^{(+)}_V}{TP + FP + TN + FN}
\]

% Step 3: Calculate Variance of Daily Returns
To determine the variance of daily returns, each squared difference between individual returns and the average return is summed and normalized:
\[
\text{Var}(r) = \frac{1}{N-1} \left( TP \cdot (r^{(+)}_G - \overline{r})^2 + FP \cdot (r^{-}_G - \overline{r})^2 + TN \cdot (r^{(-)}_V - \overline{r})^2 + FN \cdot (r^{(+)}_V - \overline{r})^2 \right)
\]
where \( N = TP + FP + TN + FN \) represents the total number of trading days.

revised version(including correlation coeffcient):

\[
\text{Var}(r) = \frac{1}{N-1} \left[ 
\begin{array}{l}
TP \cdot (r^{(+)}_G - \overline{r})^2 + \\
FP \cdot (r^{-}_G - \overline{r})^2 + \\
TN \cdot (r^{(-)}_V - \overline{r})^2 + \\
FN \cdot (r^{(+)}_V - \overline{r})^2 + \\
2 \cdot \rho \cdot \sqrt{FP \cdot TN} \cdot (r^{-}_G - \overline{r}) \cdot (r^{(-)}_V - \overline{r}) + \\
2 \cdot \rho \cdot \sqrt{TP \cdot FN} \cdot (r^{(+)}_G - \overline{r}) \cdot (r^{(+)}_V - \overline{r})
\end{array}
\right]
\]


% Step 4: Calculate Daily Standard Deviation
The daily standard deviation (\(\sigma\)) is:
\[
\sigma = \sqrt{\text{Var}(r)}
\]

% Step 5: Calculate Annualized Volatility
The annualized volatility (\(\sigma_{\text{annual}}\)) is:
\[
\sigma_{\text{annual}} = \sigma \times \sqrt{N}
\]
\end{comment}

\begin{comment}
The annualized standard deviation, $\sigma_{\text{annual}}$, is calculated by the formula:
\[
\sigma_{\text{annual}} = \sqrt{\sigma_G^2 \cdot (TP + FP) + \sigma_V^2 \cdot (TN + FN)}
\]
\[
\sigma_{\text{annual}} = \sqrt{\sigma_G^2 \cdot \left(\text{TPR} \cdot (TP + FN) + \text{PPR} \cdot (TP + FP)\right) + \sigma_V^2 \cdot \left(\text{TNR} \cdot (TN + FP) + \text{NPR} \cdot (TN + FN)\right)}
\]
\end{comment}
\begin{comment}
where:
\begin{itemize}
    \item $\sigma_G$ is the standard deviation for the group including true positives (TP) and false positives (FP).
    \item $\sigma_V$ is the standard deviation for the group including true negatives (TN) and false negatives (FN).
    \item $TP$ represents the number of true positives.
    \item $FP$ represents the number of false positives.
    \item $TN$ represents the number of true negatives.
    \item $FN$ represents the number of false negatives.
    \item $TPR$ (True Positive Rate) indicates the proportion of positives that are correctly identified as such.
    \item $TNR$ (True Negative Rate) indicates the proportion of negatives that are correctly identified as such.
    \item $PPR$ (Positive Predictive Rate) indicates the proportion of positive identifications that were actually correct.
    \item $NPR$ (Negative Predictive Rate) indicates the proportion of negative identifications that were actually correct.
\end{itemize}
\end{comment}


\begin{comment}
 
\begin{figure}[htbp]
  \centering
  \includegraphics[width=\linewidth]{plots/Yearly Performance Comparison Winner vs Loser Strategies.png}
  \caption{Yearly Performance Comparison Winner vs Loser Strategies}
  \label{fig:winer_loser_fridays}
\end{figure}
\begin{figure}[htbp]
  \centering
  \includegraphics[width=\linewidth]{plots/Max Drawdown Winner Loser.png}
  \caption{Yearly Max Drawdown: Winner vs Loser Strategies}
  \label{fig:winer_loser_fridays}
\end{figure}
Graphic ~\ref{fig:winer_loser_fridays} shows the value of winer and loser strategies.

\begin{figure}[htbp]
  \centering
  \includegraphics[width=\linewidth]{plots/Performance Comparison of Investment Strategies.png}
  \caption{Performance Comparison of Investment Strategies from a \$100}
  \label{fig:winer_loser_fridays}
\end{figure}
    

\begin{table}[!ht]
    \centering
    \caption{Percentage of Strategy Performance to Ideal Outcomes}
\begin{tabular}{lrrrrr}
\hline
Year & S\&P & Growth & Value & GV-1L & MC-1L \\
\hline
2001 & -2.4 & \cellcolor{red!25}-5.3 & -1.1 & \cellcolor{green!25}3.5 & -1.0 \\
2002 & -9.6 & \cellcolor{red!25}-14.3 & -8.0 & \cellcolor{green!25}15.2 & 4.0 \\
2003 & 17.2 & 17.3 & 15.4 & \cellcolor{green!25}31.4 & \cellcolor{red!25}10.8 \\
2004 & 14.1 & 7.0 & 17.4 & \cellcolor{green!25}24.3 & \cellcolor{red!25}6.0 \\
2005 & 8.9 & \cellcolor{red!25}5.1 & 9.8 & 17.6 & \cellcolor{green!25}19.9 \\
2006 & 27.6 & 15.7 & \cellcolor{green!25}37.5 & 33.0 & \cellcolor{red!25}7.5 \\
2007 & 8.9 & 18.6 & \cellcolor{red!25}2.4 & 21.7 & \cellcolor{green!25}33.5 \\
2008 & -65.4 & \cellcolor{red!25}-66.5 & -64.5 & -47.2 & \cellcolor{green!25}18.8 \\
2009 & 19.0 & \cellcolor{green!25}26.6 & \cellcolor{red!25}12.3 & 24.8 & 14.6 \\
2010 & 19.5 & 21.0 & 20.0 & \cellcolor{green!25}29.4 & \cellcolor{red!25}17.4 \\
2011 & 2.9 & 7.0 & -1.1 & \cellcolor{green!25}14.0 & \cellcolor{red!25}-5.2 \\
2012 & 19.0 & 16.8 & 20.5 & \cellcolor{green!25}43.1 & \cellcolor{red!25}0.8 \\
2013 & 41.7 & 42.0 & 41.0 & \cellcolor{green!25}51.9 & \cellcolor{red!25}20.6 \\
2014 & 23.5 & 25.8 & 21.2 & \cellcolor{green!25}29.4 & \cellcolor{red!25}19.1 \\
2015 & 2.5 & \cellcolor{green!25}10.5 & \cellcolor{red!25}-6.6 & 9.5 & 0.5 \\
2016 & 17.4 & \cellcolor{red!25}9.9 & 24.7 & 16.0 & \cellcolor{green!25}26.0 \\
2017 & 30.5 & \cellcolor{green!25}38.2 & 21.6 & 29.8 & \cellcolor{red!25}18.7 \\
2018 & -6.8 & \cellcolor{green!25}-0.2 & -13.5 & -10.1 & \cellcolor{red!25}-18.9 \\
2019 & 37.0 & 36.5 & \cellcolor{green!25}37.5 & 30.4 & \cellcolor{red!25}10.9 \\
2020 & 8.6 & 15.8 & \cellcolor{red!25}0.6 & 2.8 & \cellcolor{green!25}17.0 \\
2021 & 13.3 & \cellcolor{green!25}14.9 & 11.6 & 7.6 & \cellcolor{red!25}6.5 \\
2022 & -14.3 & \cellcolor{red!25}-23.2 & \cellcolor{green!25}-4.2 & -10.3 & -5.2 \\
2023 & 30.4 & \cellcolor{green!25}34.9 & 25.8 & 25.9 & \cellcolor{red!25}19.2 \\ \hline
$\max$ & 41.7 & 42.0 & 41.0 & \cellcolor{green!25}51.9 & \cellcolor{red!25}33.5 \\
$\min$ & -65.4 & \cellcolor{red!25}-66.5 & -64.5 & -47.2 & \cellcolor{green!25}-18.9 \\
$M$ & 14.1 & 15.7 & 12.3 & \cellcolor{green!25}21.7 & \cellcolor{red!25}10.9 \\
$\mu$  & 10.6 & 11.1 & \cellcolor{red!25}9.6 & \cellcolor{green!25}17.1 & 10.5 \\
$\sigma$ & 22.0 & \cellcolor{green!25}23.3 & 22.2 & 20.6 & \cellcolor{red!25}11.9 \\
\hline
\end{tabular}
\end{table}

\begin{comment}
  
\begin{figure}[htbp]
  \centering
  \includegraphics[width=\linewidth]{plots/Comparison of GV-1L and MC-1L Strategies Against Ideal Benchmarks.png}
  \caption{Comparison of GV-1L and MC-1L Strategies Against Ideal Benchmarks}
  \label{fig:winer_loser_fridays}
\end{figure}
\begin{figure}[htbp]
  \centering
  \includegraphics[width=\linewidth]{plots/Annual Correlation.png}
  \caption{Annual Correlation}
  \label{fig:winer_loser_fridays}
\end{figure}

\end{comment}



% add here





















\begin{comment}
It is instructive to compare the performance of these strategies during the best and worst days for the S{\&}P-500 during 
2001-2023. 

The data for the best 10 days is presented in Table

\begin{table}[!ht]
    \centering
        \caption{Strategy Choices for Best Ten Days For S\&P-500}
    \begin{tabular}{|c | rrr | ll  | ll|}

    \hline
         \multirow{2}{*}{Date}    & \multicolumn{3}{c|}{Indices} & \multicolumn{2}{c|}{Growth-vs-Value}  & \multicolumn{2}{c|}{Index-vs-Cash} \\ 
                 & S{\&}P  & Growth & Value  & GV-1W    & GV-1L  & MC-1W & MC-1L \\ \hline
        10/13/08 & 14.5 & 11.9 & 10.2 & Value  & Growth & Cash    & SPY \\ \hline
        10/28/08 & 11.7 & 7.2  & 10.5 & Growth & Value  & Cash    & SPY \\ \hline
        03/24/20  & 9.1 & 8.9  & 9.7  & Growth & Value  & Cash    & SPY \\ \hline
        03/13/20  & 8.5  & 9.3 & 8.9   & Growth & Value  & Cash    & SPY \\ \hline
        03/23/09  & 7.2  & 6.6 & 7.2   & Growth & Value  & Cash    & SPY \\ \hline
        11/24/08 & 6.9  & 7.4 & 7.6    & Value  & Growth & SPY     & Cash \\ \hline
        04/06/20 & 6.7  & 7.3 & 6.4    & Growth & Value  & Cash    & SPY \\ \hline
        11/13/08 & 6.2  & 6.7 & 6.4    & Growth & Value  & Cash    & SPY \\ \hline
        10/20/08 & 6.0   & 3.7 & 4.3    & Growth & Value  & Cash    & SPY \\ \hline
        07/24/02  & 6.0  & 5.3 & 6.7   & Value  & Growth & Cash    & SPY \\ \hline
        Cum. Return   & 120.9  & 104.4 & 111.4   & 104.3  & 111.5 & 6.9   & 107 \\  \hline
    \end{tabular}
\end{table}

The Market-Cash winners strategy was in a cash position for 9/10 best days. It made only 6.9\% whereas the other strategy more than doubled over these 10 days.

For the worst 10 days of S{\&}P we have the following data

\begin{table}[!ht]
    \centering
    \caption{Strategy Choices for Worst Ten Days For S\&P-500}
    \begin{tabular}{|c | rrr | ll  | ll|}
    \hline
     \multirow{2}{*}{Date}    & \multicolumn{3}{c|}{Indices} & \multicolumn{2}{c|}{Growth-vs-Value}  & \multicolumn{2}{c|}{Index-vs-Cash} \\ 
                 & S{\&}P  & Growth & Value  & GV-1W    & GV-1L  & MC-1W & MC-1L \\ \hline
        11/19/08 & -6.4 & -5.0 & -5.4  & Value   & Growth & SPY     & Cash \\ \hline
        08/08/11 & -6.5   & -5.8 & -6.6  & Value   & Growth & Cash    & SPY \\ \hline
        10/09/08 & -7.0  & -5.2 & -9.7  & Growth  & Value  & Cash    & SPY \\ \hline
        11/20/08 & -7.4 & -6.8 & -7.6  & Growth  & Value  & Cash    & SPY \\ \hline
        03/09/20 & -7.8   & -7.4 & -8.3  & Growth  & Value  & Cash    & SPY \\ \hline
        09/29/08 & -7.8  & -7.0 & -6.2  &  Growth  & Value  & SPY     & Cash \\ \hline
        12/01/08 & -8.9  & -7.4 & -9.3  & Value   & Growth & SPY     & Cash \\ \hline
        03/12/20 & -9.6  & -9.0 & -10.1 & Growth  & Value  & Cash    & SPY \\ \hline
        10/15/08 & -9.8 & -10.2 & -8.9 & Value   & Growth & Cash    & SPY \\ \hline
        03/16/20 & -10.9 & -12.0 & -11.2 & Growth & Value  & SPY     & Cash \\ \hline
         Cum. Return   & -53.9  & -54.7 & -58.2   & -55.5  & -57.4 & -30.0  & -39.5 \\  \hline       
    \end{tabular}
\end{table}

For the worst days, the Market-Cash Winners strategy lost the least, about 1/2 of the losses for the other strategy. However, it has an abysmal performance on the best days of S{\&|P. If we consider the performance of our strategies over these 20 days, we will obtain:

\end{comment}


\begin{comment}
We consider the same 11 days as before. Our results are summarized in Table below

\begin{table}[!ht]
    \centering
    \caption{Comparison of GV-1L vs MC-1L Strategies}
    \vspace{0.1in}
    \begin{tabular}{l | l | c c c c c c  c c c c c}
    \hline
    \multicolumn{2}{c|}{Day $i$}   &  $d_{1}$ & $d_{2}$ & $d_{3}$ & $d_{4}$ & $d_{5}$ & $d_{6}$ & $d_{7}$ & $d_{8}$ & $d_{9}$ & $d_{10}$ & $d_{11}$\\ \hline
    \multicolumn{13}{c}{GV-1L}\\ \hline

    \multirow{3}{*}{$R_{i}$} & Growth & 3 & 2 & -3 & -2 & 2  & 3 & 0 & -1 & -3 & 3 & -2\\
       & Value  & 1 & 0 & -2 & -1 & 1 & 2  & 1 & -2 & 1 & 2 & -1\\
       & Strategy & n/a & 0 & -2 & -2 & 2 & 2 & 1 & -1 & 1 & 3 & -1\\ \hline
       \multirow{2}{*}{$L_{i}$}  & True & $+$ & $+$ & $-$ & $-$ & $+$ & $+$ & $-$ & $+$ & $-$ & $+$ & $-$ \\
       & Predicted & n/a & $-$ & $-$ & $+$ & $+$ & $-$ & $-$ & $+$ & $-$ & $+$ & $-$\\
        \hline
    \multicolumn{13}{c}{MC-1W}\\ \hline
    \multirow{2}{*}{$R_{i}$} &   S{\&}P-500 & 3 & 1   & -2.5  & -1.5   & 1.5   & 2.5   & 0.5  & -1.5  & -1  & 2.5 & -1.5\\
        & Strategy & n/a & 1 & -2.5 & 0 & 0 & 2.5 & 0.5 & -1.5 & 0 & 0 & -1.5\\ \hline
    \multirow{2}{*}{$L_{i}$}  &  True &  $+$ & $+$ & $-$ & $+$ & $+$ & $+$ & $-$ & $-$ & $-$ & $+$ & $-$\\
      &   Predicted  & n/a & $+$ & $+$ & $-$ & $+$ & $+$ & $+$ & $-$ & $-$ & $-$ & $+$\\
        \hline
\end{tabular}
\label{tab_label_assignment}
\end{table}
\end{comment}




\begin{comment}
\begin{table}[!ht]
    \centering
    \caption{Strategy Performance Example Against SPYV and SPYG Daily Returns}
    \begin{tabular}{llllllllll}
    \hline
        Date & 12/30/05 & 1/1/00 & 1/2/00 & 1/5/06 & 1/6/06 & 1/9/06 & 1/10/06 & 1/11/06 & 1/12/06 \\ \hline
        SPYV & -0.5\% & 1.6\% & 0.4\% & 0.1\% & 0.9\% & 0.3\% & -0.2\% & 0.5\% & -0.6\% \\ 
        SPYG & -0.5\% & 1.4\% & 0.7\% & -0.1\% & 1.0\% & 0.6\% & 0.0\% & 0.3\% & -0.4\% \\ 
        Label & + & + & - & + & - & - & - & + & - \\ 
        Growth & G & G & G & G & G & G & G & G & G \\ 
        Value & V & V & V & V & V & V & V & V & V \\ 
        Winer & N/A & V & V & G & V & G & G & G & V \\ 
        Loser & N/A & G & G & V & G & V & V & V & G \\ \hline
    \end{tabular}
    \label{tab_comparison_example}
\end{table}
In Table~\ref{tab_pp}, the ‘SPYG>SPYV’ column represents the percentage probability that SPYG's daily returns exceed SPYV's during the observation period.
\end{comment}



\begin{comment}
\begin{table}[!ht]
    \centering
    \caption{Best Ten Days For S\&P}
    \begin{tabular}{|l|l|l|l|l|l|l|l|l|l|}
    \hline
        Date & SPY & SPYV & SPYG & GV-1W & GV-1L & GV-3W & GV-3L & SC GV-1W & SC GV-1L \\ \hline
        10/13/08 & 14.5 & 10.2 & 11.9 & Value & Growth & Growth & Value & Cash & SPY \\ \hline
        10/28/08 & 11.7 & 10.5 & 7.2 & Growth & Value & Growth & Value & Cash & SPY \\ \hline
        3/24/20 & 9.1 & 9.7 & 8.9 & Growth & Value & Growth & Value & Cash & SPY \\ \hline
        3/13/20 & 8.5 & 8.9 & 9.3 & Growth & Value & Growth & Value & Cash & SPY \\ \hline
        3/23/09 & 7.2 & 7.2 & 6.6 & Growth & Value & Growth & Value & Cash & SPY \\ \hline
        11/24/08 & 6.9 & 7.6 & 7.4 & Value & Growth & Growth & Value & SPY & Cash \\ \hline
        4/6/20 & 6.7 & 6.4 & 7.3 & Growth & Value & Value & Growth & Cash & SPY \\ \hline
        11/13/08 & 6.2 & 6.4 & 6.7 & Growth & Value & Value & Growth & Cash & SPY \\ \hline
        10/20/08 & 6.0 & 4.3 & 3.7 & Growth & Value & Growth & Value & Cash & SPY \\ \hline
        7/24/02 & 6.0 & 6.7 & 5.3 & Value & Growth & Value & Growth & Cash & SPY \\ \hline
    \end{tabular}
\end{table}
\begin{table}[!ht]
    \centering
    \caption{Worst Ten Days For S\&P}
    \begin{tabular}{|l|l|l|l|l|l|l|l|l|l|}
    \hline
        Date & SPY & SPYV & SPYG & GV-1W & GV-1L & GV-3W & GV-3L & SC GV-1W & SC GV-1L \\ \hline
        11/19/08 & -6.4 & -5.4 & -5.0 & Value & Growth & Value & Growth & SPY & Cash \\ \hline
        8/8/11 & -6.5 & -6.6 & -5.8 & Value & Growth & Growth & Value & Cash & SPY \\ \hline
        10/9/08 & -7.0 & -9.7 & -5.2 & Growth & Value & Growth & Value & Cash & SPY \\ \hline
        11/20/08 & -7.4 & -7.6 & -6.8 & Growth & Value & Value & Growth & Cash & SPY \\ \hline
        3/9/20 & -7.8 & -8.3 & -7.4 & Growth & Value & Growth & Value & Cash & SPY \\ \hline
        9/29/08 & -7.8 & -6.2 & -7.0 & Growth & Value & Growth & Value & SPY & Cash \\ \hline
        12/1/08 & -8.9 & -9.3 & -7.4 & Value & Growth & Value & Growth & SPY & Cash \\ \hline
        3/12/20 & -9.6 & -10.1 & -9.0 & Growth & Value & Growth & Value & Cash & SPY \\ \hline
        10/15/08 & -9.8 & -8.9 & -10.2 & Value & Growth & Value & Growth & Cash & SPY \\ \hline
        3/16/20 & -10.9 & -11.2 & -12.0 & Growth & Value & Growth & Value & SPY & Cash \\ \hline
    \end{tabular}
\end{table}
\end{comment}





\begin{comment}

\begin{table}[!ht]
    \centering
    \caption{TPR And TNR For Winner And Loser Strategies}
    \begin{tabular}{l | rrr| rrr | rrr| rrr}
    \hline
\multirow{3}{*}{Year} & \multicolumn{3}{c|}{GV-1W }   
& \multicolumn{3}{c|}{GV-1L}  & \multicolumn{3}{c|}{MC-1W} & \multicolumn{3}{c}{MC-1L}  \\
 & TPR & TNR & Acc& TPR & TNR & Acc& TPR & TNR &Acc & TPR & TNR &Acc \\
\hline
2001 & \cellcolor{red!25}44.6 & 47.2 & 46.0 & \cellcolor{green!25}55.4 & 52.8 & 54.0 & 48.8 & 48.0 & 48.4 & 51.2 & 52.0 & 51.6 \\
2002 & \cellcolor{red!25}31.4 & 40.3 & 36.1 & \cellcolor{green!25}68.6 & 59.7 & 63.9 & 39.2 & 45.5 & 42.5 & 60.8 & 54.5 & 57.5 \\
2003 & 42.5 & 41.6 & 42.1 & 57.5 & 58.4 & 57.9 & 54.2 & \cellcolor{red!25}38.9 & 47.6 & 45.8 & \cellcolor{green!25}61.1 & 52.4 \\
2004 & 41.7 & 46.2 & 44.0 & 58.3 & 53.8 & 56.0 & \cellcolor{green!25}58.6 & 43.0 & 52.0 & \cellcolor{red!25}41.4 & 57.0 & 48.0 \\
2005 & 45.5 & 49.6 & 47.6 & 54.5 & 50.4 & 52.4 & 53.6 & \cellcolor{red!25}42.0 & 48.4 & 46.4 & \cellcolor{green!25}58.0 & 51.6 \\
2006 & \cellcolor{red!25}38.5 & 53.5 & 47.0 & \cellcolor{green!25}61.5 & 46.5 & 53.0 & 58.9 & 47.3 & 53.8 & 41.1 & 52.7 & 46.2 \\
2007 & 48.6 & 37.2 & 43.4 & 51.4 & 62.8 & 56.6 & 48.9 & \cellcolor{red!25}36.6 & 43.4 & 51.1 & \cellcolor{green!25}63.4 & 56.6 \\
2008 & \cellcolor{red!25}47.2 & 47.7 & 47.4 & \cellcolor{green!25}52.8 & 52.3 & 52.6 & 50.0 & 51.2 & 50.6 & 50.0 & 48.8 & 49.4 \\
2009 & 53.8 & \cellcolor{red!25}37.4 & 46.8 & 46.2 & \cellcolor{green!25}62.6 & 53.2 & 56.0 & 43.2 & 50.4 & 44.0 & 56.8 & 49.6 \\
2010 & 45.6 & 46.5 & 46.0 & 54.4 & 53.5 & 54.0 & 57.1 & \cellcolor{red!25}41.0 & 50.4 & 42.9 & \cellcolor{green!25}59.0 & 49.6 \\
2011 & 44.9 & 44.8 & 44.8 & 55.1 & 55.2 & 55.2 & \cellcolor{green!25}56.6 & 48.3 & 52.8 & \cellcolor{red!25}43.4 & 51.7 & 47.2 \\
2012 & \cellcolor{red!25}35.3 & 44.0 & 40.0 & \cellcolor{green!25}64.7 & 56.0 & 60.0 & 59.0 & 49.5 & 54.8 & 41.0 & 50.5 & 45.2 \\
2013 & 47.4 & 41.2 & 44.4 & 52.6 & 58.8 & 55.6 & 57.0 & \cellcolor{red!25}37.9 & 49.2 & 43.0 & \cellcolor{green!25}62.1 & 50.8 \\
2014 & 52.9 & 41.1 & 47.6 & 47.1 & 58.9 & 52.4 & 56.4 & \cellcolor{red!25}35.9 & 48.0 & 43.6 & \cellcolor{green!25}64.1 & 52.0 \\
2015 & 51.8 & \cellcolor{red!25}41.7 & 47.2 & 48.2 & \cellcolor{green!25}58.3 & 52.8 & 45.5 & 49.6 & 47.6 & 54.5 & 50.4 & 52.4 \\
2016 & 51.2 & 48.8 & 50.0 & 48.8 & 51.2 & 50.0 & 46.4 & \cellcolor{red!25}35.1 & 41.3 & 53.6 & \cellcolor{green!25}64.9 & 58.7 \\
2017 & 54.5 & 37.7 & 47.4 & 45.5 & 62.3 & 52.6 & 53.5 & \cellcolor{red!25}37.4 & 46.6 & 46.5 & \cellcolor{green!25}62.6 & 53.4 \\
2018 & \cellcolor{green!25}58.3 & 44.9 & 52.6 & \cellcolor{red!25}41.7 & 55.1 & 47.4 & 55.6 & 49.1 & 52.6 & 44.4 & 50.9 & 47.4 \\
2019 & 52.8 & 54.3 & 53.6 & 47.2 & 45.7 & 46.4 & \cellcolor{green!25}62.0 & 44.1 & 54.8 & \cellcolor{red!25}38.0 & 55.9 & 45.2 \\
2020 & 62.0 & 44.7 & 54.9 & 38.0 & 55.3 & 45.1 & 50.7 & \cellcolor{red!25}32.7 & 43.1 & 49.3 & \cellcolor{green!25}67.3 & 56.9 \\
2021 & 52.8 & 53.5 & 53.2 & 47.2 & 46.5 & 46.8 & 56.8 & \cellcolor{red!25}39.6 & 49.6 & 43.2 & \cellcolor{green!25}60.4 & 50.4 \\
2022 & 43.4 & 54.3 & 49.4 & 56.6 & 45.7 & 50.6 & 44.5 & \cellcolor{green!25}56.7 & 51.4 & 55.5 & \cellcolor{red!25}43.3 & 48.6 \\
2023 & 56.3 & \cellcolor{red!25}41.7 & 50.0 & 43.7 & \cellcolor{green!25}58.3 & 50.0 & 55.3 & 42.2 & 49.6 & 44.7 & 57.8 & 50.4 \\
\hline

$\max$ & 62.0 & \cellcolor{red!25}54.3 & 54.9 & \cellcolor{green!25}68.6 & 62.8 & 63.9 & 62.0 & 56.7 & 54.8 & 60.8 & 67.3 & 58.7 \\
$\min$ & \cellcolor{red!25}31.4 & 37.2 & 36.1 & 38.0 & \cellcolor{green!25}45.7 & 45.1 & 39.2 & 32.7 & 41.3 & 38.0 & 43.3 & 45.2 \\
$M$ & 47.4 & 44.8 & 47.2 & 52.6 & 55.2 & 52.8 & 55.3 & \cellcolor{red!25}43.0 & 49.6 & 44.7 & \cellcolor{green!25}57.0 & 50.4 \\
$\mu$  & 48.0 & 45.2 & 47.0 & 52.0 & 54.8 & 53.0 & 53.2 & \cellcolor{red!25}43.3 & 49.1 & 46.8 & \cellcolor{green!25}56.7 & 50.9 \\
$\sigma$ & \cellcolor{green!25}7.4 & 5.3 & 4.4 & \cellcolor{green!25}7.4 & 5.3 & 4.4 & 5.6 & 6.1 & \cellcolor{red!25}3.8 & 5.6 & 6.1 & \cellcolor{red!25}3.8 \\



\hline
\end{tabular}
\end{table}
\end{comment}


\begin{comment}
\begin{table}[!ht]
    \centering
    \caption{Comparison of AUC Scores}
\begin{tabular}{l | rr | rr}
\hline
\multirow{2}{*}{Year} & \multicolumn{2}{r|}{Growth-Value} & 
 \multicolumn{2}{r}{Market-Cash} \\
     & GV-1W & GV-1L & MC-1W & MC-1L \\
\hline
2001 & \cellcolor{red!25}45.9 & \cellcolor{green!25}54.1 & 48.4 & 51.6 \\
2002 & \cellcolor{red!25}35.8 & \cellcolor{green!25}64.2 & 42.3 & 57.7 \\
2003 & \cellcolor{red!25}42.1 & \cellcolor{green!25}57.9 & 46.5 & 53.5 \\
2004 & \cellcolor{red!25}43.9 & \cellcolor{green!25}56.1 & 50.8 & 49.2 \\
2005 & \cellcolor{red!25}47.5 & \cellcolor{green!25}52.5 & 47.8 & 52.2 \\
2006 & \cellcolor{red!25}46.0 & \cellcolor{green!25}54.0 & 53.1 & 46.9 \\
2007 & 42.9 & 57.1 & \cellcolor{red!25}42.8 & \cellcolor{green!25}57.2 \\
2008 & \cellcolor{red!25}47.4 & \cellcolor{green!25}52.6 & 50.6 & 49.4 \\
2009 & \cellcolor{red!25}45.6 & \cellcolor{green!25}54.4 & 49.6 & 50.4 \\
2010 & \cellcolor{red!25}46.0 & \cellcolor{green!25}54.0 & 49.0 & 51.0 \\
2011 & \cellcolor{red!25}44.8 & \cellcolor{green!25}55.2 & 52.4 & 47.6 \\
2012 & \cellcolor{red!25}39.7 & \cellcolor{green!25}60.3 & 54.3 & 45.7 \\
2013 & \cellcolor{red!25}44.3 & \cellcolor{green!25}55.7 & 47.5 & 52.5 \\
2014 & 47.0 & 53.0 & \cellcolor{red!25}46.1 & \cellcolor{green!25}53.9 \\
2015 & \cellcolor{red!25}46.8 & \cellcolor{green!25}53.2 & 47.5 & 52.5 \\
2016 & 50.0 & 50.0 & \cellcolor{red!25}40.7 & \cellcolor{green!25}59.3 \\
2017 & 46.1 & 53.9 & \cellcolor{red!25}45.4 & \cellcolor{green!25}54.6 \\
2018 & 51.6 & 48.4 & \cellcolor{green!25}52.3 & \cellcolor{red!25}47.7 \\
2019 & \cellcolor{green!25}53.6 & \cellcolor{red!25}46.4 & 53.1 & 46.9 \\
2020 & 53.3 & 46.7 & \cellcolor{red!25}41.7 & \cellcolor{green!25}58.3 \\
2021 & \cellcolor{green!25}53.2 & \cellcolor{red!25}46.8 & 48.2 & 51.8 \\
2022 & \cellcolor{red!25}48.9 & \cellcolor{green!25}51.1 & 50.6 & 49.4 \\
2023 & 49.0 & 51.0 & \cellcolor{red!25}48.8 & \cellcolor{green!25}51.2 \\ \hline
$\max$ & \cellcolor{red!25}53.6 & \cellcolor{green!25}64.2 & 54.3 & 59.3 \\
$\min$ & \cellcolor{red!25}35.8 & \cellcolor{green!25}46.4 & 40.7 & 45.7 \\
$M$ & \cellcolor{red!25}46.1 & \cellcolor{green!25}53.9 & 48.4 & 51.6 \\
$\mu$  & \cellcolor{red!25}46.6 & \cellcolor{green!25}53.4 & 48.2 & 51.8 \\
$\sigma$ & \cellcolor{green!25}4.3 & \cellcolor{green!25}4.3 & \cellcolor{red!25}3.8 & \cellcolor{red!25}3.8 \\
\hline
\end{tabular}

\end{table}
\end{comment}

\begin{comment}
\begin{table}[!ht]
    \centering
    \caption{Annual Maximum Drawdown and Volatility of Investment Strategies}
    \medskip
    \begin{tabular}{l | rrr | rr | rrr | rr }
    \hline
\multirow{3}{*}{Year} & \multicolumn{5}{c|}{Maximum Drawdowns}& \multicolumn{5}{c}{Annual Volatility}     \\\hline
 & \multicolumn{3}{c|}{Buy-and-Hold}   
& \multicolumn{2}{c|}{$k=1$  day} &   \multicolumn{3}{c|}{Buy-and-Hold}   & \multicolumn{2}{c}{$k=1$  day}   \\
  & S\&P & G & V   & GV-1L & MC-1L & S\&P & G & V  & GV-1L &  MC-1L \\ \hline
   2001 & -28.8 & -48.6 & -18.1 & -32.0 & -19.7 & 21.9 & \cellcolor{green!25}40.7 & 17.2 & 33.4 & 16.2 \\ 
        2002 & -33.0 & -40.1 & -31.2 & -26.4 & -18.0 & 26.4 & \cellcolor{green!25}33.1 & 24.5 & 29.2 & 19.5 \\ 
        2003 & -13.7 & -12.8 & -15.3 & -13.9 & \cellcolor{green!25}-6.5 & 16.5 & \cellcolor{green!25}18.2 & 16.5 & 17.2 & \cellcolor{red!25}10.9 \\ 
        2004 & -7.5 & -10.8 & -7.5 & -6.5 & -6.6 & 11.1 & 11.1 & 10.5 & 11.2 & \cellcolor{red!25}7.4 \\ 
        2005 & -7.0 & -8.0 & -6.2 & -5.6 & \cellcolor{green!25}-4.9 & \cellcolor{green!25}10.3 & 10.2 & 9.8 & 10.1 & \cellcolor{red!25}7.1 \\ 
        2006 & -7.6 & \cellcolor{red!25}-9.0 & -7.4 & -7.7 & -6.8 & 10.0 & \cellcolor{green!25}10.7 & 9.8 & 10.6 & \cellcolor{red!25}7.0 \\ 
        2007 & -9.9 & -9.0 & -12.0 & -10.1 & \cellcolor{green!25}-5.7 & 15.9 & 14.6 & \cellcolor{green!25}16.7 & 15.9 & \cellcolor{red!25}11.0 \\ 
        2008 & -47.1 & -48.0 & -47.0 & -41.8 & \cellcolor{green!25}-20.9 & \cellcolor{green!25}41.4 & 37.4 & 40.9 & 39.9 & 33.3 \\ 
        2009 & -27.1 & -23.0 & \cellcolor{red!25}-30.6 & -28.4 & \cellcolor{green!25}-14.7 & 26.6 & 25.5 & \cellcolor{green!25}27.8 & 26.9 & 18.9 \\ 
        2010 & -15.7 & -16.3 & -14.5 & -14.3 & \cellcolor{green!25}-8.0 & 17.9 & \cellcolor{green!25}18.8 & 17.8 & 18.1 & 14.6 \\ 
        2011 & -18.6 & -16.5 & -21.9 & -16.8 & -16.1 & 23.0 & 21.5 & \cellcolor{green!25}23.7 & 22.2 & 16.3 \\ 
        2012 & -9.7 & -8.3 & -11.2 & -6.3 & -9.2 & 12.7 & 12.0 & \cellcolor{green!25}13.9 & 13.1 & 9.5 \\ 
        2013 & -5.6 & -5.8 & -5.1 & \cellcolor{green!25}-4.3 & -4.7 & \cellcolor{green!25}11.1 & 10.9 & 10.4 & 10.8 & \cellcolor{red!25}7.7 \\ 
        2014 & -7.3 & -7.4 & -7.4 & -7.4 & \cellcolor{green!25}-4.2 & 11.2 & \cellcolor{green!25}12.2 & 10.3 & 11.3 & 8.0 \\ 
        2015 & -11.9 & -11.8 & \cellcolor{red!25}-13.6 & -13.2 & -11.7 & 15.4 & \cellcolor{green!25}15.8 & 15.3 & 15.7 & 12.6 \\ 
        2016 & -9.2 & -9.5 & -8.6 & \cellcolor{red!25}-10.8 & \cellcolor{green!25}-3.8 & 13.1 & 13.4 & 13.3 & \cellcolor{green!25}13.8 & \cellcolor{red!25}8.8 \\ 
        2017 & -2.6 & -2.4 & \cellcolor{red!25}-4.4 & -3.3 & \cellcolor{green!25}-1.8 & 6.7 & 7.2 & 7.2 & 7.2 & 4.9 \\ 
        2018 & -19.3 & -20.6 & -19.2 & -19.5 & -19.9 & 17.0 & \cellcolor{green!25}19.5 & 15.1 & 17.1 & 14.0 \\ 
        2019 & -6.6 & -6.4 & -7.7 & -7.3 & -6.6 & 12.5 & 12.9 & 12.8 & \cellcolor{green!25}13.0 & 9.1 \\ 
        2020 & -33.7 & -31.3 & \cellcolor{red!25}-36.9 & -36.0 & \cellcolor{green!25}-17.6 & 33.5 & 34.7 & 35.2 & \cellcolor{green!25}35.3 & 26.3 \\ 
        2021 & -5.1 & \cellcolor{red!25}-8.7 & -5.8 & -7.9 & \cellcolor{green!25}-4.3 & 13.0 & \cellcolor{green!25}16.3 & 13.0 & 15.3 & 9.4 \\ 
        2022 & -24.5 & -32.3 & -17.9 & -19.5 & -17.0 & 24.2 & \cellcolor{green!25}30.6 & 19.2 & 26.3 & 18.0 \\ 
        2023 & -10.0 & -9.1 & -10.9 & -9.8 & \cellcolor{green!25}-7.0 & 13.0 & 13.3 & 13.3 & \cellcolor{green!25}13.5 & 9.3 \\ 
        \hline
        $\max$ & -2.6 & -2.4 & \cellcolor{red!25}-4.4 & -3.3 & \cellcolor{green!25}-1.8 & 21.9 & \cellcolor{red!25}10.9 & \cellcolor{green!25}40.9 & 39.9 & 33.3 \\ 
        $\min$ & -47.1 & -48.6 & -47.0 & -41.8 & \cellcolor{green!25}-20.9 & 6.7 & 7.2 & 7.2 & 7.2 & 4.9 \\ 
        $M$ & -10.0 & -10.8 & -12.0 & -10.8 & \cellcolor{green!25}-7.0 & 10.0 & \cellcolor{red!25}9.0 & 9.8 & \cellcolor{green!25}15.7 & 10.9 \\ 
        $\mu$ & -15.7 & -17.2 & -15.7 & -15.2 & \cellcolor{green!25}-10.2 & 12.9 & \cellcolor{red!25}9.0 & 17.4 & \cellcolor{green!25}18.6 & 13.0 \\ 
        $\sigma$ & 11.5 & 13.6 & 11.2 & 10.8 & \cellcolor{red!25}6.2 & 8.0 & \cellcolor{red!25}2.6 & \cellcolor{green!25}13.9 & 9.0 & 6.8 \\ 
\hline
    \end{tabular}
    \label{tab_comparison_drawdown}
\end{table}
\end{comment}

\begin{comment}
\begin{table}[!ht]
    \centering
    \caption{Annual Volatility of Investment Strategies}
    \medskip
    \begin{tabular}{l | rrr | rr | rr}
    \hline
\multirow{2}{*}{Year} & \multicolumn{3}{c|}{Buy-and-Hold}   
& \multicolumn{2}{c|}{Growth-vs-Value}  &  \multicolumn{2}{c}{Index-vs-Cash}  \\
  & S\&P & Growth & Value & GV-1W & GV-1L & MC-1W & MC-1L \\ \hline

2001 & 21.9 & \cellcolor{green!25}40.7 & 17.2 & 28.8 & 33.4 & \cellcolor{red!25}7.2 & 16.2 \\
2002 & 26.4 & \cellcolor{green!25}33.1 & 24.5 & 28.7 & 29.2 & \cellcolor{red!25}14.7 & 19.5 \\
2003 & 16.5 & \cellcolor{green!25}18.2 & 16.5 & 17.5 & 17.2 & 17.8 & \cellcolor{red!25}10.9 \\
2004 & 11.1 & 11.1 & 10.5 & 10.4 & 11.2 & \cellcolor{green!25}12.4 & \cellcolor{red!25}7.4 \\
2005 & \cellcolor{green!25}10.3 & 10.2 & 9.8 & 9.9 & 10.1 & 8.3 & \cellcolor{red!25}7.1 \\
2006 & 10.0 & \cellcolor{green!25}10.7 & 9.8 & 9.9 & 10.6 & 7.4 & \cellcolor{red!25}7.0 \\
2007 & 15.9 & 14.6 & \cellcolor{green!25}16.7 & 15.4 & 15.9 & 11.4 & \cellcolor{red!25}11.0 \\
2008 & \cellcolor{green!25}41.4 & 37.4 & 40.9 & 38.4 & 39.9 & \cellcolor{red!25}24.4 & 33.3 \\
2009 & 26.6 & 25.5 & \cellcolor{green!25}27.8 & 26.4 & 26.9 & \cellcolor{red!25}18.7 & 18.9 \\
2010 & 17.9 & \cellcolor{green!25}18.8 & 17.8 & 18.6 & 18.1 & \cellcolor{red!25}10.4 & 14.6 \\
2011 & 23.0 & 21.5 & \cellcolor{green!25}23.7 & 23.1 & 22.2 & \cellcolor{red!25}16.2 & 16.3 \\
2012 & 12.7 & 12.0 & \cellcolor{green!25}13.9 & 12.8 & 13.1 & \cellcolor{red!25}8.4 & 9.5 \\
2013 & \cellcolor{green!25}11.1 & 10.9 & 10.4 & 10.6 & 10.8 & 8.0 & \cellcolor{red!25}7.7 \\
2014 & 11.2 & \cellcolor{green!25}12.2 & 10.3 & 11.3 & 11.3 & \cellcolor{red!25}8.0 & 8.0 \\
2015 & 15.4 & \cellcolor{green!25}15.8 & 15.3 & 15.5 & 15.7 & \cellcolor{red!25}8.9 & 12.6 \\
2016 & 13.1 & 13.4 & 13.3 & 12.8 & \cellcolor{green!25}13.8 & 9.6 & \cellcolor{red!25}8.8 \\
2017 & 6.7 & 7.2 & 7.2 & \cellcolor{green!25}7.2 & 7.2 & \cellcolor{red!25}4.7 & 4.9 \\
2018 & 17.0 & \cellcolor{green!25}19.5 & 15.1 & 17.8 & 17.1 & \cellcolor{red!25}9.6 & 14.0 \\
2019 & 12.5 & 12.9 & 12.8 & 12.7 & \cellcolor{green!25}13.0 & \cellcolor{red!25}8.7 & 9.1 \\
2020 & 33.5 & 34.7 & 35.2 & 34.6 & \cellcolor{green!25}35.3 & \cellcolor{red!25}20.7 & 26.3 \\
2021 & 13.0 & \cellcolor{green!25}16.3 & 13.0 & 14.1 & 15.3 & \cellcolor{red!25}9.0 & 9.4 \\
2022 & 24.2 & \cellcolor{green!25}30.6 & 19.2 & 24.9 & 26.3 & \cellcolor{red!25}16.2 & 18.0 \\
2023 & 13.0 & 13.3 & 13.3 & 13.1 & \cellcolor{green!25}13.5 & \cellcolor{red!25}9.2 & 9.3 \\ \hline
$\max$ & 21.9 & \cellcolor{red!25}10.9 & \cellcolor{green!25}40.9 & 38.4 & 39.9 & 24.4 & 33.3 \\
$\min$ & 6.7 & 7.2 & 7.2 & \cellcolor{green!25} 7.2 & 7.2 & \cellcolor{red!25}4.7 & 4.9 \\
$M$ & 10.0 & \cellcolor{red!25}9.0 & 9.8 & 15.4 & \cellcolor{green!25}15.7 & 9.6 & 10.9 \\
$\mu$ & 12.9 & \cellcolor{red!25}9.0 & 17.4 & 18.0 & \cellcolor{green!25}18.6 & 11.7 & 13.0 \\
$\sigma$ & 8.0 & \cellcolor{red!25}2.6 & \cellcolor{green!25}13.9 & 8.5 & 9.0 & 5.0 & 6.8 \\
\hline
    \end{tabular}
    \label{tab_comparison_av}
\end{table}
\begin{table}[!ht]
    \centering
    \caption{TPR And TNR For Winner And Loser Strategies }
    \begin{tabular}{l | rrr| rrr | rrr| rrr}
    \hline
\multirow{3}{*}{Year} & \multicolumn{3}{c|}{GV-1W}   
& \multicolumn{3}{c|}{GV-1L}  & \multicolumn{3}{c|}{MC-1W} & \multicolumn{3}{c}{MC-1L}  \\
 & TPR & TNR & Ac& TPR & TNR & Ac& TPR & TNR &Ac & TPR & TNR &Ac \\
\hline
        2001 & 45 & 47 & \cellcolor{red!25}46 & 55 & 53 & \cellcolor{green!25}54 & 49 & 48 & 48 & 51 & 52 & 52 \\ 
        2002 & 31 & 40 & \cellcolor{red!25}36 & 69 & 60 & \cellcolor{green!25}64 & 39 & 45 & 42 & 61 & 55 & 58 \\ 
        2003 & 43 & 42 & \cellcolor{red!25}42 & 57 & 58 & \cellcolor{green!25}58 & 54 & 39 & 48 & 46 & 61 & 52 \\ 
        2004 & 42 & 46 & \cellcolor{red!25}44 & 58 & 54 & \cellcolor{green!25}56 & 59 & 43 & 52 & 41 & 57 & 48 \\ 
        2005 & 45 & 50 & \cellcolor{red!25}48 & 55 & 50 & \cellcolor{green!25}52 & 54 & 42 & 48 & 46 & 58 & 52 \\ 
        2006 & 39 & 54 & 47 & 61 & 46 & 53 & 59 & 47 & \cellcolor{green!25}54 & 41 & 53 & \cellcolor{red!25}46 \\ 
        2007 & 49 & 37 & \cellcolor{red!25}43 & 51 & 63 & \cellcolor{green!25}57 & 49 & 37 & \cellcolor{red!25}43 & 51 & 63 & \cellcolor{green!25}57 \\ 
        2008 & 47 & 48 & \cellcolor{red!25}47 & 53 & 52 & \cellcolor{green!25}53 & 50 & 51 & 51 & 50 & 49 & 49 \\ 
        2009 & 54 & 37 & \cellcolor{red!25}47 & 46 & 63 & \cellcolor{green!25}53 & 56 & 43 & 50 & 44 & 57 & 50 \\ 
        2010 & 46 & 46 & \cellcolor{red!25}46 & 54 & 54 & \cellcolor{green!25}54 & 57 & 41 & 50 & 43 & 59 & 50 \\ 
        2011 & 45 & 45 & \cellcolor{red!25}45 & 55 & 55 & \cellcolor{green!25}55 & 57 & 48 & 53 & 43 & 52 & 47 \\ 
        2012 & 35 & 44 & \cellcolor{red!25}40 & 65 & 56 & \cellcolor{green!25}60 & 59 & 50 & 55 & 41 & 50 & 45 \\ 
        2013 & 47 & 41 & \cellcolor{red!25}44 & 53 & 59 & \cellcolor{green!25}56 & 57 & 38 & 49 & 43 & 62 & 51 \\ 
        2014 & 53 & 41 & \cellcolor{red!25}48 & 47 & 59 & \cellcolor{green!25}52 & 56 & 36 & 48 & 44 & 64 & 52 \\ 
        2015 & 52 & 42 & \cellcolor{red!25}47 & 48 & 58 & \cellcolor{green!25}53 & 45 & 50 & 48 & 55 & 50 & 52 \\ 
        2016 & 51 & 49 & 50 & 49 & 51 & 50 & 46 & 35 & \cellcolor{red!25}41 & 54 & 65 & \cellcolor{green!25}59 \\ 
        2017 & 54 & 38 & 47 & 46 & 62 & 53 & 53 & 37 & \cellcolor{red!25}47 & 47 & 63 & \cellcolor{green!25}53 \\ 
        2018 & 58 & 45 & \cellcolor{green!25}53 & 42 & 55 & \cellcolor{red!25}47 & 56 & 49 & \cellcolor{green!25}53 & 44 & 51 & \cellcolor{red!25}47 \\ 
        2019 & 53 & 54 & 54 & 47 & 46 & 46 & 62 & 44 & \cellcolor{green!25}55 & 38 & 56 & \cellcolor{red!25}45 \\ 
        2020 & 62 & 45 & 55 & 38 & 55 & 45 & 51 & 33 & \cellcolor{red!25}43 & 49 & 67 & \cellcolor{green!25}57 \\ 
        2021 & 53 & 54 & \cellcolor{green!25}53 & 47 & 46 & \cellcolor{red!25}47 & 57 & 40 & 50 & 43 & 60 & 50 \\ 
        2022 & 43 & 54 & 49 & 57 & 46 & 51 & 45 & 57 & \cellcolor{green!25}51 & 55 & 43 & \cellcolor{red!25}49 \\ 
        2023 & 56 & 42 & 50 & 44 & 58 & 50 & 55 & 42 & \cellcolor{red!25}50 & 45 & 58 & \cellcolor{green!25}50 \\  \hline
        $\max$ & 62 & 54 & 55 & 69 & 63 & \cellcolor{green!25}64 & 62 & 57 & \cellcolor{red!25}55 & 61 & 67 & 59 \\ 
        $\min$ & 31 & 37 & \cellcolor{red!25}36 & 38 & 46 & 45 & 39 & 33 & 41 & 38 & 43 & \cellcolor{green!25}45 \\ 
        $M$ & 47 & 45 & \cellcolor{red!25}47 & 53 & 55 & \cellcolor{green!25}53 & 55 & 43 & 50 & 45 & 57 & 50 \\ 
        $\mu$  & 48 & 45 & \cellcolor{red!25}47 & 52 & 55 & \cellcolor{green!25}53 & 53 & 43 & 49 & 47 & 57 & 51 \\ 
        $\sigma$ & 7 & 5 & \cellcolor{green!25}4 & 7 & 5 & \cellcolor{green!25}4 & 6 & 6 & \cellcolor{red!25}4 & 6 & 6 & \cellcolor{red!25}4 \\ 
\hline
\end{tabular}
\end{table}
\end{comment}

\begin{comment}
\begin{table}[!h]
    \centering
    \caption{Annual Growth Comparison for Different $k$ Days}
    \medskip
    \begin{tabular}{l | cc| cc| cc| cc}
    \hline
\multirow{2}{*}{Year} & \multicolumn{2}{c|}{$k=1$}   
& \multicolumn{2}{c|}{$k=3$}  & \multicolumn{2}{c|}{$k=5$}   
& \multicolumn{2}{c}{$k=7$}   \\
 & GV-1W & GV-1L & GV-3W & GV-3L & GV-5W & GV-5L & GV-7W & GV-7L \\
\hline
                2001 & 60 & 117 & 80 & 88 & 76 & 92 & 83 & 84  \\ 
        2002 & 25 & 157 & 36 & 108 & 39 & 99 & 40 & 99  \\ 
        2003 & 26 & 238 & 43 & 146 & 46 & 137 & 47 & 132  \\ 
        2004 & 27 & 281 & 45 & 166 & 49 & 151 & 51 & 147  \\ 
        2005 & 26 & 308 & 47 & 173 & 51 & 158 & 54 & 149  \\ 
        2006 & 29 & 367 & 50 & 213 & 59 & 181 & 62 & 174  \\ 
        2007 & 29 & 413 & 51 & 238 & 60 & 200 & 61 & 198  \\ 
        2008 & 16 & 303 & 26 & 185 & 30 & 158 & 33 & 146  \\ 
        2009 & 19 & 408 & 32 & 242 & 37 & 210 & 40 & 193  \\ 
        2010 & 21 & 501 & 34 & 299 & 40 & 257 & 47 & 222  \\ 
        2011 & 20 & 547 & 33 & 327 & 39 & 276 & 47 & 230  \\ 
        2012 & 19 & 745 & 34 & 418 & 41 & 351 & 51 & 280  \\ 
        2013 & 24 & 1,045 & 43 & 578 & 52 & 479 & 66 & 378  \\ 
        2014 & 26 & 1,222 & 47 & 686 & 60 & 539 & 74 & 434  \\ 
        2015 & 26 & 1,278 & 47 & 700 & 59 & 553 & 73 & 452  \\ 
        2016 & 29 & 1,419 & 52 & 790 & 66 & 620 & 84 & 487  \\ 
        2017 & 35 & 1,720 & 63 & 955 & 81 & 745 & 102 & 591  \\ 
        2018 & 34 & 1,604 & 63 & 864 & 79 & 697 & 100 & 549  \\ 
        2019 & 47 & 2,016 & 86 & 1,097 & 107 & 882 & 128 & 739  \\ 
        2020 & 60 & 2,137 & 89 & 1,430 & 117 & 1,097 & 137 & 932  \\ 
        2021 & 85 & 2,486 & 121 & 1,743 & 176 & 1,199 & 210 & 1,004  \\ 
        2022 & 65 & 2,161 & 89 & 1,592 & 138 & 1,024 & 173 & 813  \\ 
        2023 & 85 & 2,642 & 119 & 1,878 & 182 & 1,234 & 234 & 957  \\ \hline
%        TPR & \cellcolor{red!25}44.1\% &\cellcolor{green!25} 55.9\% & 44.2\% & 55.8\% & 45.0\% & 55.0\% & 44.4\% & 55.6\% & 45.7\% & 54.3\% \\ 
%        TNR & \cellcolor{green!25}54.7\% & \cellcolor{red!25}45.3\% & 54.2\% & 45.8\% & 53.5\% & 46.5\% & 50.3\% & 49.7\% & 50.3\% & 49.7\% \\ \hline
    \end{tabular}
    \label{tab_comparison_wl}
\end{table}
\end{comment}

\begin{comment}

\begin{table}[!h]
    \centering
    \caption{Comparison of Annual Return  for Different $k$ }
    \medskip
    \begin{tabular}{l | rr| rr| rr| rr| rr}
    \hline
\multirow{2}{*}{Year} & \multicolumn{2}{c|}{$k=1$ day}   
& \multicolumn{2}{c|}{$k=3$ day}  & \multicolumn{2}{c|}{$k=5$ days}   
& \multicolumn{2}{c|}{$k=7$ days}   \\
 & GV-1W & GV-1L & GV-3W & GV-3L & GV-5W & GV-5L & GV-7W & GV-7L \\
\hline
        2001 & \cellcolor{red!25}-40.2 & \cellcolor{green!25}17.0 & -20.2 & -12.4 & -23.7 & -8.3 & -17.0 & -15.8 \\ 
        2002 & \cellcolor{red!25}-58.5 & \cellcolor{green!25}34.2 & -54.7 & 22.8 & -48.3 & 7.6 & -52.4 & 17.0 \\ 
        2003 & \cellcolor{red!25}6.1 & \cellcolor{green!25}51.4 & 18.8 & 35.3 & 15.7 & 39.0 & 19.7 & 34.2 \\ 
        2004 & \cellcolor{red!25}0.7 & \cellcolor{green!25}18.4 & 4.2 & 14.4 & 8.1 & 10.2 & 7.4 & 11.0 \\ 
        2005 & \cellcolor{red!25}-1.2 & \cellcolor{green!25}9.6 & 4.0 & 4.1 & 3.6 & 4.6 & 6.4 & 1.8 \\ 
        2006 & 11.4 & 19.0 & \cellcolor{red!25}7.9 & \cellcolor{green!25}22.8 & 15.8 & 14.4 & 14.1 & 16.2 \\ 
        2007 & -0.2 & 12.5 & 0.5 & 11.7 & 1.3 & 10.9 & \cellcolor{red!25}-1.8 & \cellcolor{green!25}14.3 \\ 
        2008 & -45.7 & -26.5 & -48.8 & -22.1 & \cellcolor{red!25}-49.4 & \cellcolor{green!25}-21.2 & -45.9 & -26.3 \\ 
        2009 & \cellcolor{red!25}19.3 & \cellcolor{green!25}34.4 & 22.6 & 30.9 & 20.9 & 32.6 & 21.6 & 31.9 \\ 
        2010 & 9.4 & 22.7 & \cellcolor{red!25}8.7 & \cellcolor{green!25}23.5 & 9.4 & 22.7 & 16.7 & 15.0 \\ 
        2011 & -4.9 & 9.2 & \cellcolor{red!25}-5.0 & \cellcolor{green!25}9.3 & -3.4 & 7.6 & 0.3 & 3.5 \\ 
        2012 & \cellcolor{red!25}-1.8 & \cellcolor{green!25}36.3 & 4.8 & 27.7 & 5.3 & 27.1 & 9.7 & 22.0 \\ 
        2013 & \cellcolor{red!25}24.6 & \cellcolor{green!25}40.3 & 26.3 & 38.3 & 28.2 & 36.2 & 29.6 & 34.9 \\ 
        2014 & 10.2 & 16.9 & \cellcolor{red!25}8.6 & \cellcolor{green!25}18.6 & 14.5 & 12.5 & 12.2 & 14.8 \\ 
        2015 & \cellcolor{red!25}-2.7 & \cellcolor{green!25}4.6 & -0.3 & 2.1 & -0.9 & 2.7 & -2.3 & 4.2 \\ 
        2016 & 12.6 & 11.1 & 10.8 & 12.9 & 11.6 & 12.1 & \cellcolor{green!25}16.0 & \cellcolor{red!25}7.8 \\ 
        2017 & 21.1 & 21.2 & 21.5 & 20.8 & \cellcolor{green!25}22.2 & \cellcolor{red!25}20.2 & 21.1 & 21.3 \\ 
        2018 & -2.5 & -6.7 & 0.4 & -9.5 & -2.8 & -6.5 & -2.3 & -7.0 \\ 
        2019 & \cellcolor{green!25}37.1 & \cellcolor{red!25}25.7 & 35.8 & 26.9 & 36.1 & 26.6 & 28.2 & 34.5 \\ 
        2020 & 27.7 & 6.0 & \cellcolor{red!25}3.8 & \cellcolor{green!25}30.4 & 8.8 & 24.4 & 7.3 & 26.2 \\ 
        2021 & 41.7 & 16.3 & 35.3 & 21.9 & 50.8 & 9.3 & \cellcolor{green!25}53.0 & \cellcolor{red!25}7.7 \\ 
        2022 & -23.1 & -13.1 & \cellcolor{red!25}-26.8 & \cellcolor{green!25}-8.7 & -21.7 & -14.6 & -17.4 & -19.0 \\ 
        2023 & 30.0 & 22.2 & 34.7 & 18.0 & 31.9 & 20.5 & \cellcolor{green!25}35.0 & \cellcolor{red!25}17.7 \\ 
        \hline
        $\max$ & 41.7 & 51.4 & 35.8 & 38.3 & 50.8 & 39.0 & \cellcolor{green!25}53.0 & \cellcolor{red!25}34.9 \\ 
        $\min$ & \cellcolor{red!25}-58.5 & -26.5 & -54.7 & -22.1 & -49.4 & \cellcolor{green!25}-21.2 & -52.4 & -26.3 \\ 
        $M$ & 6.1 & 17.0 & \cellcolor{red!25}4.8 & \cellcolor{green!25}18.6 & 8.8 & 12.1 & 9.7 & 14.8 \\ 
        $\mu$ & \cellcolor{red!25}3.1 & \cellcolor{green!25}16.6 & 4.0 & 14.8 & 5.8 & 12.6 & 6.9 & 11.6 \\ 
        $\sigma$ & \cellcolor{green!25}25.4 & 17.4 & 23.6 & 16.0 & 24.1 &\cellcolor{red!25} 15.6 & 23.8 & 16.8 \\ 
        IRR & \cellcolor{red!25}-0.7 &\cellcolor{green!25} 15.3 & 0.8 & 13.6 & 2.6 & 11.5 & 3.8 & 10.3 \\ \hline
    \end{tabular}
    \label{tab_comparison_wl}
\end{table}

\end{comment}
\begin{comment}

\begin{table}[!ht]
    \centering
    \caption{Annual Volatility for Different $k$}
    \medskip
\begin{tabular}{l| rr | rr| rr| rr| rr}
\hline
\multirow{2}{*}{Year} & \multicolumn{2}{c|}{1 day}   
& \multicolumn{2}{c|}{3 day}  & \multicolumn{2}{c|}{5 days}   
& \multicolumn{2}{c|}{7 day}  \\
 & GV-1W & GV-1L & GV-3W & GV-3L & GV-5W & GV-5L & GV-7W & GV-7L  \\
\hline
        2001 & 28.8 & 33.4 & 25.7 & 35.9 & 25.3 & 36.2 & \cellcolor{red!25}24.0 & \cellcolor{green!25}37.1 \\ 
        2002 & 28.7 & 29.2 & 27.8 & 30.1 & 27.4 & 30.6 & 27.3 & 30.6 \\ 
        2003 & 17.5 & 17.2 & 17.6 & 17.1 & 17.7 & 17.1 & 17.9 & 16.8 \\ 
        2004 & \cellcolor{red!25}10.4 & \cellcolor{green!25}11.2 & 10.6 & 11.1 & 10.9 & 10.8 & 10.9 & 10.8 \\ 
        2005 & 9.9 & 10.1 & \cellcolor{red!25}9.7 & \cellcolor{green!25}10.3 & 10.0 & 10.0 & 10.0 & 10.0 \\ 
        2006 & 9.9 & 10.6 & 9.9 & 10.7 & \cellcolor{red!25}9.8 & \cellcolor{green!25}10.7 & 9.9 & 10.6 \\ 
        2007 & \cellcolor{red!25}15.4 & \cellcolor{green!25}15.9 & 15.7 & 15.6 & 15.6 & 15.7 & 15.4 & 15.8 \\ 
        2008 & 38.4 & 39.9 & \cellcolor{red!25}38.4 & \cellcolor{green!25}39.9 & 39.9 & 38.4 & 39.5 & 38.8 \\ 
        2009 & 26.4 & 26.9 & 26.8 & 26.5 & \cellcolor{red!25}26.2 & \cellcolor{green!25}27.1 & 26.3 & 27.0 \\ 
        2010 & \cellcolor{green!25}18.6 & \cellcolor{red!25}18.1 & 18.5 & 18.2 & 18.3 & 18.4 & 18.1 & 18.5 \\ 
        2011 & \cellcolor{green!25}23.1 & \cellcolor{red!25}22.2 & 22.9 & 22.4 & 22.8 & 22.5 & 22.8 & 22.5 \\ 
        2012 & 12.8 & 13.1 & 12.9 & 13.1 & \cellcolor{red!25}12.8 & \cellcolor{green!25}13.2 & 13.0 & 13.0 \\ 
        2013 & 10.6 & 10.8 & \cellcolor{green!25}10.8 & \cellcolor{red!25}10.5 & 10.7 & 10.7 & 10.7 & 10.7 \\ 
        2014 & 11.3 & 11.3 & \cellcolor{green!25}11.6 & \cellcolor{red!25}11.0 & 11.2 & 11.4 & 11.3 & 11.3 \\ 
        2015 & 15.5 & 15.7 & 15.3 & 15.9 & \cellcolor{red!25}15.1 & \cellcolor{green!25}16.1 & 15.3 & 15.9 \\ 
        2016 & \cellcolor{red!25}12.8 & \cellcolor{green!25}13.8 & 13.2 & 13.4 & 13.4 & 13.3 & 13.5 & 13.2 \\ 
        2017 & 7.2 & 7.2 & \cellcolor{red!25}6.9 & \cellcolor{green!25}7.5 & 7.1 & 7.3 & 7.2 & 7.2 \\ 
        2018 & 17.8 & 17.1 & 17.4 & 17.4 & 17.4 & 17.5 & \cellcolor{red!25}17.0 & \cellcolor{green!25}17.8 \\ 
        2019 & 12.7 & 13.0 & 12.5 & 13.2 & \cellcolor{red!25}12.5 & \cellcolor{green!25}13.2 & 12.8 & 12.9 \\ 
        2020 & 34.6 & 35.3 & 34.4 & 35.5 & 34.3 & 35.6 & \cellcolor{red!25}34.3 & \cellcolor{green!25}35.6 \\ 
        2021 & 14.1 & 15.3 & 13.9 & 15.5 & 13.7 & 15.6 & 14.0 & 15.4 \\ 
        2022 & 24.9 & 26.3 & 24.4 & 26.7 & 23.8 & 27.2 & \cellcolor{red!25}23.0 & \cellcolor{green!25}28.0 \\ 
        2023 & 13.1 & 13.5 & \cellcolor{red!25}12.7 & \cellcolor{green!25}13.9 & 13.1 & 13.5 & 13.2 & 13.4 \\ \hline
        $\max$ & 38.4 & 39.9 & \cellcolor{red!25}38.4 & \cellcolor{green!25}39.9 & 39.9 & 38.4 & 39.5 & 38.8 \\ 
        $\min$ & 7.2 & 7.2 & \cellcolor{red!25}6.9 & \cellcolor{green!25}7.5 & 7.1 & 7.3 & 7.2 & 7.2 \\ 
        $M$ & 15.4 & 15.7 & 15.3 & 15.6 & \cellcolor{red!25}15.1 & 15.7 & 15.3 & \cellcolor{green!25}15.8 \\ 
        $\mu$ & 18.0 & 18.6 & 17.8 & 18.8 & 17.8 & 18.8 & \cellcolor{red!25}17.7 & \cellcolor{green!25}18.8 \\ 
        $\sigma$ & 8.5 & 9.0 & 8.3 & 9.3 & 8.4 & 9.2 & \cellcolor{red!25}8.2 & \cellcolor{green!25}9.4 \\ 

\hline
\end{tabular}
\end{table}
\end{comment}

\begin{comment}

\begin{table}[!h]
    \centering
    \caption{Comparison of Maximum Drawdowns for Different $k$}
    \medskip
    \begin{tabular}{l | rr| rr| rr| rr| rr}
    \hline
\multirow{2}{*}{Year} & \multicolumn{2}{c|}{$k=1$ day}   
& \multicolumn{2}{c|}{$k=3$ day}  & \multicolumn{2}{c|}{$k=5$ days}   
& \multicolumn{2}{c|}{$k=7$ days}   \\
 & GV-1W & GV-1L & GV-3W & GV-3L & GV-5W & GV-5L & GV-7W & GV-7L \\
\hline

        2001 & \cellcolor{red!25}-53.7 & -32.0 & -35.5 & -36.9 & -33.6 & -39.8 & \cellcolor{green!25}-28.3 & -41.8 \\ 
        2002 & \cellcolor{red!25}-58.9 & -26.4 & -55.3 & -17.9 & -50.9 & -25.8 & -55.2 & \cellcolor{green!25}-16.4 \\ 
        2003 & \cellcolor{red!25}-19.1 & -13.9 & -14.6 & -13.5 & -16.0 & -12.3 & -16.0 & -12.2 \\ 
        2004 & \cellcolor{red!25}-13.7 & \cellcolor{green!25}-6.5 & -10.5 & -7.9 & -9.3 & -8.9 & -8.3 & -9.4 \\ 
        2005 & -8.0 & \cellcolor{green!25}-5.6 & -8.4 & -6.4 & \cellcolor{red!25}-9.1 & -6.2 & -8.0 & -6.9 \\ 
        2006 & -8.0 & -7.7 & \cellcolor{red!25}-8.9 & \cellcolor{green!25}-6.9 & -7.5 & -8.8 & -8.4 & -7.9 \\ 
        2007 & -11.1 & -10.1 & \cellcolor{red!25}-12.6 & \cellcolor{green!25}-8.9 & -10.8 & -10.3 & -12.0 & -9.2 \\ 
        2008 & -53.6 & -41.8 & -56.8 & -39.6 & \cellcolor{red!25}-58.6 & \cellcolor{green!25}-34.1 & -55.3 & -38.9 \\ 
        2009 & -25.4 & -28.4 & -25.5 & -28.2 & -26.4 & -27.4 & -26.6 & -27.1 \\ 
        2010 & -16.5 & -14.3 & -16.9 & -13.9 & -16.4 & -14.4 & -14.9 & -15.9 \\ 
        2011 & \cellcolor{red!25}-22.7 & \cellcolor{green!25}-16.8 & -22.1 & -17.9 & -19.2 & -19.2 & -18.9 & -19.1 \\ 
        2012 & \cellcolor{red!25}-14.7 & \cellcolor{green!25}-6.3 & -12.8 & -7.0 & -12.7 & -7.2 & -11.9 & -8.0 \\ 
        2013 & \cellcolor{red!25}-7.1 & \cellcolor{green!25}-4.3 & -6.9 & -4.9 & -6.5 & -4.7 & -5.9 & -5.1 \\ 
        2014 & -8.2 & -7.4 & -8.2 & -6.9 & \cellcolor{red!25}-8.4 & \cellcolor{green!25}-6.4 & -6.8 & -8.0 \\ 
        2015 & -12.5 & \cellcolor{red!25}-13.2 & -13.1 & -12.6 & -12.7 & \cellcolor{green!25}-11.9 & -12.8 & -12.6 \\ 
        2016 & \cellcolor{green!25}-7.3 & \cellcolor{red!25}-10.8 & -9.5 & -8.5 & -10.1 & -8.1 & -8.5 & -9.5 \\ 
        2017 & -3.7 & -3.3 & -2.6 & -3.3 & \cellcolor{green!25}-2.3 & -4.1 & -3.0 & -3.9 \\ 
        2018 & -19.5 & -19.5 & -18.3 & -21.1 & -18.1 & -20.6 & -18.5 & -20.0 \\ 
        2019 & -7.3 & -7.3 & -6.1 & -8.0 & -6.4 & -7.7 & \cellcolor{red!25}-8.4 & \cellcolor{green!25}-6.1 \\ 
        2020 & -32.3 & -36.0 & -33.9 & -34.4 & -32.9 & -35.4 & -32.5 & -35.8 \\ 
        2021 & -5.7 & -7.9 & -5.7 & -7.1 & \cellcolor{green!25}-4.6 & -8.7 & -6.1 & \cellcolor{red!25}-10.1 \\ 
        2022 & -31.2 & -19.5 & \cellcolor{red!25}-32.4 & -20.3 & -27.9 & -25.3 & -23.8 & -28.1 \\ 
        2023 & -10.6 & -9.8 & \cellcolor{green!25}-7.7 & \cellcolor{red!25}-12.2 & -8.5 & -11.9 & -8.3 & -11.8 \\ 
\hline
        $\max$ & -3.7 & -3.3 & -2.6 & -3.3 & \cellcolor{green!25}-2.3 & -4.1 & -3.0 & -3.9 \\ 
        $\min$ & \cellcolor{red!25}-58.9 & -41.8 & -56.8 & -39.6 & -58.6 & -39.8 & -55.3 & -41.8 \\ 
        $M$ & \cellcolor{red!25}-13.7 & \cellcolor{green!25}-10.8 & -12.8 & -12.2 & -12.7 & -11.9 & -12.0 & -11.8 \\ 
        $\mu$ & \cellcolor{red!25}-19.6 & -15.2 & -18.5 & \cellcolor{green!25}-15.0 & -17.8 & -15.6 & -17.3 & -15.8 \\ 
        $\sigma$ & \cellcolor{green!25}16.2 & 10.8 & 15.0 & \cellcolor{red!25}10.6 & 14.6 & 10.7 & 14.3 & 11.1 \\ 

\hline
    \end{tabular}
\end{table}
\end{comment}







\begin{comment}

 Ayala et al. \cite{tech-Ayala2021} combine TEMA and MACD with machine learning to optimize trading strategies.

 Gerlein et al. \cite{tech-Gerlein2016} assess the profitability of machine learning models in FOREX trading.

 Wong et al. \cite{tech-Wong2023} explore high-frequency stock price prediction using simple ML models.

 Patel et al. \cite{tech-Patel2015} compare ML models for predicting stock price movements using discrete data.

 Pradip et al. \cite{tech-Pradip2018} integrate fundamental and technical analysis for stock price prediction.

 Beg et al. \cite{tech-Beg2019} use technical indicators and ML models to predict stock prices.

Jagadisha et al. \cite{tech-Jagadisha2022} focus on ML model effectiveness in predicting stock price movements.

 Ndikum \cite{tech-Ndikum2020} shows ML models outperform CAPM in financial forecasting.

 Chen \cite{tech-Chen2020} investigates LSTM, CNN, and SVR models for stock price prediction.

 Bitvai and Cohn \cite{tech-Bitvai2015} develop a stochastic trading algorithm improving over buy-and-hold.

 Tsantekidis et al. \cite{tech-Tsantekidis2020} use deep reinforcement learning to create a trading agent with improved profit, Sharpe ratio, and maximum drawdown.

 Joiner et al. \cite{tech-Joiner2022} review the state of the art in ML for algorithmic trading, focusing on sentiment and technical analysis.

Pasupulety et al. \cite{tech-Pasupulety2019} propose an ensemble model combining SVM and Random Forest for stock price prediction with sentiment analysis.

 Wenqing et al. \cite{tech-yao2022stock} use ML algorithms like neural networks and SVMs to predict stock prices, addressing randomness and nonlinearity.

 Nousi et al. \cite{tech-Nousi2018} predict future price movements using limit order book data and ML-extracted features.

 Ntakaris et al. \cite{tech-Ntakaris2018} focus on short-term mid-price movement prediction using technical indicators and quantitative analysis.

 Hsu et al. \cite{tech-Hsu2016} benchmark ML methods against econometric methods for financial forecasting, finding ML more accurate.

Gu et al. \cite{tech-Gu2018} compare ML methods for measuring asset risk premiums, identifying trees and neural networks as top performers.

 Wang et al. \cite{tech-Wang2018} propose a 1D CNN model for financial market prediction, outperforming traditional indicators.

 Zarkias et al. \cite{tech-Zarkias2019} introduce a novel price trailing method for trading using deep reinforcement learning.

Agrawal et al. \cite{tech-Agrawal2019} propose a predictive model using LSTM and adaptive stock indicators, achieving higher accuracy.

 Dash and Dash \cite{tech-Dash2016} propose a decision support system using CEFLANN for trading decisions.

 Buachuen and Kantavat \cite{tech-Buachuen2023} integrate LSTM-CNN hybrids with technical analysis for stock prediction.

Zhong and Enke \cite{tech-Zhong2019} use DNNs and ANNs with PCA-transformed datasets for stock prediction.

 Choudhry and Garg \cite{tech-Choudhry2008} propose a hybrid GA and SVM system for stock prediction.

 Meesad and Boonmatham \cite{tech-Meesad2023} combine NLP and technical analysis for high-accuracy trading signals.

Oyewola et al. \cite{tech-Oyewola2019} apply ML methods for predicting Nigerian stock returns.

 Pholsri \cite{tech-Pholsri2023} develop a hybrid model for intraday stock trading using BiLSTM and CNN.

 Padhi et al. \cite{tech-Padhi2022} propose a framework combining portfolio construction and ML for stock prediction.

Van et al. \cite{tech-Van2023} evaluate deep learning and reinforcement learning for trading signals.

Yu et al. \cite{tech-Yu2023} propose an intelligent stock trading system using deep learning models.

 Chavarnakul and Enke \cite{tech-Chavarnakul2009} develop a hybrid system for trading using neural networks and genetic algorithms.

 Grigoryan \cite{tech-Grigoryan2017} propose a model using ML and statistical analysis for stock trend prediction.

 Karthik \cite{tech-Karthik2023} explore ML and algorithmic trading for risk management and stock trend prediction.

 Lumoring et al. \cite{tech-Lumoring2023} review ML techniques for stock prediction, highlighting LSTM's accuracy.

 Mndawe et al. \cite{tech-Mndawe2022} integrate sentiment analysis and ML for stock price prediction.

 B. K et al. \cite{tech-BK2023} combine sentiment analysis with technical indicators for stock trend prediction.

 Mahfooz et al. \cite{tech-Mahfooz2022} use LSTM with technical indicators for improved stock trend prediction.

 Zhang et al. \cite{tech-Zhang2021} use a hybrid deep learning model combining CNN and LSTM to predict stock prices.

Park and Shin \cite{tech-Park2019} investigate deep neural networks with technical analysis indicators for stock prediction.

 Kim and Won \cite{tech-Kim2018} develop an ensemble model integrating ML algorithms with technical indicators for stock prediction.

 Khan et al. \cite{tech-khan2023performance} compare various ML models for stock market prediction, using confusion matrices to evaluate performance.
\end{comment}

\begin{comment}
In the past decades, the prediction of movements in stock markets has been one of the most challenging problems for academicians and practitioners. That is, the inherent volatility and multifactorial nature of the financial markets, with many variables interacting in unpredictable ways, make this endeavor one of the most daunting tasks human intelligence is ever likely to engage in. They offer a framework for the analysis of market dynamics, but the subtleties of markets escape the models generally---a fact creating the search for more sophisticated computation techniques \cite{lara2021,saivijayalakshmi2021}. These include LSTM models, a class of artificial neural networks meant to deal with sequence problems, as an effective tool in financial forecasting \cite{shukla2023,chen2023}. Specifically, the uniqueness of LSTM is in its ability to model long-term dependencies, which is critical in the analysis of financial time series data \cite{ahmed2022,huang2019}. This enables them to retain information for a very long time and thus helps discover patterns and tendencies---an advantage in predicting stock market movements \cite{nguyen2022,mehra2023}. The paper attempts to assess how well LSTM models predict the movements in the S{\&}P-500, a barometer for the US stock market. Importantly, this paper sets out to demonstrate the efficacy of LSTMs in devising investment strategies that could surpass the benchmark S{\&}P 500 index. This involves the processing and analysis of financial data, harnessing the power of LSTM to extract market signals at the subsurface level. Therefore, the work will contribute to an open debate about how to improve predictions and hence generate wealth through better investment strategies for the analyses of financial markets \cite{joshi2022,yusuf2022}. The objective is to expand the work of LSTM-based models and applications by financial communities, mainly in stock price prediction, volatility, and market sentiment analysis \cite{vishwakarma2022,diqi2022,hu2021}, providing the tools needed for them to navigate the complexities within the stock market with better conviction.
\end{comment}




\section{Growth-Value vs. Market-Cash Strategies}
Growth investing targets companies likely to increase earnings, especially in emerging markets \citep{albuquerque2013}. Value investing focuses on undervalued stocks, which can yield higher returns due to behavioral finance and lower management costs \citep{chan2004}. Both strategies follow the Fama-French model, which explains returns using market risk, size, and value factors \citep{fama1993}. Market timing strategies adjust asset allocation based on market cycles. However, they are risky due to the challenges of predicting market changes \citep{sorensen2008}. Market timing can be volatile and risky \citep{bollen2001}. These strategies demand accurate timing and extensive market knowledge, which differs significantly from the basic principles of growth and value investing.

\subsection{Winners and Losers Strategies}
Investment strategies can focus on winners or losers. Jegadeesh et al. \citep{jegadeesh1993} noted that strategies focus on winners investing in assets with past gains to leverage momentum. Recent studies confirm these insights. Assogbavi et al. \citep{assogbavi2011} showed that Momentum Investment Strategies are more effective than Contrarian Strategies. These strategies buy past winners and sell past losers. Market conditions also influence these strategies. Low market impact and risk-benefit trend-following strategies \citep{xu2022}. Conversely, high market impact and risk favor trend-rejecting strategies.
